
% This LaTeX was auto-generated from MATLAB code.
% To make changes, update the MATLAB code and republish this document.

\documentclass{standalone}
\usepackage{graphicx}
\usepackage{listings}
\usepackage{xcolor}
\usepackage{textcomp}
\usepackage[framed, numbered]{matlab-prettifier}

\sloppy
\definecolor{lightgray}{gray}{0.5}
\setlength{\parindent}{0pt}

\begin{document}

    
    
\section*{plotDipoleMagnet}

\begin{par}
Plot dipole magnet which approximate a spherical magnet in its far field.
\end{par} \vspace{1em}

\subsection*{Contents}

\begin{itemize}
\setlength{\itemsep}{-1ex}
   \item Syntax
   \item Description
   \item Examples
   \item Input Arguments
   \item Output Arguments
   \item Requirements
   \item See Also
\end{itemize}


\subsection*{Syntax}

\begin{lstlisting}[style=Matlab-editor]
plotDipoleMagnet()
\end{lstlisting}


\subsection*{Description}

\begin{par}
plotDipoleMagnet() load dipole constants from config.mat and construct magnet in its rest position in x and z layer for y = 0.
\end{par} \vspace{1em}


\subsection*{Examples}

\begin{lstlisting}[style=Matlab-editor]
plotDipoleMagnet();
\end{lstlisting}


\subsection*{Input Arguments}

\begin{par}
\textbf{None}
\end{par} \vspace{1em}


\subsection*{Output Arguments}

\begin{par}
\textbf{None}
\end{par} \vspace{1em}


\subsection*{Requirements}

\begin{itemize}
\setlength{\itemsep}{-1ex}
   \item Other m-files: generateDipoleRotationMoments.m, computeDipoleH0Norm.m,   computeDipoleHField
   \item Subfunctions: none
   \item MAT-files required: data/config.mat
\end{itemize}


\subsection*{See Also}

\begin{itemize}
\setlength{\itemsep}{-1ex}
   \item \begin{verbatim}quiver\end{verbatim}
   \item \begin{verbatim}imagesc\end{verbatim}
   \item \begin{verbatim}streamslice\end{verbatim}
\end{itemize}
\begin{par}
Created on November 20. 2020 by Tobias Wulf. Copyright Tobias Wulf 2020.
\end{par} \vspace{1em}
\begin{par}

\end{par} \vspace{1em}
\begin{lstlisting}[style=Matlab-editor]
function plotDipoleMagnet()
    try
        % load dataset path and dataset content into function workspace
        load('config.mat', 'PathVariables', 'DipoleOptions');
%         close all;
    catch ME
        rethrow(ME)
    end

    % figure save path for different formats
    %%%%%%%%%%%%%%%%%%%%%%%%%%%%%%%%%%%%%%%%%%%%%%%%%%%%%%%%%%%%%%%%%%%%%%%%%%%%
    %%%%%%%%%%%%%%%%%%%%%%%%%%%%%%%%%%%%%%%%%%%%%%%%%%%%%%%%%%%%%%%%%%%%%%%%%%%%
    figFilename = 'dipole_magnet';
    figPath = fullfile(PathVariables.saveImagesPath, figFilename);

    % load needed data from dataset in to local variables for better handling
    %%%%%%%%%%%%%%%%%%%%%%%%%%%%%%%%%%%%%%%%%%%%%%%%%%%%%%%%%%%%%%%%%%%%%%%%%%%%
    %%%%%%%%%%%%%%%%%%%%%%%%%%%%%%%%%%%%%%%%%%%%%%%%%%%%%%%%%%%%%%%%%%%%%%%%%%%%
    % Radius in mm of magnetic sphere in which the magnetic dipole is centered.
    % So it can be seen as z-offset to the sensor array.
    rsp = DipoleOptions.sphereRadius;

    % H-field magnitude to multiply of generated and relative normed dipole
    Hmag = DipoleOptions.H0mag;

    % Distance in zero position of the spherical magnet in which is imprinted
    z0 = DipoleOptions.z0;

    % Magnetic moment magnitude attach rotation to the dipole field
    m0 = DipoleOptions.M0mag;

    % clear dataset all loaded
    clear DipoleOptions;

    % set construction dipole magnet, all length in mm and areas mm^2
    %%%%%%%%%%%%%%%%%%%%%%%%%%%%%%%%%%%%%%%%%%%%%%%%%%%%%%%%%%%%%%%%%%%%%%%%%%%%
    %%%%%%%%%%%%%%%%%%%%%%%%%%%%%%%%%%%%%%%%%%%%%%%%%%%%%%%%%%%%%%%%%%%%%%%%%%%%
    % number of samples for good looking
    nSamples = 501;
    % slice in view for quiver, every 25th point
    slice = 25:25:nSamples-25;
    % grid edge of meshgrid, square grid
    xz = 15;
    % y layer in coordinate system
    y = 0;
    % orientat of magnet along z axes
    pz = pi/2:0.01:3*pi/2;
    % distances magnet surface to display in plot
    zd = -rsp:-z0:-xz;
    xd = zeros(1, length(zd));
    % scale grid to simulate
    x = linspace(-xz, xz, nSamples);
    z = linspace(xz, -xz, nSamples);
    [X, Z, Y] = meshgrid(x, z, y);

    % compute dipole and fetch to far field to approximate a sperical magnet
    %%%%%%%%%%%%%%%%%%%%%%%%%%%%%%%%%%%%%%%%%%%%%%%%%%%%%%%%%%%%%%%%%%%%%%%%%%%%
    %%%%%%%%%%%%%%%%%%%%%%%%%%%%%%%%%%%%%%%%%%%%%%%%%%%%%%%%%%%%%%%%%%%%%%%%%%%%
    % generate dipole moment for 0
    m = generateDipoleRotationMoments(m0, 1);
    % compute H-field norm factor imprieng H magnitude on dipole, rest position
    H0norm = computeDipoleH0Norm(Hmag, m, [0; 0 ;-(z0 + rsp)]);
    % compute dipole H-field for rest position in y = 0 layer
    H = computeDipoleHField(X, Y, Z, m, H0norm);
    % calculate magnitudes for each point in the grid
    Habs = reshape(sqrt(sum(H.^2, 1)), nSamples, nSamples);
    % split H-field in componets and reshape to meshgrid
    Hx = reshape(H(1,:), nSamples, nSamples) ./ Habs;
    Hy = reshape(H(2,:), nSamples, nSamples) ./ Habs;
    Hz = reshape(H(3,:), nSamples, nSamples) ./ Habs;
    % exculde value within the spherical magnet, < rsp
    innerField = X.^2 + Z.^2 <= rsp.^2;
    Habs(innerField) = NaN;
    % find relevant magnitudes at anounced distances
    Hd = interp2(X, Z, Habs, xd, zd, 'nearest', NaN);


    % figure dipole magnet
    %%%%%%%%%%%%%%%%%%%%%%%%%%%%%%%%%%%%%%%%%%%%%%%%%%%%%%%%%%%%%%%%%%%%%%%%%%%%
    %%%%%%%%%%%%%%%%%%%%%%%%%%%%%%%%%%%%%%%%%%%%%%%%%%%%%%%%%%%%%%%%%%%%%%%%%%%%
    fig = figure('Name', 'Dipole Magnet');

    % plot magnitude as colormap
    imagesc(x, z, log10(Habs), 'AlphaData', 1);
    set(gca, 'YDir', 'normal');
    colormap('jet');
    shading flat;

    % set colorbar to log10 scaling of map
    cb = colorbar;
    cb.Label.String = '$log_{10}(|H|)$ in kA/m';
    cb.TickLabelInterpreter = 'latex';
    cb.Label.Interpreter = 'latex';
    cb.Label.FontSize = 24;

    hold on;
    grid on;

    % plot field lines
    st = streamslice(X, Z, Hx, Hz, 'arrows', 'cubic');
    set(st, 'Color', 'k');


    % plot magnet with north and south pole
    rectangle('Position', [-rsp -rsp 2*rsp 2*rsp], ...
        'Curvature', [1 1], 'LineWidth', 3.5);
    semicrc = rsp.*[cos(pz); sin(pz)];
    patch(semicrc(1,:), semicrc(2,:),'r');
    patch(-semicrc(1,:), -semicrc(2,:),'g');
    text(-1.25, 0, '\textbf{N}');
    text(0.5, 0, '\textbf{S}');

    % additional figure text and lines
    text(-(xz-1), -(xz-1), ...
        sprintf('$\\mathbf{Y = %.1f}$ \\textbf{mm}', y), 'Color', 'w');

    % distance scale in -z direction for x =0, distance from magnet surface
    line(xd, zd, 'Marker', '_', 'LineStyle', '-', 'LineWidth', 3.5, ...
        'Color', 'w');

    % place text along marker
    for i = 2:length(zd)-1
        markstr = "\\textbf{$\\mathbf{%d}$ mm, $\\mathbf{%.1f}$ kA/m}";
        mark = sprintf(markstr, abs(zd(i))-rsp, Hd(i));
        text(0.5, zd(i), mark, 'Color', 'w');
    end

    % limits ticks and labels
    xlim([-xz xz]);
    ylim([-xz xz]);
    xticks(-xz:xz);
    yticks(-xz:xz);
    labels = string(xticks);
    labels(1:2:end) = "";
    xticklabels(labels)
    yticklabels(labels)
    xlabel('$X$ in mm');
    ylabel('$Z$ in mm');

    % axis shape set
    axis equal;
    axis tight;
    grid off;

    % title and description
    disp('Title: Approximated Spherical Magnet with Dipole Far Field');
    disp("Description:");
    fprintf("Sphere whith imprinted H-field magnitude of %.1f kA/m\n", Hmag); ...
    fprintf("at distance d = %.1f mm with d_z = |z| - r\n", z0);
    fprintf("and sphere radius r = %.1f mm\n", rsp);

    % save results of figure
%     yesno = input('Save? [y/n]: ', 's');
%     if strcmp(yesno, 'y')
%         savefig(fig, figPath);
%         print(fig, figPath, '-dsvg');
%         print(fig, figPath, '-depsc', '-tiff', '-loose');
%         print(fig, figPath, '-dpdf', '-loose', '-fillpage');
%     end
%     close(fig)
end
\end{lstlisting}



\end{document}

