
% This LaTeX was auto-generated from MATLAB code.
% To make changes, update the MATLAB code and republish this document.

\documentclass{standalone}
\usepackage{graphicx}
\usepackage{listings}
\usepackage{xcolor}
\usepackage{textcomp}
\usepackage[framed, numbered]{matlab-prettifier}

\sloppy
\definecolor{lightgray}{gray}{0.5}
\setlength{\parindent}{0pt}

\begin{document}

    
    
\section*{plotFunctions}

\begin{par}
Project related reusable plots for datasets and results.
\end{par} \vspace{1em}

\subsection*{Contents}

\begin{itemize}
\setlength{\itemsep}{-1ex}
   \item plotTDKTransferCurves
   \item plotKMZ60TransferCurves
   \item plotKMZ60CharField
   \item plotKMZ60CharDataset
   \item plotSimulationDatasetCircle
   \item plotSimulationCosSinStats
   \item plotSimulationSubset
   \item plotSingleSimulationAngle
   \item plotSimulationDataset
   \item plotTDKCharField
   \item plotTDKCharDataset
   \item plotDipoleMagnet
\end{itemize}


\subsection*{plotTDKTransferCurves}

\begin{par}
Plot transfer curves for bridge output voltages of TDK TAS2141.
\end{par} \vspace{1em}


\subsection*{plotKMZ60TransferCurves}

\begin{par}
Plot transfer curves for bridge output voltages of NXP KMZ60.
\end{par} \vspace{1em}


\subsection*{plotKMZ60CharField}

\begin{par}
Plot NXP KMZ60 characterization field and slice around 0, 5, 10 and  15 kA/m.
\end{par} \vspace{1em}


\subsection*{plotKMZ60CharDataset}

\begin{par}
Explore the basic dataset of characterized NXP AMR sensor KMZ60 and plot the dataset content to visualize the base of dipole simulations.
\end{par} \vspace{1em}


\subsection*{plotSimulationDatasetCircle}

\begin{par}
Plot circular path of Hx, Hy and Vcos, Vsin at each sensor array position. Normed to max overall array positions and normed to max at each array position.
\end{par} \vspace{1em}


\subsection*{plotSimulationCosSinStats}

\begin{par}
Statistical compare plot of Vcos and Vsin output voltages for each sensor array members.
\end{par} \vspace{1em}


\subsection*{plotSimulationSubset}

\begin{par}
Plot subset of angles and sensor array position from training or test dataset.
\end{par} \vspace{1em}


\subsection*{plotSingleSimulationAngle}

\begin{par}
Plot single rotation step of test or training dataset.
\end{par} \vspace{1em}


\subsection*{plotSimulationDataset}

\begin{par}
Plot simulation test or training dataset created by sensor array simulation.
\end{par} \vspace{1em}


\subsection*{plotTDKCharField}

\begin{par}
Plot TDK TAS2141 characterization field and slice around 0, 5, 10 and  15 kA/m.
\end{par} \vspace{1em}


\subsection*{plotTDKCharDataset}

\begin{par}
Explore the basic dataset of characterized TDK TMR Sensor TAS2141 and plot the dataset content to visualize the base of dipole simulations.
\end{par} \vspace{1em}


\subsection*{plotDipoleMagnet}

\begin{par}
Plot dipole magnet and its approximation as spherical magnet from constants set in config file. Plot manget in rest position.
\end{par} \vspace{1em}
\begin{par}
Created on October 24. 2020 by Tobias Wulf. Copyright Tobias Wulf 2020.
\end{par} \vspace{1em}
\begin{par}

\end{par} \vspace{1em}



\end{document}

