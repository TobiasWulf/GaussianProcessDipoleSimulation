
% This LaTeX was auto-generated from MATLAB code.
% To make changes, update the MATLAB code and republish this document.

\documentclass{standalone}
\usepackage{graphicx}
\usepackage{listings}
\usepackage{xcolor}
\usepackage{textcomp}
\usepackage[framed, numbered]{matlab-prettifier}

\sloppy
\definecolor{lightgray}{gray}{0.5}
\setlength{\parindent}{0pt}

\begin{document}

    
    
\section*{computeDipoleH0Norm}

\begin{par}
Compute the norm factor for magnetic field generated by an Dipole in its zero position. That means the maximum H-field magnitude in zero position with no position shifts in x or y direction. So that norm factor is related to the center point of the coordinate system in x and y direction and to the dipoles initial z position. Which can be seen as sphere magnet for far field of the sphere. The norm relates that a dipole magnet in center of a sphere with a radius has certain field strength in related distance. For example a sphere of 2 mm radius has in 5 mm distance a field strength of 200 kA/m
\end{par} \vspace{1em}
\begin{par}
It is simplified computation for the dipole equation for one position in inital state without tilt in z-axes tor bring on a free choosen field strength to define the magnet. Because far field of sphere can be seen as dipole.
\end{par} \vspace{1em}
\begin{par}
$$\vec{H_0}(\vec{r_0}) = \frac{1}{4\pi} \cdot \Biggl(
\frac{3\vec{r_0}\left(\vec{m_0}^T\vec{r_0}\right)}{|\vec{r_0}|^5} -
\frac{\vec{m_0}}{|\vec{r_0}|^3}\Biggr)$$
\end{par} \vspace{1em}
\begin{par}
$$H_{0norm} = \frac{H_{mag}}{|H_0(r_0)|}$$
\end{par} \vspace{1em}

\subsection*{Contents}

\begin{itemize}
\setlength{\itemsep}{-1ex}
   \item Syntax
   \item Description
   \item Examples
   \item Input Argurments
   \item Output Argurments
   \item Requirements
   \item See Also
\end{itemize}


\subsection*{Syntax}

\begin{lstlisting}[style=Matlab-editor]
H0norm = computeDipoleH0Norm(Hmag, m0, r0)
\end{lstlisting}


\subsection*{Description}

\begin{par}
\textbf{H0norm = computeDipoleH0Norm(Hmag, m0, r0)} computes scalar norm factor related to dipole rest position. Multiply that factor to dipole generated fields which are computed with the same magnetic moment magnitude to imprint a choosen magnetic field strength magnitude on the dipole field rotation.
\end{par} \vspace{1em}


\subsection*{Examples}

\begin{lstlisting}[style=Matlab-editor]
% distance where the magnetic field strength is the value of wished
% magnitude, in mm
r0 = [0; 0; -5]
% field strength to imprint in norm factor in kA/m
Hmag = 200
% magnetic moment magnitude which is used generate rotation moments
m0 = [-1e6; 0; 0]
% compute norm factor for dipole rest position
H0norm = computeDipoleH0Norm(Hmag, m0, r0)
\end{lstlisting}


\subsection*{Input Argurments}

\begin{par}
\textbf{Hmag} real scalar of H-field strength magnitude to imprint in norm factor to define a dipole sphere with constant radius and field strength at this radius.
\end{par} \vspace{1em}
\begin{par}
\textbf{m0} vector of magnetic moment magnitude which must be same as for later rotation of the dipole.
\end{par} \vspace{1em}
\begin{par}
\textbf{r0} vector of distance in rest position of magnet center.
\end{par} \vspace{1em}


\subsection*{Output Argurments}

\begin{par}
\textbf{H0norm} real scalar of norm factor which relates to the zero position of the dipole sphere and can be multiplied to generated dipole H-field to imprint a magnetic field strength relative to the position of sensor array. The imprinted field strength magnitude relates to the rest position z0 + rsp.
\end{par} \vspace{1em}


\subsection*{Requirements}

\begin{itemize}
\setlength{\itemsep}{-1ex}
   \item Other m-files required: None
   \item Subfunctions: None
   \item MAT-files required: None
\end{itemize}


\subsection*{See Also}

\begin{itemize}
\setlength{\itemsep}{-1ex}
   \item \begin{verbatim}rotate3DVector\end{verbatim}
   \item \begin{verbatim}generateDipoleRotationMoments\end{verbatim}
   \item \begin{verbatim}Wikipedia Magnetic Dipole\end{verbatim}
\end{itemize}
\begin{par}
Created on November 11. 2020 by Tobias Wulf. Copyright Tobias Wulf 2020.
\end{par} \vspace{1em}
\begin{par}

\end{par} \vspace{1em}
\begin{lstlisting}[style=Matlab-editor]
function [H0norm] = computeDipoleH0Norm(Hmag, m0, r0)
    arguments
        % validate inputs as real scalars
        Hmag (1,1) double {mustBeReal}
        m0 (3,1) double {mustBeReal, mustBeVector}
        r0 (3,1) double {mustBeReal, mustBeVector}
    end

    % calculate the magnitude of all positions
    r0abs = sqrt(sum(r0.^2, 1));

    % calculate the the unit vector of all positions
    r0hat = r0 ./ r0abs;

    % calculate field strength and magnitude at position
    H0 = (3 * r0hat .* (m0' * r0hat) - m0) ./ (4 * pi *r0abs.^3);
    H0abs = sqrt(sum(H0.^2, 1));

    % compute the norm factor like described in the equations
    H0norm = Hmag / H0abs;
end
\end{lstlisting}



\end{document}

