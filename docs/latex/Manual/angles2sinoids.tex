
% This LaTeX was auto-generated from MATLAB code.
% To make changes, update the MATLAB code and republish this document.

\documentclass{standalone}
\usepackage{graphicx}
\usepackage{listings}
\usepackage{xcolor}
\usepackage{textcomp}
\usepackage[framed, numbered]{matlab-prettifier}

\sloppy
\definecolor{lightgray}{gray}{0.5}
\setlength{\parindent}{0pt}

\begin{document}

    
    \begin{par}
Converts angles (rad or degree) to sine and cosine waves with respect to a period factor which gives the abillity to apstract higher periodicity. Additionally the angles are recalculated according to passed period factor.
\end{par} \vspace{1em}
\begin{par}
Computes sine and cosine by product of angle in rad multiplied by period factor.
\end{par} \vspace{1em}
\begin{par}
$$f_{sin} = \sin(p_f \cdot f_{ang})$$
\end{par} \vspace{1em}
\begin{par}
$$f_{cos} = \cos(p_f \cdot f_{ang})$$
\end{par} \vspace{1em}
\begin{par}
If needed a recomputation of the given angels takes place by computed sinoids.
\end{par} \vspace{1em}


{\footnotesize\textbf{Syntax}}

\begin{lstlisting}[style=Matlab-editor, basicstyle=\ttfamily\scriptsize]
[fsin, fcos, fang] = angles2sinoids(fang, rad, pf)
\end{lstlisting}


{\footnotesize\textbf{Description}}

\begin{par}
\textbf{[fsin, fcos, fang] = angles2sinoids(fang, rad, pf)} computes sinoids from passed angles in rad or degree with respect to periodicity of angles. The flag rad converts input angles from degree to rad if set to false.
\end{par} \vspace{1em}


{\footnotesize\textbf{Examples}}

\begin{lstlisting}[style=Matlab-editor, basicstyle=\ttfamily\scriptsize]
fang = linspace(0, 360, 100);
[fsin, fcos, fang] = angles2sinoids(fang, true, 1)
\end{lstlisting}


{\footnotesize\textbf{Input Argurments}}

\begin{par}
\textbf{fang} is a scalar of vector of angles in rad or degree.
\end{par} \vspace{1em}
\begin{par}
\textbf{rad} is a boolean flag. Input angles are converted to rad if set to false.
\end{par} \vspace{1em}
\begin{par}
\textbf{pf} is a positive integer factor. The period factor describes the periodicity of angles in data.
\end{par} \vspace{1em}


{\footnotesize\textbf{Output Argurments}}

\begin{par}
\textbf{fsin} is a scalar or vector of sine values corresponding to passed angles with respect of the periodicity of angles.
\end{par} \vspace{1em}
\begin{par}
\textbf{fcos} is a scalar or vector of cosine values corresponding to passed angles with respect of the periodicity of angles.
\end{par} \vspace{1em}
\begin{par}
\textbf{fang} is a scalar or vector of recalculated angles with respect of periodicity.
\end{par} \vspace{1em}


{\footnotesize\textbf{Requirements}}

\begin{itemize}
\setlength{\itemsep}{-1ex}
   \item Other m-files required: sinoids2angles
   \item Subfunctions: sin, cos
   \item MAT-files required: None
\end{itemize}


{\footnotesize\textbf{See Also}}

\begin{itemize}
\setlength{\itemsep}{-1ex}
   \item \begin{verbatim}sinoids2angles\end{verbatim}
\end{itemize}
\begin{par}
Created on December 31. 2020 by Tobias Wulf. Copyright Tobias Wulf 2020.
\end{par} \vspace{1em}
\begin{par}

\end{par} \vspace{1em}
\begin{lstlisting}[style=Matlab-editor, basicstyle=\ttfamily\scriptsize]
function [fsin, fcos, fang] = angles2sinoids(fang, rad, pf)
    arguments
        % validate angles as scalar or vector
        fang (:,1) double {mustBeReal}
        % validate rad as boolean flag with default true
        rad (1,1) logical {mustBeNumericOrLogical} = true
        % validate period factor as positive scalar with default 1
        pf (1,1) double {mustBeInteger, mustBePositive} = 1
    end

    % if rad flag is false and angles in degree convert to rad
    if ~rad, fang = fang * pi / 180; end

    % calculate sinoids
    fsin = sin(pf * fang);
    fcos = cos(pf * fang);

    % compute radius
    frad = sqrt(fcos.^2 + fsin.^2);

    % recalculate angles to corrected sinoids in rad
    if nargout > 2, fang = sinoids2angles(fsin, fcos, frad); end
end
\end{lstlisting}



\end{document}

