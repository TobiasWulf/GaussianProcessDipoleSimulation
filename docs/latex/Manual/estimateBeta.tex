
% This LaTeX was auto-generated from MATLAB code.
% To make changes, update the MATLAB code and republish this document.

\documentclass{standalone}
\usepackage{graphicx}
\usepackage{listings}
\usepackage{xcolor}
\usepackage{textcomp}
\usepackage[framed, numbered]{matlab-prettifier}

\sloppy
\definecolor{lightgray}{gray}{0.5}
\setlength{\parindent}{0pt}

\begin{document}

    
    \begin{par}
Find beta coefficients to basis matrix H and the current set of hyperparameters theta as vector of s2f and sl, s2n represented by the current inverse of noisy covariance matrix Ky\^{}-1 and the feature target vector y of the observations. It calculates several inverse Matrix products so instead passing the current Ky the function uses the infront decomposed lower triangle matrix L of Ky.
\end{par} \vspace{1em}


{\footnotesize\textbf{Syntax}}

\begin{lstlisting}[style=Matlab-editor, basicstyle=\ttfamily\scriptsize]
[beta, alpha0]= estimateBeta(H, L, y)
\end{lstlisting}


{\footnotesize\textbf{Description}}

\begin{par}
\textbf{[beta, alpha0]= estimateBeta(H, L, y)} compute polynom coefficients to solve mean approximation.
\end{par} \vspace{1em}


{\footnotesize\textbf{Input Argurments}}

\begin{par}
\textbf{H} basis matrix of training data. Polynomial represents of training data.
\end{par} \vspace{1em}
\begin{par}
\textbf{L} lower triangle matrix of decomposed K matrix.
\end{par} \vspace{1em}
\begin{par}
\textbf{y} regression targets.
\end{par} \vspace{1em}


{\footnotesize\textbf{Output Argurments}}

\begin{par}
\textbf{beta} beta coefficients for polynomial approximation with basis matrix \textbf{H}.
\end{par} \vspace{1em}
\begin{par}
\textbf{alpha0} regression weights based on regression targets \textbf{y}.
\end{par} \vspace{1em}


{\footnotesize\textbf{Requirements}}

\begin{itemize}
\setlength{\itemsep}{-1ex}
   \item Other m-files required: None
   \item Subfunctions: chol, computeInverseMatrixProduct,   computeTransposeInverseProduct
   \item MAT-files required: None
\end{itemize}


{\footnotesize\textbf{See Also}}

\begin{itemize}
\setlength{\itemsep}{-1ex}
   \item \begin{verbatim}computeInverseMatrixProduct\end{verbatim}
   \item \begin{verbatim}computeTransposeInverseProduct\end{verbatim}
   \item \begin{verbatim}initKernelParameters\end{verbatim}
\end{itemize}
\begin{par}
Created on February 15. 2021 by Tobias Wulf. Copyright Tobias Wulf 2021.
\end{par} \vspace{1em}
\begin{par}

\end{par} \vspace{1em}
\begin{lstlisting}[style=Matlab-editor, basicstyle=\ttfamily\scriptsize]
function [beta, alpha0]= estimateBeta(H, L, y)
    % Ky^-1 * y
    alpha0 = computeInverseMatrixProduct(L, y);

    % H * Ky^-1 * HT
    alpha1 = computeTransposeInverseProduct(L, H');

    % (H * Ky^-1 * HT)^-1 * H
    L1 = chol(alpha1, 'lower');
    alpha2 = computeInverseMatrixProduct(L1, H);

    % ((H * (Ky^-1 * HT))^-1 * H) * (Ky^-1 * y)
    beta = alpha2 * alpha0;
end
\end{lstlisting}



\end{document}

