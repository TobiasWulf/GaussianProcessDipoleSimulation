
% This LaTeX was auto-generated from MATLAB code.
% To make changes, update the MATLAB code and republish this document.

\documentclass{standalone}
\usepackage{graphicx}
\usepackage{listings}
\usepackage{xcolor}
\usepackage{textcomp}
\usepackage[framed, numbered]{matlab-prettifier}

\sloppy
\definecolor{lightgray}{gray}{0.5}
\setlength{\parindent}{0pt}

\begin{document}

    
    \begin{par}
Explore TDK TAS2141 characterization field.
\end{par} \vspace{1em}


{\textbf{Syntax}}

\begin{lstlisting}[style=Matlab-editor, basicstyle=\ttfamily\scriptsize]
plotTDKCharField()
\end{lstlisting}


{\textbf{Description}}

\begin{par}
\textbf{plotTDKCharField()} explore characterization field of TDK sensor.
\end{par} \vspace{1em}


{\textbf{Examples}}

\begin{lstlisting}[style=Matlab-editor, basicstyle=\ttfamily\scriptsize]
plotTDKCharField();
\end{lstlisting}


{\textbf{Input Arguments}}

\begin{par}
\textbf{None}
\end{par} \vspace{1em}


{\textbf{Output Arguments}}

\begin{par}
\textbf{None}
\end{par} \vspace{1em}


{\textbf{Requirements}}

\begin{itemize}
\setlength{\itemsep}{-1ex}
   \item Other m-files: none
   \item Subfunctions: none
   \item MAT-files required: data/TDK\_TAS2141\_Characterization\_2020-10-22\_18-12-16-827.mat,   data/config.mat
\end{itemize}


{\textbf{See Also}}

\begin{itemize}
\setlength{\itemsep}{-1ex}
   \item \begin{verbatim}plotTDKCharDataset\end{verbatim}
\end{itemize}
\begin{par}
Created on October 28. 2020 by Tobias Wulf. Copyright Tobias Wulf 2020.
\end{par} \vspace{1em}
\begin{par}

\end{par} \vspace{1em}
\begin{lstlisting}[style=Matlab-editor, basicstyle=\ttfamily\scriptsize]
function plotTDKCharField()
    try
        % load dataset path and dataset content into function workspace
        load('config.mat', 'PathVariables');
        load(PathVariables.tdkDatasetPath, 'Data', 'Info');
%         close all;
    catch ME
        rethrow(ME)
    end

    % load needed data from dataset in to local variables for better handling %%
    %%%%%%%%%%%%%%%%%%%%%%%%%%%%%%%%%%%%%%%%%%%%%%%%%%%%%%%%%%%%%%%%%%%%%%%%%%%%
    % get from user which field to investigate and limits for plateau
    fields = Info.SensorOutput.CosinusBridge.Determination;
    nFields = length(fields);
    fprintf('Choose 1 of %d fields ...\n', nFields);
    for i = 1:nFields
        fprintf('%s\t:\t(%d)\n', fields{i}, i);
    end

    iField = 1; % input('Choice: ');
    field = fields{iField};
    pl = 5; % input('Plateu limit in kA/m: ');

    Vcos = Data.SensorOutput.CosinusBridge.(field);
    Vsin = Data.SensorOutput.SinusBridge.(field);
    gain = Info.SensorOutput.BridgeGain;
    HxScale = Data.MagneticField.hx;
    HyScale = Data.MagneticField.hy;
    Hmin = Info.MagneticField.MinAmplitude;
    Hmax = Info.MagneticField.MaxAmplitude;

    % get unit strings from
    kApm = Info.Units.MagneticFieldStrength;
    mV = Info.Units.SensorOutputVoltage;

    % get dataset infos and format strings to place in figures
    % subtitle string for all figures
    infoStr = join([Info.SensorManufacturer, ...
        Info.Sensor, Info.SensorTechnology, ...
        Info.SensorType, "Sensor Characterization Dataset."]);
    dateStr = join(["Created on", Info.Created, "by", 'Thorben Sch\"uthe', ...
        "and updated on", Info.Edited, "by", Info.Editor + "."]);

    % clear dataset all loaded
    clear Data Info;

    % figure save path for different formats %%%%%%%%%%%%%%%%%%%%%%%%%%%%%%%%%%%
    %%%%%%%%%%%%%%%%%%%%%%%%%%%%%%%%%%%%%%%%%%%%%%%%%%%%%%%%%%%%%%%%%%%%%%%%%%%%
    fName = sprintf("tdk_char_field_%s", field);
    fPath = fullfile(PathVariables.saveImagesPath, fName);

    % define slices and limits to plot %%%%%%%%%%%%%%%%%%%%%%%%%%%%%%%%%%%%%%%%%
    %%%%%%%%%%%%%%%%%%%%%%%%%%%%%%%%%%%%%%%%%%%%%%%%%%%%%%%%%%%%%%%%%%%%%%%%%%%%
    Hslice = [128 154 180 205]; % hit ca. 0, 5, 10, 15 kA/m
    Hlims = [-pl pl];
    mVpVlims = [-175 175];

    % create figure for plots %%%%%%%%%%%%%%%%%%%%%%%%%%%%%%%%%%%%%%%%%%%%%%%%%%
    %%%%%%%%%%%%%%%%%%%%%%%%%%%%%%%%%%%%%%%%%%%%%%%%%%%%%%%%%%%%%%%%%%%%%%%%%%%%
    fig = figure('Name', 'Char Field', 'OuterPosition', [0 0 35 30]);
    tiledlayout(fig, 2, 2);

    % title and description
    disp('Info:');
    disp([infoStr; dateStr]);
    fprintf('Title: TDK Characterization Field - %s\n', field);
    disp('Description:');
    disp(["a) Cosine Bridge Characteristic"; ...
          "b) Transfer slices for different const. H_y of Vcos"; ...
          "c) Sine Bridge Characteristic"; ...
          "d) Transfer slices for different const. H_x of Vsin"]);

    % set colormap
    colormap('jet');

    % cosinus bridge %%%%%%%%%%%%%%%%%%%%%%%%%%%%%%%%%%%%%%%%%%%%%%%%%%%%%%%%%%%
    %%%%%%%%%%%%%%%%%%%%%%%%%%%%%%%%%%%%%%%%%%%%%%%%%%%%%%%%%%%%%%%%%%%%%%%%%%%%
    nexttile(1);
    im = imagesc(HxScale, HyScale, Vcos);
    set(gca, 'YDir', 'normal');
    set(im, 'AlphaData', ~isnan(Vcos));
    xticks(-20:10:20);
    yticks(-20:10:20);
    axis square xy;

    % plot lines for slice to investigate
    hold on;
    for i = Hslice
        yline(HyScale(i), 'k:', 'LineWidth', 3.5);
    end
    hold off;

    xlabel(sprintf('$H_x$ in %s', kApm));
    ylabel(sprintf('$H_y$ in %s', kApm));
    title(sprintf('a) $V_{cos}(H_x,H_y)$, Gain $ = %.1f$', gain));

    cb = colorbar;
    cb.Label.String = sprintf('$V_{cos}$ in %s', mV);
    cb.Label.Interpreter = 'latex';
    cb.TickLabelInterpreter = 'latex';
    cb.Label.FontSize = 20;

    % cosinus bridge sclices %%%%%%%%%%%%%%%%%%%%%%%%%%%%%%%%%%%%%%%%%%%%%%%%%%%
    %%%%%%%%%%%%%%%%%%%%%%%%%%%%%%%%%%%%%%%%%%%%%%%%%%%%%%%%%%%%%%%%%%%%%%%%%%%%
    nexttile(2);
    % slices
    p = plot(HxScale, Vcos(Hslice,:));

    % plateau limits
    if pl > 0
        hold on;
        xline(Hlims(1), 'k-.', 'LineWidth', 2.5);
        xline(Hlims(2), 'k-.', 'LineWidth', 2.5);
        hold off;
    end

    legend(p, {'$H_y \approx 0$ kA/m', ...
               '$H_y \approx 5$ kA/m', ...
               '$H_y \approx 10$ kA/m', ...
               '$H_y \approx 15$ kA/m'},...
            'Location', 'SouthEast');
    xlabel(sprintf('$H_x$ in %s', kApm));
    title('b) $V_{cos}(H_x,H_y)$, $H_y = $ const.');
    ylim(mVpVlims);
    xlim([Hmin Hmax])

    % sinus bridge %%%%%%%%%%%%%%%%%%%%%%%%%%%%%%%%%%%%%%%%%%%%%%%%%%%%%%%%%%%%%
    %%%%%%%%%%%%%%%%%%%%%%%%%%%%%%%%%%%%%%%%%%%%%%%%%%%%%%%%%%%%%%%%%%%%%%%%%%%%
    nexttile(3);
    im = imagesc(HxScale, HyScale, Vsin);
    set(gca, 'YDir', 'normal');
    set(im, 'AlphaData', ~isnan(Vsin));
    xticks(-20:10:20);
    yticks(-20:10:20);
    axis square xy;

    % plot lines for slice to investigate
    hold on;
    for i = Hslice
        xline(HxScale(i), 'k:', 'LineWidth', 3.5);
    end
    hold off;

    xlabel(sprintf('$H_x$ in %s', kApm));
    ylabel(sprintf('$H_y$ in %s', kApm));
    title(sprintf('c) $V_{sin}(H_x,H_y)$, Gain $ = %.1f$', gain));

    cb = colorbar;
    cb.Label.String = sprintf('$V_{sin}$ in %s', mV);
    cb.Label.Interpreter = 'latex';
    cb.TickLabelInterpreter = 'latex';
    cb.Label.FontSize = 20;

    % sinus bridge sclices %%%%%%%%%%%%%%%%%%%%%%%%%%%%%%%%%%%%%%%%%%%%%%%%%%%%%
    %%%%%%%%%%%%%%%%%%%%%%%%%%%%%%%%%%%%%%%%%%%%%%%%%%%%%%%%%%%%%%%%%%%%%%%%%%%%
    nexttile(4);
    % slices
    p = plot(HxScale, Vsin(:,Hslice));

    % plateau limits
    if pl > 0
        hold on;
        xline(Hlims(1), 'k-.', 'LineWidth', 2.5);
        xline(Hlims(2), 'k-.', 'LineWidth', 2.5);
        hold off;
    end

    legend(p, {'$H_x \approx 0$ kA/m', ...
               '$H_x \approx 5$ kA/m', ...
               '$H_x \approx 10$ kA/m', ...
               '$H_x \approx 15$ kA/m'},...
            'Location', 'SouthEast');
    xlabel(sprintf('$H_y$ in %s', kApm));
    title('d) $V_{sin}(H_x,H_y)$, $H_x = $ const.');
    ylim(mVpVlims);
    xlim([Hmin Hmax])

    % save results of figure %%%%%%%%%%%%%%%%%%%%%%%%%%%%%%%%%%%%%%%%%%%%%%%%%%%
    %%%%%%%%%%%%%%%%%%%%%%%%%%%%%%%%%%%%%%%%%%%%%%%%%%%%%%%%%%%%%%%%%%%%%%%%%%%%
%     yesno = input('Save? [y/n]: ', 's');
%     if strcmp(yesno, 'y')
%         savefig(fig, fPath);
%         print(fig, fPath, '-dsvg');
%         print(fig, fPath, '-depsc', '-tiff', '-loose');
%         print(fig, fPath, '-dpdf', '-loose', '-fillpage');
%     end
%     close(fig)
end
\end{lstlisting}



\end{document}

