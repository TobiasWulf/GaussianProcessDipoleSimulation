
% This LaTeX was auto-generated from MATLAB code.
% To make changes, update the MATLAB code and republish this document.

\documentclass{standalone}
\usepackage{graphicx}
\usepackage{listings}
\usepackage{xcolor}
\usepackage{textcomp}
\usepackage[framed, numbered]{matlab-prettifier}

\sloppy
\definecolor{lightgray}{gray}{0.5}
\setlength{\parindent}{0pt}

\begin{document}

    
    
\section*{Source Code}

\begin{par}
The project source code is clustered in modules where every subdirectory represents one certain module. Each module gathers functions and classes which are related to module specific themes or task fields. So the basic structured source code is located here. The combination of module functionality takes place in executable area of the project. So use the functions and classes in scripts and further on compiled binaries. Do not write bare executable source code here. For reproducible results and source code tracebility each module has its own documentation entry where all underlaying functions and classses are listed. The best practice to develop new source code or modules is to do it in test driven way. This means write a test m-file for every new function or class m-file and test the functionality of the source code with assertion. This test driven development is called unittest and provides in combination with detailed documentation a high percentage of reusable source code.
\end{par} \vspace{1em}

\subsection*{Contents}

\begin{itemize}
\setlength{\itemsep}{-1ex}
   \item sensorArraySimulation
   \item gaussianProcessRegression
   \item util
\end{itemize}


\subsection*{sensorArraySimulation}

\begin{par}
Function space to solve sensor array simulation with a certain magnetic stimulus. The Array simulation is based on the TDK TAS2141 characterization dataset. A magnetic dipole is used as basic magnetic stimulus and moved as imaginary sphere magnet with a certain radius. The magnet rotates in z-direction counterclockwise.
\end{par} \vspace{1em}


\subsection*{gaussianProcessRegression}

\begin{par}
Source code establish a struct based regression model which has the abillity to work with datasets generated by sensorArraySimulation module. Generated regression models and datasets can be passed to prediction functions and loss computation as arguments. So it is possible to build up multi model evaluation in scripts at a time.
\end{par} \vspace{1em}


\subsection*{util}

\begin{par}
Util function and classes to provide reuse for often upcommings tasks and functionality besides project kernel and module source code. Located are plot functions and file operations.
\end{par} \vspace{1em}
\begin{par}
Created on October 10. 2020 by Tobias Wulf. Copyright Tobias Wulf 2020.
\end{par} \vspace{1em}
\begin{par}

\end{par} \vspace{1em}



\end{document}

