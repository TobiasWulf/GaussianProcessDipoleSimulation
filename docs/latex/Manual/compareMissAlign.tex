
% This LaTeX was auto-generated from MATLAB code.
% To make changes, update the MATLAB code and republish this document.

\documentclass{standalone}
\usepackage{graphicx}
\usepackage{listings}
\usepackage{xcolor}
\usepackage{textcomp}
\usepackage[framed, numbered]{matlab-prettifier}

\sloppy
\definecolor{lightgray}{gray}{0.5}
\setlength{\parindent}{0pt}

\begin{document}

    
    \begin{enumerate}
\setlength{\itemsep}{-1ex}
   \item Create reference dataset to compare against drift and retraining. Reference   Position.
   \item Create same number of drift datasets for each drift in x,y,z and tilt drift.   Position drifts in x,y,z must have the same drift steps diverging from the   reference position to match a drift plots. Tilt drift is assigned to a   separate axis.
   \item Execute this script and ensure dataset chaining. First dataset is reference   dataset, second bunch must be drift in x, third bunch in drift in y, fourth   bunch drift in z and fifth bunch drift in tilt. So loaded file pattern   schould have following indices for e.g. 25 drift iterations each:
\end{enumerate}
\begin{itemize}
\setlength{\itemsep}{-1ex}
   \item (1) reference dataset
   \item (2-26) drift in x
   \item (27-51) drift in y
   \item (52-76) drift in z
   \item (77-101) drift in tilt
\end{itemize}
\begin{par}
Script forces simulation to run with TDK characterization dataset. Identifier is set manually to TDK.
\end{par} \vspace{1em}


{\footnotesize\textbf{Requirements}}

\begin{itemize}
\setlength{\itemsep}{-1ex}
   \item Other m-files required: gaussianProcessRegression module files
   \item Subfunctions: none
   \item MAT-files required: data/config.mat, corresponding Training and Test dataset
\end{itemize}


{\footnotesize\textbf{See Also}}

\begin{itemize}
\setlength{\itemsep}{-1ex}
   \item \begin{verbatim}gaussianProcessRegression\end{verbatim}
   \item \begin{verbatim}initGPR\end{verbatim}
   \item \begin{verbatim}tuneGPR\end{verbatim}
   \item \begin{verbatim}optimGPR\end{verbatim}
   \item \begin{verbatim}generateConfigMat\end{verbatim}
\end{itemize}
\begin{par}
Created on May 06. 2021 by Tobias Wulf. Copyright Tobias Wulf 2021.
\end{par} \vspace{1em}
\begin{par}

\end{par} \vspace{1em}


{\footnotesize\textbf{Start Script and Clear All}}

\begin{lstlisting}[style=Matlab-editor, basicstyle=\ttfamily\scriptsize]
clc;
disp('Start compare miss align ...');
disp('Clear all ...');
clearvars;
close all;
deleteSimulationDatasets;
\end{lstlisting}


{\footnotesize\textbf{Declare Reference Position and Drift Vectors}}

\begin{lstlisting}[style=Matlab-editor, basicstyle=\ttfamily\scriptsize]
disp('Declare reference and drift ...');
refX      = 0;
refY      = 0;
refZ      = 7.5;
refTilt   = 0;
driftX    = -3:0.25:3;
driftY    = -3:0.25:3;
driftZ    = 4.5:0.25:10.5;
driftTilt = 0:0.5:12;
\end{lstlisting}


{\footnotesize\textbf{Declare Common Variables}}

\begin{lstlisting}[style=Matlab-editor, basicstyle=\ttfamily\scriptsize]
disp('Declare common vars ...');
trainPath   = PathVariables.trainingDataPath;
testPath    = PathVariables.testDataPath;
maxErrorTDK = 0.6;
errorTick   = [1e-1 1 10 1e2];
\end{lstlisting}


{\footnotesize\textbf{Calculated Datasets Indices on Drift Parameter}}

\begin{lstlisting}[style=Matlab-editor, basicstyle=\ttfamily\scriptsize]
disp('Numbering drift sets ...');
driftXN    = length(driftX);
driftYN    = length(driftY);
driftZN    = length(driftZ);
driftTiltN = length(driftTilt);
driftAllN  = driftXN + driftYN + driftZN + driftTiltN;
fprintf('Overall drifts: %d ...', driftAllN);

assert(driftXN == driftYN,    'Inbalance drift in x2y')
assert(driftXN == driftZN,    'Inbalance drift in x2z')
assert(driftXN == driftTiltN, 'Inbalance drift in x2tilt')
disp('Number of drift iterations equals: pass ...');

disp('Index drift set ...');
refIndex       = 1;
driftXIndex    = refIndex + 1       : driftXN    + refIndex;
driftYIndex    = driftXIndex(end) + 1 : driftYN    + driftXIndex(end);
driftZIndex    = driftYIndex(end) + 1 : driftZN    + driftYIndex(end);
driftTiltIndex = driftZIndex(end) + 1 : driftTiltN + driftZIndex(end);
\end{lstlisting}


{\footnotesize\textbf{Allocate Memory for Error and Parameter Drifts [X, Y, Z, Tilt]}}

\begin{lstlisting}[style=Matlab-editor, basicstyle=\ttfamily\scriptsize]
disp('Allocate memory ...');
meanError2Ref       = zeros(driftXN, 4);
maxError2Ref        = zeros(driftXN, 4);
meanError2Retrained = zeros(driftXN, 4);
maxError2Retrained  = zeros(driftXN, 4);
s2nParamDrift       = zeros(driftXN, 4);
s2fParamDrift       = zeros(driftXN, 4);
slParamDrift        = zeros(driftXN, 4);
\end{lstlisting}


{\footnotesize\textbf{Load Config}}

\begin{lstlisting}[style=Matlab-editor, basicstyle=\ttfamily\scriptsize]
disp('Load config ...');
load config.mat PathVariables  GeneralOptions SensorArrayOptions ...
    DipoleOptions TrainingOptions TestOptions GPROptions;
\end{lstlisting}


{\footnotesize\textbf{Load TDK Characterization Dataset and Set Base Reference in Options}}

\begin{lstlisting}[style=Matlab-editor, basicstyle=\ttfamily\scriptsize]
disp('Load TDK characterization dataset ...');

CharDS = load(PathVariables.tdkDatasetPath);

TrainingOptions.BaseReference = 'TDK';
TestOptions.BaseReference     = 'TDK';
\end{lstlisting}


{\footnotesize\textbf{Generate Reference Datasets}}

\begin{lstlisting}[style=Matlab-editor, basicstyle=\ttfamily\scriptsize]
disp('Set reference position ...');
TrainingOptions.xPos = refX;
TrainingOptions.yPos = refY;
TrainingOptions.zPos = refZ;
TrainingOptions.tilt = refTilt;

TestOptions.xPos = TrainingOptions.xPos;
TestOptions.yPos = TrainingOptions.yPos;
TestOptions.zPos = TrainingOptions.zPos;
TestOptions.tilt = TrainingOptions.tilt;

disp('Generate reference datasets ...');
simulateDipoleSquareSensorArray(GeneralOptions, PathVariables, ...
    SensorArrayOptions, DipoleOptions, TrainingOptions, CharDS);
simulateDipoleSquareSensorArray(GeneralOptions, PathVariables, ...
    SensorArrayOptions, DipoleOptions, TestOptions, CharDS);
\end{lstlisting}


{\footnotesize\textbf{Generate X Drift Datasets}}

\begin{lstlisting}[style=Matlab-editor, basicstyle=\ttfamily\scriptsize]
disp('Set x drift positions ...');
TrainingOptions.xPos = driftX;
% TrainingOptions.yPos = refY;
% TrainingOptions.zPos = refZ;
% TrainingOptions.tilt = refTilt;

TestOptions.xPos = TrainingOptions.xPos;
% TestOptions.yPos = TrainingOptions.yPos;
% TestOptions.zPos = TrainingOptions.zPos;
% TestOptions.tilt = TrainingOptions.tilt;

disp('Generate x drift datasets ...');
simulateDipoleSquareSensorArray(GeneralOptions, PathVariables, ...
    SensorArrayOptions, DipoleOptions, TrainingOptions, CharDS);
simulateDipoleSquareSensorArray(GeneralOptions, PathVariables, ...
    SensorArrayOptions, DipoleOptions, TestOptions, CharDS);
\end{lstlisting}


{\footnotesize\textbf{Generate Y Drift Datasets}}

\begin{lstlisting}[style=Matlab-editor, basicstyle=\ttfamily\scriptsize]
disp('Set y drift positions ...');
TrainingOptions.xPos = refX;
TrainingOptions.yPos = driftY;
% TrainingOptions.zPos = refZ;
% TrainingOptions.tilt = refTilt;

TestOptions.xPos = TrainingOptions.xPos;
TestOptions.yPos = TrainingOptions.yPos;
% TestOptions.zPos = TrainingOptions.zPos;
% TestOptions.tilt = TrainingOptions.tilt;

disp('Generate y drift datasets ...');
simulateDipoleSquareSensorArray(GeneralOptions, PathVariables, ...
    SensorArrayOptions, DipoleOptions, TrainingOptions, CharDS);
simulateDipoleSquareSensorArray(GeneralOptions, PathVariables, ...
    SensorArrayOptions, DipoleOptions, TestOptions, CharDS);
\end{lstlisting}


{\footnotesize\textbf{Generate Z Drift Datasets}}

\begin{lstlisting}[style=Matlab-editor, basicstyle=\ttfamily\scriptsize]
disp('Set z drift positions ...');
% TrainingOptions.xPos = refX;
TrainingOptions.yPos = refY;
TrainingOptions.zPos = driftZ;
% TrainingOptions.tilt = refTilt;

% TestOptions.xPos = TrainingOptions.xPos;
TestOptions.yPos = TrainingOptions.yPos;
TestOptions.zPos = TrainingOptions.zPos;
% TestOptions.tilt = TrainingOptions.tilt;

disp('Generate z drift datasets ...');
simulateDipoleSquareSensorArray(GeneralOptions, PathVariables, ...
    SensorArrayOptions, DipoleOptions, TrainingOptions, CharDS);
simulateDipoleSquareSensorArray(GeneralOptions, PathVariables, ...
    SensorArrayOptions, DipoleOptions, TestOptions, CharDS);
\end{lstlisting}


{\footnotesize\textbf{Generate Tilt Drift Datasets}}

\begin{lstlisting}[style=Matlab-editor, basicstyle=\ttfamily\scriptsize]
disp('Set tilt drift positions ...');
% TrainingOptions.xPos = refX;
% TrainingOptions.yPos = refY;
TrainingOptions.zPos = refZ;
% TrainingOptions.tilt = refTilt;

% TestOptions.xPos = TrainingOptions.xPos;
% TestOptions.yPos = TrainingOptions.yPos;
TestOptions.zPos = TrainingOptions.zPos;
% TestOptions.tilt = TrainingOptions.tilt;

disp('Generate tilt drift datasets ...');
for tilt = driftTilt
    TrainingOptions.tilt = tilt;
    TestOptions.tilt     = tilt;

    simulateDipoleSquareSensorArray(GeneralOptions, PathVariables, ...
        SensorArrayOptions, DipoleOptions, TrainingOptions, CharDS);
    simulateDipoleSquareSensorArray(GeneralOptions, PathVariables, ...
        SensorArrayOptions, DipoleOptions, TestOptions, CharDS);
end
\end{lstlisting}


{\footnotesize\textbf{Reload Path of Generated Datasets}}

\begin{lstlisting}[style=Matlab-editor, basicstyle=\ttfamily\scriptsize]
disp('Scanning directories ...');
TrainFiles = dir(fullfile(trainPath, 'Training*.mat'));
TestFiles  = dir(fullfile(testPath, 'Test*.mat'));

TrainFilesN = length(TrainFiles);
TestFilesN  = length(TestFiles);

assert(TrainFilesN == TestFilesN, 'Inbalance in files.');
disp('Data is consistent ...');
\end{lstlisting}


{\footnotesize\textbf{Load Reference Datasets Train Reference Model}}

\begin{lstlisting}[style=Matlab-editor, basicstyle=\ttfamily\scriptsize]
fprintf('Load datasets (%d) ...\n', refIndex);
TrainDS = load(fullfile(trainPath, TrainFiles(refIndex).name));
TestDS  = load(fullfile(testPath, TestFiles(refIndex).name));
\end{lstlisting}


{\footnotesize\textbf{Train Reference Model on Reference Datasets}}

\begin{lstlisting}[style=Matlab-editor, basicstyle=\ttfamily\scriptsize]
disp('Train model in reference position ...')
RefMdl = optimGPR(TrainDS, TestDS, GPROptions, 0);
close all;
\end{lstlisting}


{\footnotesize\textbf{Run Drift in X}}

\begin{lstlisting}[style=Matlab-editor, basicstyle=\ttfamily\scriptsize]
disp('Run drift in x');
iDrift = 1;
for iFile = driftXIndex
    fprintf('Load datasets (%d) ...\n', iFile);
    TrainDS = load(fullfile(trainPath, TrainFiles(iFile).name));
    TestDS  = load(fullfile(testPath, TestFiles(iFile).name));

    fprintf('Measure errors x drift %.2f to ref ...\n', driftX(iDrift));
    absError2Ref             = lossDS(RefMdl, TestDS);
    meanError2Ref(iDrift, 1) = mean(absError2Ref);
    maxError2Ref(iDrift, 1)  = max(absError2Ref);

    disp('Retrain model on drift position ...');
    RetrainedMdl = optimGPR(TrainDS, TestDS, GPROptions, 0);
    close all;

    disp('Measure errors on retrained model ...')
    absErrorRetrained              = lossDS(RetrainedMdl, TestDS);
    meanError2Retrained(iDrift, 1) = mean(absErrorRetrained);
    maxError2Retrained(iDrift, 1)  = max(absErrorRetrained);

    disp('Track model parameters');
    s2nParamDrift(iDrift, 1) = RetrainedMdl.s2n;
    s2fParamDrift(iDrift, 1) = RetrainedMdl.theta(1);
    slParamDrift(iDrift, 1)  = RetrainedMdl.theta(2);

    iDrift = iDrift + 1;
end
\end{lstlisting}


{\footnotesize\textbf{Run Drift in Y}}

\begin{lstlisting}[style=Matlab-editor, basicstyle=\ttfamily\scriptsize]
disp('Run drift in y');
iDrift = 1;
for iFile = driftYIndex
    fprintf('Load datasets (%d) ...\n', iFile);
    TrainDS = load(fullfile(trainPath, TrainFiles(iFile).name));
    TestDS  = load(fullfile(testPath, TestFiles(iFile).name));

    fprintf('Measure errors y drift %.2f to ref ...\n', driftY(iDrift));
    absError2Ref             = lossDS(RefMdl, TestDS);
    meanError2Ref(iDrift, 2) = mean(absError2Ref);
    maxError2Ref(iDrift, 2)  = max(absError2Ref);

    disp('Retrain model on drift position ...');
    RetrainedMdl = optimGPR(TrainDS, TestDS, GPROptions, 0);
    close all;

    disp('Measure errors on retrained model ...')
    absErrorRetrained              = lossDS(RetrainedMdl, TestDS);
    meanError2Retrained(iDrift, 2) = mean(absErrorRetrained);
    maxError2Retrained(iDrift, 2)  = max(absErrorRetrained);

    disp('Track model parameters');
    s2nParamDrift(iDrift, 2) = RetrainedMdl.s2n;
    s2fParamDrift(iDrift, 2) = RetrainedMdl.theta(1);
    slParamDrift(iDrift, 2)  = RetrainedMdl.theta(2);

    iDrift = iDrift + 1;
end
\end{lstlisting}


{\footnotesize\textbf{Run Drift in Z}}

\begin{lstlisting}[style=Matlab-editor, basicstyle=\ttfamily\scriptsize]
disp('Run drift in z');
iDrift = 1;
for iFile = driftZIndex
    fprintf('Load datasets (%d) ...\n', iFile);
    TrainDS = load(fullfile(trainPath, TrainFiles(iFile).name));
    TestDS  = load(fullfile(testPath, TestFiles(iFile).name));

    fprintf('Measure errors z drift %.2f to ref ...\n', driftZ(iDrift));
    absError2Ref             = lossDS(RefMdl, TestDS);
    meanError2Ref(iDrift, 3) = mean(absError2Ref);
    maxError2Ref(iDrift, 3)  = max(absError2Ref);

    disp('Retrain model on drift position ...');
    RetrainedMdl = optimGPR(TrainDS, TestDS, GPROptions, 0);
    close all;

    disp('Measure errors on retrained model ...')
    absErrorRetrained              = lossDS(RetrainedMdl, TestDS);
    meanError2Retrained(iDrift, 3) = mean(absErrorRetrained);
    maxError2Retrained(iDrift, 3)  = max(absErrorRetrained);

    disp('Track model parameters');
    s2nParamDrift(iDrift, 3) = RetrainedMdl.s2n;
    s2fParamDrift(iDrift, 3) = RetrainedMdl.theta(1);
    slParamDrift(iDrift, 3)  = RetrainedMdl.theta(2);

    iDrift = iDrift + 1;
end
\end{lstlisting}


{\footnotesize\textbf{Run Drift in Tilt}}

\begin{lstlisting}[style=Matlab-editor, basicstyle=\ttfamily\scriptsize]
disp('Run drift in tilt');
iDrift = 1;
for iFile = driftTiltIndex
    fprintf('Load datasets (%d) ...\n', iFile);
    TrainDS = load(fullfile(trainPath, TrainFiles(iFile).name));
    TestDS  = load(fullfile(testPath, TestFiles(iFile).name));

    fprintf('Measure errors tilt drift %.2f to ref ...\n', driftTilt(iDrift));
    absError2Ref             = lossDS(RefMdl, TestDS);
    meanError2Ref(iDrift, 4) = mean(absError2Ref);
    maxError2Ref(iDrift, 4)  = max(absError2Ref);

    disp('Retrain model on drift position ...');
    RetrainedMdl = optimGPR(TrainDS, TestDS, GPROptions, 0);
    close all;

    disp('Measure errors on retrained model ...')
    absErrorRetrained              = lossDS(RetrainedMdl, TestDS);
    meanError2Retrained(iDrift, 4) = mean(absErrorRetrained);
    maxError2Retrained(iDrift, 4)  = max(absErrorRetrained);

    disp('Track model parameters');
    s2nParamDrift(iDrift, 4) = RetrainedMdl.s2n;
    s2fParamDrift(iDrift, 4) = RetrainedMdl.theta(1);
    slParamDrift(iDrift, 4)  = RetrainedMdl.theta(2);

    iDrift = iDrift + 1;
end
\end{lstlisting}


{\footnotesize\textbf{Plot Drift Errors}}

\begin{lstlisting}[style=Matlab-editor, basicstyle=\ttfamily\scriptsize]
% clc; close all;

figure('Name', 'Errors', 'Units', 'normalize', 'OuterPosition', [0 0 1 1]);
t = tiledlayout(1, 4);
bgAx = axes(t,'XTick',[],'YTick',[],'Box','off');
bgAx.Layout.TileSpan = [1 4];

% plot drift in x
ax1 = axes(t);
ax1.Box = 'off';
driftXTick = 1:driftXN;

hold on;
yline(ax1, find(driftX == refX), '--', 'LineWidth', 4.5);
xline(ax1, maxErrorTDK, 'r', 'LineWidth', 4.5);

plot(ax1, meanError2Ref(:,1), driftXTick', 'mo:', 'LineWidth', 3.5);
plot(ax1, maxError2Ref(:,1), driftXTick', 'bo:', 'LineWidth', 3.5);
plot(ax1, meanError2Retrained(:,1), driftXTick', 'ms-', 'LineWidth', 3.5);
plot(ax1, maxError2Retrained(:,1), driftXTick', 'bs-', 'LineWidth', 3.5);

[minMeanErrX, iMeanErrX] = min(meanError2Retrained(:,1));
s1 = scatter(ax1, minMeanErrX, iMeanErrX, 120, 'ys', 'filled', 'MarkerEdgeColor', ...
    'k', 'LineWidth', 1);

[minMaxErrX, iMaxErrX] = min(maxError2Retrained(:,1));
s2 = scatter(ax1, minMaxErrX, iMaxErrX, 120, 'gs', 'filled', 'MarkerEdgeColor', ...
    'k', 'LineWidth', 1);

yticks(ax1, driftXTick);
ylim(ax1, [0 driftXN + 1]);
yticklabels(ax1, driftX);
ax1.YTickLabel(2:2:end) = {''};
% ax1.YTickLabel(3:4:end) = {''};

ax1.XAxis.Scale = 'log';
xticks(ax1, errorTick);
xlim(ax1, [0.04 400]);

xlabel(ax1, '$\epsilon_{abs}$ in $^\circ$');
ylabel(ax1, sprintf('Drift from Ref.: $(%.1f,%.1f,%.1f)^T$ mm, $%.1f^\\circ$', ...
    refX, refY, refZ, refTilt)),

subtitle(ax1, 'Drift x in $0.25$ mm Steps');

% plot drift in y
ax2 = axes(t);
ax2.Layout.Tile = 2;
ax2.YAxis.Visible = 'on';
ax2.Box = 'off';

driftYTick = 1:driftYN;

hold on;
yline(ax2, find(driftY == refY), '--', 'LineWidth', 4.5);
xline(ax2, maxErrorTDK, 'r', 'LineWidth', 4.5);

plot(ax2, meanError2Ref(:,2), driftYTick', 'mo:', 'LineWidth', 3.5);
plot(ax2, maxError2Ref(:,2), driftYTick', 'bo:', 'LineWidth', 3.5);
plot(ax2, meanError2Retrained(:,2), driftYTick', 'ms-', 'LineWidth', 3.5);
plot(ax2, maxError2Retrained(:,2), driftYTick', 'bs-', 'LineWidth', 3.5);

[minMeanErrY, iMeanErrY] = min(meanError2Retrained(:,2));
s3 = scatter(ax2, minMeanErrY, iMeanErrY, 140, 'yh', 'filled', 'MarkerEdgeColor', ...
    'k', 'LineWidth', 1);

[minMaxErrY, iMaxErrY] = min(maxError2Retrained(:,2));
s4 = scatter(ax2, minMaxErrY, iMaxErrY, 140, 'gh', 'filled', 'MarkerEdgeColor', ...
    'k', 'LineWidth', 1);

yticks(ax2, driftYTick);
ylim(ax2, [0 driftYN + 1]);
yticklabels(ax2, driftY);
ax2.YTickLabel(2:2:end) = {''};
% ax2.YTickLabel(3:4:end) = {''};

ax2.XAxis.Scale = 'log';
xticks(ax2, errorTick);
xlim(ax2, [0.04 400]);

xlabel(ax2, '$\epsilon_{abs}$ in $^\circ$');

subtitle(ax2, 'Drift y in $0.25$ mm Steps');

% Link the axes
linkaxes([ax1 ax2], 'y')

% plot drift in y
ax3 = axes(t);
ax3.Layout.Tile = 3;
ax3.YAxis.Visible = 'on';
ax3.Box = 'off';

driftZTick = 1:driftZN;

hold on;
yline(ax3, find(driftZ == refZ), '--', 'LineWidth', 4.5);
xline(ax3, maxErrorTDK, 'r', 'LineWidth', 4.5);

plot(ax3, meanError2Ref(:,3), driftZTick', 'mo:', 'LineWidth', 3.5);
plot(ax3, maxError2Ref(:,3), driftZTick', 'bo:', 'LineWidth', 3.5);
plot(ax3, meanError2Retrained(:,3), driftZTick', 'ms-', 'LineWidth', 3.5);
plot(ax3, maxError2Retrained(:,3), driftZTick', 'bs-', 'LineWidth', 3.5);

[minMeanErrZ, iMeanErrZ] = min(meanError2Retrained(:,3));
s5 = scatter(ax3, minMeanErrZ, iMeanErrZ, 120, 'yd', 'filled', 'MarkerEdgeColor', ...
    'k', 'LineWidth', 1);

[minMaxErrZ, iMaxErrZ] = min(maxError2Retrained(:,3));
s6 = scatter(ax3, minMaxErrZ, iMaxErrZ, 120, 'gd', 'filled', 'MarkerEdgeColor', ...
    'k', 'LineWidth', 1);

yticks(ax3, driftZTick);
ylim(ax3, [0 driftZN + 1]);
yticklabels(ax3, driftZ);
ax3.YTickLabel(2:2:end) = {''};
% ax3.YTickLabel(3:4:end) = {''};

ax3.XAxis.Scale = 'log';
xticks(ax3, errorTick);
xlim(ax3, [0.04 400]);

xlabel(ax3, '$\epsilon_{abs}$ in $^\circ$');

subtitle(ax3, 'Drift z in $0.25$ mm Steps');

% plot drift in tilt
ax4 = axes(t);
ax4.Layout.Tile = 4;
ax4.YAxis.Visible = 'on';
ax4.Box = 'off';

driftTiltTick = 1:driftTiltN;

hold on;
l1 = yline(ax4, find(driftTilt == refTilt), '--', 'LineWidth', 4.5);
l2 = xline(ax4, maxErrorTDK, 'r', 'LineWidth', 4.5);

p1 = plot(ax4, meanError2Ref(:,4), driftTiltTick', 'mo:', 'LineWidth', 3.5);
p2 = plot(ax4, maxError2Ref(:,4), driftTiltTick', 'bo:', 'LineWidth', 3.5);
p3 = plot(ax4, meanError2Retrained(:,4), driftTiltTick', 'ms-', 'LineWidth', 3.5);
p4= plot(ax4, maxError2Retrained(:,4), driftTiltTick', 'bs-', 'LineWidth', 3.5);

[minMeanErrTilt, iMeanErrTilt] = min(meanError2Retrained(:,4));
s7 = scatter(ax4, minMeanErrTilt, iMeanErrTilt, 120, 'yo', 'filled', ...
    'MarkerEdgeColor', 'k', 'LineWidth', 1);

[minMaxErrTilt, iMaxErrTilt] = min(maxError2Retrained(:,4));
s8 = scatter(ax4, minMaxErrTilt, iMaxErrTilt, 120, 'go', 'filled', ...
    'MarkerEdgeColor', 'k', 'LineWidth', 1);

yticks(ax4, driftTiltTick);
ylim(ax4, [0 driftTiltN + 1]);
yticklabels(ax4, driftTilt);
ax4.YTickLabel(2:2:end) = {''};
% ax4.YTickLabel(3:4:end) = {''};

ax4.XAxis.Scale = 'log';
xticks(ax4, errorTick);
xlim(ax4, [0.04 400]);

xlabel(ax4, '$\epsilon_{abs}$ in $^\circ$');

subtitle(ax4, 'Drift $\alpha_y$ in $0.5^\circ$ Steps');

legend([l1, l2, p1, p2, p3, p4, s1, s2, s3, s4, s5, s6, s7, s8], ...
    {'Start/Ref', 'TDK Max $\rightarrow 0.6^\circ$', 'Mean to Ref', 'Max to Ref', ...
    'Mean Retrained', 'Max Retrained', ...
    sprintf('Min $\\rightarrow %.2f^\\circ$', minMeanErrX), ...
    sprintf('Min $\\rightarrow %.2f^\\circ$', minMaxErrX), ...
    sprintf('Min $\\rightarrow %.2f^\\circ$', minMeanErrY), ...
    sprintf('Min $\\rightarrow %.2f^\\circ$', minMaxErrY), ...
    sprintf('Min $\\rightarrow %.2f^\\circ$', minMeanErrZ), ...
    sprintf('Min $\\rightarrow %.2f^\\circ$', minMaxErrZ), ...
    sprintf('Min $\\rightarrow %.2f^\\circ$', minMeanErrTilt), ...
    sprintf('Min $\\rightarrow %.2f^\\circ$', minMaxErrTilt)}, ...
    'Location', 'bestoutside')
\end{lstlisting}


{\footnotesize\textbf{Plot Model Parameter Drift in X and Y}}

\begin{lstlisting}[style=Matlab-editor, basicstyle=\ttfamily\scriptsize]
% clc; close all;

figure('Name', 'Param X Y', 'Units', 'normalize', 'OuterPosition', [0 0 1 1]);
t1 = tiledlayout(1, 3);
bgAx1 = axes(t1,'XTick',[],'YTick',[],'Box','off');
bgAx1.Layout.TileSpan = [1 3];
title(bgAx1, 'a) Retrained Model Parameter in Horizontal Drifts')

% plot parameter drift in x and y
ax5 = axes(t1);
ax5.Box = 'off';

hold on;

xline(ax5, 1e-6, 'k-', 'LineWidth', 4.5);
xline(ax5, 1e-4, 'k-', 'LineWidth', 4.5);
xline(ax5, 6e-7, 'r-', 'LineWidth', 4.5);
xline(ax5, 1e-5, 'r-', 'LineWidth', 4.5);
xline(ax5, 3e-4, 'r-.', 'LineWidth', 4.5);
xline(ax5, 3e-7, 'r-.', 'LineWidth', 4.5);
xline(ax5, 1e-7, 'g-', 'LineWidth', 4.5);
xline(ax5, 1e-3, 'g-', 'LineWidth', 4.5);

plot(ax5, s2nParamDrift(:,1), driftXTick', 'bo-', 'LineWidth', 3.5)
plot(ax5, s2nParamDrift(:,2), driftYTick', 'mo-', 'LineWidth', 3.5)
ax5.XAxis.Scale = 'log';
xlabel(ax5, '$\sigma_n^2$');
xlim(ax5, [5e-8 1e-3]);
xticks(ax5, [1e-7, 1e-6, 1e-5, 1e-4, 1e-3]);
ylabel(ax5, 'Horizontal Drift $x$ in mm')
yticks(ax5, driftXTick);
ylim(ax5, [0 driftXN + 1]);
yticklabels(ax5, driftX);
ax5.YTickLabel(2:2:end) = {''};
ax5.YAxis.Color = 'b';
ax5.GridColor = [0.1500    0.1500    0.1500];

ax6 = axes(t1);
ax6.Layout.Tile = 2;
ax6.YAxis.Visible = 'off';
ax6.Box = 'off';

hold on;

xline(ax6, 1e0, 'k-', 'LineWidth', 4.5);
xline(ax6, 1e1, 'k-', 'LineWidth', 4.5);
xline(ax6, 4e-1, 'r-', 'LineWidth', 4.5);
xline(ax6, 2e1, 'r-', 'LineWidth', 4.5);
xline(ax6, 5e-1, 'r-.', 'LineWidth', 4.5);
xline(ax6, 2e2, 'r-.', 'LineWidth', 4.5);
xline(ax6, 1e-1, 'g-', 'LineWidth', 4.5);
xline(ax6, 1e2, 'g-', 'LineWidth', 4.5);

plot(ax6, s2fParamDrift(:,1), driftXTick', 'bo-', 'LineWidth', 3.5)
plot(ax6, s2fParamDrift(:,2), driftYTick', 'mo-', 'LineWidth', 3.5)
ax6.XAxis.Scale = 'log';
xlabel(ax6, '$\sigma_f^2$');
xlim(ax6, [1e-1 2e2]);
xticks(ax6, [1e-1, 1e0, 1e1, 1e2]);
yticks(ax6, driftXTick);
ylim(ax6, [0 driftXN + 1]);

ax7 = axes(t1);
ax7.Layout.Tile = 3;
ax7.YAxis.Visible = 'on';
ax7.Box = 'off';
ax7.YAxisLocation = 'right';
ax7.YAxis.Color = 'm';
ax7.GridColor = [0.1500    0.1500    0.1500];

hold on;

l3 = xline(ax7, 1e1, 'k-', 'LineWidth', 4.5);
xline(ax7, 3e1, 'k-', 'LineWidth', 4.5);
l4 = xline(ax7, 4e0, 'r-', 'LineWidth', 4.5);
xline(ax7, 4e1, 'r-', 'LineWidth', 4.5);
l5 = xline(ax7, 5e0, 'r-.', 'LineWidth', 4.5);
xline(ax7, 9e1, 'r-.', 'LineWidth', 4.5);
l6 = xline(ax7, 1e0, 'g-', 'LineWidth', 4.5);
xline(ax7, 1e2, 'g-', 'LineWidth', 4.5);

plot(ax7, slParamDrift(:,1), driftXTick', 'bo-', 'LineWidth', 3.5);
plot(ax7, slParamDrift(:,2), driftYTick', 'mo-', 'LineWidth', 3.5);
ax7.XAxis.Scale = 'log';
xlabel(ax7, '$\sigma_l$');
xlim(ax7, [1e0 2e2]);
xticks(ax7, [1e0, 1e1, 1e2]);
ylabel(ax7, 'Horizontal Drift $y$ in mm')
yticks(ax7, driftYTick);
ylim(ax7, [0 driftYN + 1]);
yticklabels(ax7, driftY);
ax7.YTickLabel(2:2:end) = {''};

linkaxes([ax5 ax6, ax7], 'y')

legend(ax6, [l3, l4, l5, l6], ...
    {'Limits Reference', ...
     'Limits Horizontal', ...
     'Limits Vertical', ...
     'Limits Experiment',}, ...
    'Location', 'southeast');
\end{lstlisting}


{\footnotesize\textbf{Plot Model Parameter Drift in Z and Tilt}}

\begin{lstlisting}[style=Matlab-editor, basicstyle=\ttfamily\scriptsize]
% clc; close all;

figure('Name', 'Param Z Tilt', 'Units', 'normalize', 'OuterPosition', [0 0 1 1]);
t2 = tiledlayout(1, 3);
bgAx2 = axes(t2,'XTick',[],'YTick',[],'Box','off');
bgAx2.Layout.TileSpan = [1 3];
title(bgAx2, 'b) Retrained Model Parameter in Vertical Drifts')

% plot parameter drift in x and y
ax8 = axes(t2);
ax8.Box = 'off';
ax8.YAxis.Visible = 'on';
ax8.YAxis.Color = 'b';
ax8.GridColor = [0.1500    0.1500    0.1500];

hold on;

xline(ax8, 1e-6, 'k-', 'LineWidth', 4.5);
xline(ax8, 1e-4, 'k-', 'LineWidth', 4.5);
xline(ax8, 6e-7, 'r-', 'LineWidth', 4.5);
xline(ax8, 1e-5, 'r-', 'LineWidth', 4.5);
xline(ax8, 3e-4, 'r-.', 'LineWidth', 4.5);
xline(ax8, 3e-7, 'r-.', 'LineWidth', 4.5);
xline(ax8, 1e-7, 'g-', 'LineWidth', 4.5);
xline(ax8, 1e-3, 'g-', 'LineWidth', 4.5);

plot(ax8, s2nParamDrift(:,3), driftZTick', 'bo-', 'LineWidth', 3.5)
plot(ax8, s2nParamDrift(:,4), driftTiltTick', 'mo-', 'LineWidth', 3.5)
ax8.XAxis.Scale = 'log';
xlabel(ax8, '$\sigma_n^2$');
xlim(ax8, [5e-8 1e-3]);
xticks(ax8, [1e-7, 1e-6, 1e-5, 1e-4, 1e-3]);
ylabel(ax8, 'Vertical Drift $z$ in mm')
yticks(ax8, driftZTick);
ylim(ax8, [0 driftZN + 1]);
yticklabels(ax8, driftZ);
ax8.YTickLabel(2:2:end) = {''};

ax9 = axes(t2);
ax9.Layout.Tile = 2;
ax9.YAxis.Visible = 'off';
ax9.Box = 'off';

hold on;

l7 = xline(ax9, 1e0, 'k-', 'LineWidth', 4.5);
xline(ax9, 1e1, 'k-', 'LineWidth', 4.5);
l8 = xline(ax9, 4e-1, 'r-', 'LineWidth', 4.5);
xline(ax9, 2e1, 'r-', 'LineWidth', 4.5);
l9 = xline(ax9, 5e-1, 'r-.', 'LineWidth', 4.5);
xline(ax9, 2e2, 'r-.', 'LineWidth', 4.5);
l10 = xline(ax9, 1e-1, 'g-', 'LineWidth', 4.5);
xline(ax9, 1e2, 'g-', 'LineWidth', 4.5);

plot(ax9, s2fParamDrift(:,3), driftZTick', 'bo-', 'LineWidth', 3.5)
plot(ax9, s2fParamDrift(:,4), driftTiltTick', 'mo-', 'LineWidth', 3.5)
ax9.XAxis.Scale = 'log';
xlabel(ax9, '$\sigma_f^2$');
xlim(ax9, [1e-1 2e2]);
xticks(ax9, [1e-1, 1e0, 1e1, 1e2]);
yticks(ax9, driftZTick);
ylim(ax9, [0 driftZN + 1]);

ax10 = axes(t2);
ax10.Layout.Tile = 3;
ax10.YAxis.Visible = 'on';
ax10.Box = 'off';
ax10.YAxisLocation = 'right';
ax10.YAxis.Color = 'm';
ax10.GridColor = [0.1500    0.1500    0.1500];

hold on;

xline(ax10, 1e1, 'k-', 'LineWidth', 4.5);
xline(ax10, 3e1, 'k-', 'LineWidth', 4.5);
xline(ax10, 4e0, 'r-', 'LineWidth', 4.5);
xline(ax10, 4e1, 'r-', 'LineWidth', 4.5);
xline(ax10, 5e0, 'r-.', 'LineWidth', 4.5);
xline(ax10, 9e1, 'r-.', 'LineWidth', 4.5);
xline(ax10, 1e0, 'g-', 'LineWidth', 4.5);
xline(ax10, 1e2, 'g-', 'LineWidth', 4.5);

plot(ax10, slParamDrift(:,3), driftZTick', 'bo-', 'LineWidth', 3.5)
plot(ax10, slParamDrift(:,4), driftTiltTick', 'mo-', 'LineWidth', 3.5)
ax10.XAxis.Scale = 'log';
xlabel(ax10, '$\sigma_l$');
xlim(ax10, [1e0 2e2]);
xticks(ax10, [1e0, 1e1, 1e2]);
ylabel(ax10, 'Vertical Drift $\alpha_y$ in $^\circ$')
yticks(ax10, driftTiltTick);
ylim(ax10, [0 driftTiltN + 1]);
yticklabels(ax10, driftTilt);
ax10.YTickLabel(2:2:end) = {''};

linkaxes([ax8 ax9, ax10], 'y')

legend(ax10, [l7, l8, l9, l10], ...
    {'Limits Reference', ...
     'Limits Horizontal', ...
     'Limits Vertical', ...
     'Limits Experiment',}, ...
    'Location', 'southwest');
\end{lstlisting}



\end{document}

