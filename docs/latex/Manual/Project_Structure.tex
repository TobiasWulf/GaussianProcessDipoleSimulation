
% This LaTeX was auto-generated from MATLAB code.
% To make changes, update the MATLAB code and republish this document.

\documentclass{standalone}
\usepackage{graphicx}
\usepackage{listings}
\usepackage{xcolor}
\usepackage{textcomp}
\usepackage[framed, numbered]{matlab-prettifier}

\sloppy
\definecolor{lightgray}{gray}{0.5}
\setlength{\parindent}{0pt}

\begin{document}

    
    
\section*{Project Structure}

\begin{par}
A good project directory structure is the key to build scalable and expandable software projects. Therfore each project folder has to fulfill an associated task. Additionally, a good structure facilitates project navigation and the retrieval and reuse of project content. Further on Matlab provides stratgies to add content to existing project structures and label it for script based execution of project task to manage project files. To add new content have a look at the links below.
\end{par} \vspace{1em}

\subsection*{Contents}

\begin{itemize}
\setlength{\itemsep}{-1ex}
   \item See Also
   \item Directory Tasks
   \item Add New Elements
\end{itemize}


\subsection*{See Also}

\begin{itemize}
\setlength{\itemsep}{-1ex}
   \item \begin{verbatim}Specify Project Path\end{verbatim}
   \item \begin{verbatim}Add Files to the Project\end{verbatim}
   \item \begin{verbatim}Add Labels to Files\end{verbatim}
\end{itemize}


\subsection*{Directory Tasks}

\begin{par}
\textbf{Directory} Task.
\end{par} \vspace{1em}
\begin{par}
\textbf{./} Main project directory which contains the Matlab project sandbox files and the hidden repository files. Matlab project sandbox directory. Project root directory which contains the Matlab project file, the info.xml, .gitignore, .gitattributes files and all other project related subdirectories. Startup directory.
\end{par} \vspace{1em}
\begin{par}
\textbf{./.git} Hidden repository for local standalone work. Saves daily working results. Provide a Git clonable instance of sandbox the directory. Replacable. Not Matlab driven, simulates remote repository.
\end{par} \vspace{1em}
\begin{par}
\textbf{./resources} Autogenerated directory from Matlab project. Contains the local project versioning and project xml-files.
\end{par} \vspace{1em}
\begin{par}
\textbf{./data} Contains all project related datasets e.g. mat-files.
\end{par} \vspace{1em}
\begin{par}
\textbf{./data/trainig} Contains mat-files from sensor array simulation for training cases of the gaussian process.
\end{par} \vspace{1em}
\begin{par}
\textbf{./data/test} Contains mat-files from sensor array simulation for test cases of the gaussian process.
\end{par} \vspace{1em}
\begin{par}
\textbf{./docs} Documentation directory which contains m-files only for documentation use and the directory where all project remarked files are published into HTML output files.
\end{par} \vspace{1em}
\begin{par}
\textbf{./docs/html} Publish directory where published m-files are collected and bind to a Matlab help browser readable documentation. It contains html-files and subdirectory for images and figures which are used in the documentaion. The help browser search database is placed here too. Much more important the directory contains the helptoc.xml which pointed by the info.xml from root project directory.
\end{par} \vspace{1em}
\begin{par}
\textbf{./docs/html/helpsearch} Contains autogenerated help search database entries. The directory is rewritten during the publish documentation process.
\end{par} \vspace{1em}
\begin{par}
\textbf{./docs/html/images} Contains all needed image files like png-files which are used in the documentation (HTML).
\end{par} \vspace{1em}
\begin{par}
\textbf{./docs/latex} Documentation directory which LaTeX documentation of the project including subfolders for Thesis of each project participant.
\end{par} \vspace{1em}
\begin{par}
\textbf{./docs/latex/BA\_Thesis\_Tobias\_Wulf} Bachelor Thesis directory of Tobias Wulf.
\end{par} \vspace{1em}
\begin{par}
\textbf{./docs/latex/Manual} Export directory for documentation written in Matlab as pdf export.
\end{par} \vspace{1em}
\begin{par}
\textbf{./scripts} The sripts directory contains all executable script m-files to solve certain tasks in the project, to generate datasets or execute parts of the toolbox source code.
\end{par} \vspace{1em}
\begin{par}
\textbf{./src} Source code directory which contains reusable source code clustered in submodule directories. The code can be function oriented or class oriented or a mix of both. Contains no bare script files.
\end{par} \vspace{1em}
\begin{par}
\textbf{./src/sensorArraySimulation} Sensor Array Simulation function and class. Contains functions, mathematical functions and classes to simulate an N x N sensor array on base of the TDK TAS2141 characterization dataset.
\end{par} \vspace{1em}
\begin{par}
\textbf{./src/gaussianProcessRegression} Gaussian Process Regression module which contains basic math functions and submodules to implement GPR models with different kernels using same regression and optimization process.
\end{par} \vspace{1em}
\begin{par}
\textbf{./src/gaussianProcessRegression/basicMathFunctions} Basic math functions to perform GPR angular predictions.
\end{par} \vspace{1em}
\begin{par}
\textbf{./src/gaussianProcessRegression/kernelQFC} Exact Quadratic Fractional Covariance kernel functions which bases on matrice training data.
\end{par} \vspace{1em}
\begin{par}
\textbf{./src/gaussianProcessRegression/kernelQFCAPX} Approximated Quadratic Fractional Covariance kernel functions which bases on vector training data and uses norm scalar presentation of input matrix data. Using triangle inequation to norm matrix data before compute the covariances.
\end{par} \vspace{1em}
\begin{par}
\textbf{./src/util} Util function and class space. Function and class source code to solve upcoming help tasks e.g. to manage project content, to support plot framework or reporting or publishing processes.
\end{par} \vspace{1em}
\begin{par}
\textbf{./src/util/plotFunctions} Contains plot functions for reuse.
\end{par} \vspace{1em}
\begin{par}
\textbf{./tests} For test driven development each function or class needs a own test space or file. The directory contains these tests.
\end{par} \vspace{1em}
\begin{par}
\textbf{./temp} Temporally working directory to save intermediate results or the last software state from session before or scratch files which flies arround. Either made pre investigasition and supported work basics are saved here.
\end{par} \vspace{1em}


\subsection*{Add New Elements}

\begin{par}
\textbf{Add new folder to project:}
\end{par} \vspace{1em}
\begin{enumerate}
\setlength{\itemsep}{-1ex}
   \item Create a new folder and add to Project Path after Matlab flow.
   \item Run Checks \textbf{\ensuremath{>}} Add Files.
   \item Run tree command from shell to update directory for the documentation   (optional).
   \item Update directorry task table of this document.
\end{enumerate}
\begin{par}
\textbf{Add new file to project:}
\end{par} \vspace{1em}
\begin{enumerate}
\setlength{\itemsep}{-1ex}
   \item Create new File and edit the file after Documentation Workflow.   and Conventions.
   \item Run Checks \textbf{\ensuremath{>}} Add Files.
   \item Label the new file from project pane.
   \item Commit file into active branch.
   \item Registrate to the documentation if needed (publish, toc and listings docs).
\end{enumerate}
\begin{par}
Created on October 10. 2020 by Tobias Wulf. Copyright Tobias Wulf 2020.
\end{par} \vspace{1em}
\begin{par}

\end{par} \vspace{1em}



\end{document}

