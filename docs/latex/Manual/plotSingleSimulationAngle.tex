
% This LaTeX was auto-generated from MATLAB code.
% To make changes, update the MATLAB code and republish this document.

\documentclass{standalone}
\usepackage{graphicx}
\usepackage{listings}
\usepackage{xcolor}
\usepackage{textcomp}
\usepackage[framed, numbered]{matlab-prettifier}

\sloppy
\definecolor{lightgray}{gray}{0.5}
\setlength{\parindent}{0pt}

\begin{document}

    
    \begin{par}
Search for available trainings or test dataset and plot dataset. Follow user input dialog to choose which dataset and decide how many angles to plot. Plot single Angle and save figure to file. File name same as dataset with attach angle index.
\end{par} \vspace{1em}


{\textbf{Syntax}}

\begin{lstlisting}[style=Matlab-editor, basicstyle=\ttfamily\scriptsize]
plotSingleSimulationAngle()
\end{lstlisting}


{\textbf{Description}}

\begin{par}
\textbf{plotSingleSimulationAngle()} plot training or test dataset which are loacated in data/test or data/training. The function lists all datasets and the user must decide during user input dialog which dataset to plot and which angle to visualize to. It loads path from config.mat and scans for file automatically.
\end{par} \vspace{1em}


{\textbf{Examples}}

\begin{lstlisting}[style=Matlab-editor, basicstyle=\ttfamily\scriptsize]
plotSingleSimulationAngle()
\end{lstlisting}


{\textbf{Input Argurments}}

\begin{par}
\textbf{None}
\end{par} \vspace{1em}


{\textbf{Output Argurments}}

\begin{par}
\textbf{None}
\end{par} \vspace{1em}


{\textbf{Requirements}}

\begin{itemize}
\setlength{\itemsep}{-1ex}
   \item Other m-files required: None
   \item Subfunctions: None
   \item MAT-files required: config.mat
\end{itemize}


{\textbf{See Also}}

\begin{itemize}
\setlength{\itemsep}{-1ex}
   \item \begin{verbatim}generateSimulationDatasets\end{verbatim}
   \item \begin{verbatim}sensorArraySimulation\end{verbatim}
   \item \begin{verbatim}generateConfigMat\end{verbatim}
\end{itemize}
\begin{par}
Created on November 28. 2020 by Tobias Wulf. Copyright Tobias Wulf 2020.
\end{par} \vspace{1em}
\begin{par}

\end{par} \vspace{1em}
\begin{lstlisting}[style=Matlab-editor, basicstyle=\ttfamily\scriptsize]
function plotSingleSimulationAngle()
    % scan for datasets and load needed configurations %%%%%%%%%%%%%%%%%%%%%%%%%
    %%%%%%%%%%%%%%%%%%%%%%%%%%%%%%%%%%%%%%%%%%%%%%%%%%%%%%%%%%%%%%%%%%%%%%%%%%%%
    try
        disp('Plot single simulation angle ...');
        close all;
        % load path variables
        load('config.mat', 'PathVariables');
        % scan for datasets
        TrainingDatasets = dir(fullfile(PathVariables.trainingDataPath, ...
            'Training_*.mat'));
        TestDatasets = dir(fullfile(PathVariables.testDataPath, 'Test_*.mat'));
        allDatasets = [TrainingDatasets; TestDatasets];
        % check if files available
        if isempty(allDatasets)
            error('No training or test datasets found.');
        end
    catch ME
        rethrow(ME)
    end

    % display availabe datasets to user, decide which to plot %%%%%%%%%%%%%%%%%%
    %%%%%%%%%%%%%%%%%%%%%%%%%%%%%%%%%%%%%%%%%%%%%%%%%%%%%%%%%%%%%%%%%%%%%%%%%%%%

    % number of datasets
    nDatasets = length(allDatasets);
    fprintf('Found %d datasets:\n', nDatasets)
    for i = 1:nDatasets
        fprintf('%s\t:\t(%d)\n', allDatasets(i).name, i)
    end
    % get numeric user input to indicate which dataset to plot
    iDataset = input('Type number to choose dataset to plot to: ');
    % iDataset = 2;

    % load dataset and ask user which one and how many angles %%%%%%%%%%%%%%%%%%
    %%%%%%%%%%%%%%%%%%%%%%%%%%%%%%%%%%%%%%%%%%%%%%%%%%%%%%%%%%%%%%%%%%%%%%%%%%%%
    try
        ds = load(fullfile(allDatasets(iDataset).folder, ...
            allDatasets(iDataset).name));
        % check how many angles in dataset and let user decide how many to
        % render in polt
        fprintf('Detect %d angles ([1:%d]) in dataset ...\n', ...
            ds.Info.UseOptions.nAngles, ds.Info.UseOptions.nAngles);
        fprintf('Resolution\t:\t%.1f\n', ds.Info.UseOptions.angleRes);
        fprintf('Step width\t:\t%.1f\n', ds.Data.angleStep);
        fprintf('Start angle\t:\t%.1f\n', ds.Data.angles(1))
        idx = input('Which angle do you wish to plot (enter index): ');
        angle = interp1(ds.Data.angles, idx, 'nearest');
    catch ME
        rethrow(ME)
    end

    % figure save path for different formats %%%%%%%%%%%%%%%%%%%%%%%%%%%%%%%%%%%
    %%%%%%%%%%%%%%%%%%%%%%%%%%%%%%%%%%%%%%%%%%%%%%%%%%%%%%%%%%%%%%%%%%%%%%%%%%%%
    fPath = PathVariables.saveImagesPath;

    % create dataset figure for a subset or all angle %%%%%%%%%%%%%%%%%%%%%%%%%%
    %%%%%%%%%%%%%%%%%%%%%%%%%%%%%%%%%%%%%%%%%%%%%%%%%%%%%%%%%%%%%%%%%%%%%%%%%%%%
    fig = figure('Name', 'Sensor Array', ...
        'NumberTitle' , 'off', ...
        'WindowStyle', 'normal', ...
        'Position', [4381 15 1244 983], ...
        'Units', 'pixels', ...
        'WindowState', 'maximized');

    tdl = tiledlayout(fig, 2, 2, ...
        'Padding', 'normal', ...
        'TileSpacing' , 'compact');


    disp('Sensor Array Simulation');

    subline1 = "Sensor Array (%s) of %dx%d sensors, " + ...
        "an edge length of %.1f mm, a rel. pos. to magnet surface of";
    subline2 = " (%.1f, %.1f, -(%.1f)) in mm, a magnet tilt" + ...
        " of %.1f, a sphere radius of %.1f mm, a imprinted";
    subline3 = "field strength of %.1f kA/m at %.1f mm from" + ...
        " sphere surface in z-axis, %d rotation angles with a ";
    subline4 = "step width of %.1f and a resolution of" + ...
        " %.1f. Visualized is rotatation angle %d (%.1f)$.";
    subline5 = "Based on %s characterization reference %s.";
    sub = [sprintf(subline1, ...
                   ds.Info.SensorArrayOptions.geometry, ...
                   ds.Info.SensorArrayOptions.dimension, ...
                   ds.Info.SensorArrayOptions.dimension, ...
                   ds.Info.SensorArrayOptions.edge); ...
           sprintf(subline2, ...
                   ds.Info.UseOptions.xPos, ...
                   ds.Info.UseOptions.yPos, ...
                   ds.Info.UseOptions.zPos, ...
                   ds.Info.UseOptions.tilt, ...
                   ds.Info.DipoleOptions.sphereRadius); ...
           sprintf(subline3, ...
                   ds.Info.DipoleOptions.H0mag, ...
                   ds.Info.DipoleOptions.z0, ...
                   ds.Info.UseOptions.nAngles); ...
           sprintf(subline4, ...
                   ds.Data.angleStep, ...
                   ds.Info.UseOptions.angleRes, ...
                   idx, angle)
           sprintf(subline5, ...
                   ds.Info.CharData, ...
                   ds.Info.UseOptions.BridgeReference)];

    disp(sub);

    % get subset of needed data to plot, only one load %%%%%%%%%%%%%%%%%%%%%%%%%
    %%%%%%%%%%%%%%%%%%%%%%%%%%%%%%%%%%%%%%%%%%%%%%%%%%%%%%%%%%%%%%%%%%%%%%%%%%%%
    N = ds.Info.SensorArrayOptions.dimension;
    X = ds.Data.X;
    Y = ds.Data.Y;
    Z = ds.Data.Z;

    % calc limits of plot 1
    maxX = ds.Info.UseOptions.xPos + ds.Info.SensorArrayOptions.edge;
    maxY = ds.Info.UseOptions.yPos + ds.Info.SensorArrayOptions.edge;
    minX = ds.Info.UseOptions.xPos - ds.Info.SensorArrayOptions.edge;
    minY = ds.Info.UseOptions.yPos - ds.Info.SensorArrayOptions.edge;

    % calculate colormap to identify scatter points
    c=zeros(N,N,3);
    for i = 1:N
        for j = 1:N
            c(i,j,:) = [(2*N+1-2*i), (2*N+1-2*j), (i+j)]/2/N;
        end
    end
    c = squeeze(reshape(c, N^2, 1, 3));

    % load offset voltage to subtract from cosinus, sinus voltage
    Voff = ds.Info.SensorArrayOptions.Voff;

    % plot sensor grid in x and y coordinates and constant z layer %%%%%%%%%%%%%
    %%%%%%%%%%%%%%%%%%%%%%%%%%%%%%%%%%%%%%%%%%%%%%%%%%%%%%%%%%%%%%%%%%%%%%%%%%%%
    ax1 = nexttile(1);
    % plot each cooredinate in loop to create a special shading constant
    % reliable to orientation for all matrice
    hold on;
    scatter(X(:), Y(:), [], c, 'filled', 'MarkerEdgeColor', 'k', ...
        'LineWidth', 0.8);

    % axis shape and ticks
    axis square xy;
    axis tight;
    grid on;
    xlim([minX maxX]);
    ylim([minY maxY]);

    % text and labels
    text(minX+0.2, minY+0.2, ...
        sprintf('$Z = %.1f$ mm', Z(1)), ...
        'Color', 'k', ...
        'FontSize', 20, ...
        'FontName', 'Times', ...
        'Interpreter', 'latex');

    xlabel('$X$ in mm');

    ylabel('$Y$ in mm');

    title(sprintf('a) Sensor-Array $%d\\times%d$', N, N));

    hold off;

    % plot rotation angles in polar view %%%%%%%%%%%%%%%%%%%%%%%%%%%%%%%%%%%%%%%
    %%%%%%%%%%%%%%%%%%%%%%%%%%%%%%%%%%%%%%%%%%%%%%%%%%%%%%%%%%%%%%%%%%%%%%%%%%%%
    nexttile(2);
    % plot all angles grayed out
    polarscatter(ds.Data.angles/180*pi, ones(1, ds.Info.UseOptions.nAngles), ...
        [], [0.8 0.8 0.8], 'filled');

    % radius ticks and label
    rticks(1);
    rticklabels("");
    hold on;

    % plot subset of angles
    % polarscatter(subAngles/180*pi, ones(1, nSubAngles),...
    %    'k', 'LineWidth', 0.8);
    ax2 = gca;

    % axis shape
    axis tight;

    % text an labels
    % init first rotation step label
    tA = text(2/3*pi, 1.5, ...
        '$\\theta$', ...
        'Color', 'b', ...
        'FontSize', 20, ...
        'FontName', 'Times', ...
        'Interpreter', 'latex');

    title('b) Rotation Angle');

    hold off;

    % Cosinus bridge outputs for rotation step %%%%%%%%%%%%%%%%%%%%%%%%%%%%%%%%%
    %%%%%%%%%%%%%%%%%%%%%%%%%%%%%%%%%%%%%%%%%%%%%%%%%%%%%%%%%%%%%%%%%%%%%%%%%%%%
    ax3 = nexttile(3);
    hold on;

    % set colormap
    colormap('gray');

    % plot cosinus reference, set NaN values to white color, orient Y to normal
    imC = imagesc(ds.Data.HxScale, ds.Data.HyScale, ds.Data.VcosRef);
    set(imC, 'AlphaData', ~isnan(ds.Data.VcosRef));
    set(gca, 'YDir', 'normal')

    % axis shape and ticks
    axis square xy;
    axis tight;
    yticks(xticks);
    grid on;

    % test and labels
    xlabel('$H_x$ in kA/m');

    ylabel('$H_y$ in kA/m');

    title('c) $V_{cos}(H_x, H_y)$ in V');

    % add colorbar and place it
    cb1 = colorbar;
    cb1.Label.String = sprintf(...
        '$V_{cc} = %1.1f$ V, $V_{off} = %1.2f$ V', ...
        ds.Info.SensorArrayOptions.Vcc, ds.Info.SensorArrayOptions.Voff);
    cb1.TickLabelInterpreter = 'latex';
    cb1.Label.Interpreter = 'latex';
    cb1.Label.FontSize = 20;

    hold off;

    % Sinus bridge outputs for rotation step %%%%%%%%%%%%%%%%%%%%%%%%%%%%%%%%%%%
    %%%%%%%%%%%%%%%%%%%%%%%%%%%%%%%%%%%%%%%%%%%%%%%%%%%%%%%%%%%%%%%%%%%%%%%%%%%%
    ax4 = nexttile(4);
    hold on;

    % set colormap
    colormap('gray');

    % plot sinus reference, set NaN values to white color, orient Y to normal
    imS = imagesc(ds.Data.HxScale, ds.Data.HyScale, ds.Data.VsinRef);
    set(imS, 'AlphaData', ~isnan(ds.Data.VsinRef));
    set(gca, 'YDir', 'normal')

    % axis shape and ticks
    axis square xy;
    axis tight;
    yticks(xticks);
    grid on;

    % test and labels
    xlabel('$H_x$ in kA/m');

    ylabel('$H_y$ in kA/m');

    title('d) $V_{sin}(H_x, H_y)$ in V');

    % add colorbar and place it
    cb2 = colorbar;
    cb2.Label.String = sprintf(...
        '$V_{cc} = %1.1f$ V, $V_{off} = %1.2f$ V', ...
        ds.Info.SensorArrayOptions.Vcc, ds.Info.SensorArrayOptions.Voff);
    cb2.TickLabelInterpreter = 'latex';
    cb2.Label.Interpreter = 'latex';
    cb2.Label.FontSize = 20;

    hold off;

    % zoom axes for scatter on cosinuns reference images %%%%%%%%%%%%%%%%%%%%%%%
    %%%%%%%%%%%%%%%%%%%%%%%%%%%%%%%%%%%%%%%%%%%%%%%%%%%%%%%%%%%%%%%%%%%%%%%%%%%%
    nexttile(3);
    ax5 = axes('Position', [0.15 0.115 0.12 0.12], ...
        'XColor', 'r', 'YColor', 'r');
    xticklabels(ax5, []);
    yticklabels(ax5, []);
    hold on;
    axis square xy;
    grid on;
    hold off;

    ax6 = axes('Position', [0.581 0.115 0.12 0.12], ...
        'XColor', 'r', 'YColor', 'r');
    xticklabels(ax6, []);
    yticklabels(ax6, []);
    hold on;
    axis square xy;
    grid on;
    hold off;

    % plot angle into plots %%%%%%%%%%%%%%%%%%%%%%%%%%%%%%%%%%%%%%%%%%%%%%%%%%%%
    %%%%%%%%%%%%%%%%%%%%%%%%%%%%%%%%%%%%%%%%%%%%%%%%%%%%%%%%%%%%%%%%%%%%%%%%%%%%
    % H load subset
    Hx = ds.Data.Hx(:,:,idx);
    Hy = ds.Data.Hy(:,:,idx);
    % get min max
    maxHx = max(Hx, [], 'all');
    maxHy = max(Hy, [], 'all');
    minHx = min(Hx, [], 'all');
    minHy = min(Hy, [], 'all');
    dHx = abs(maxHx - minHx);
    dHy = abs(maxHy - minHy);

    % load V subset
    Vcos = ds.Data.Vcos(:,:,idx) - Voff;
    Vsin = ds.Data.Vsin(:,:,idx) - Voff;
    angle = ds.Data.angles(idx);

    % lock plots
    hold(ax1, 'on');
    hold(ax2, 'on');
    hold(ax3, 'on');
    hold(ax4, 'on');
    hold(ax5, 'on');
    hold(ax6, 'on');

    % update plot 1
    qH = quiver(ax1, X, Y, Hx, Hy, 0.7, 'b');
    qV = quiver(ax1, X, Y, Vcos, Vsin, 0.7, 'r');
    legend([qH qV], {'$quiver(H_x,H_y)$', ...
        '$quiver(V_{cos}-V_{off},V_{sin}-V_{off})$'},...
        'FontSize', 14, ...
        'Location', 'NorthEast');

    % update plot 2
    tA.String = sprintf('$%.1f^\\circ$', angle);
    polarscatter(ax2, angle/180*pi, 1, 'b', 'filled', ...
        'MarkerEdgeColor', 'k', 'LineWidth', 0.8);

    % update plot 3 and 4
    scatter(ax3, Hx(:), Hy(:), 5, c, 'filled', 'MarkerEdgeColor', 'k', ...
        'LineWidth', 0.8);
    scatter(ax4, Hx(:), Hy(:), 5, c, 'filled', 'MarkerEdgeColor', 'k', ...
        'LineWidth', 0.8);

    % calc position of scatter area frame and reframe
    pos = [minHx - 0.3 * dHx, minHy - 0.3 * dHy, 1.6 * dHx, 1.6 * dHy];
    rectangle(ax3, 'Position', pos, 'LineWidth', 1.5,  'EdgeColor', 'r');
    rectangle(ax4, 'Position', pos, 'LineWidth', 1.5,  'EdgeColor', 'r');

    % update plot 5 (zoom)
    scatter(ax5, Hx(:), Hy(:), [], c, 'filled', 'MarkerEdgeColor', 'k', ...
        'LineWidth', 0.8);
    xlim(ax5, [pos(1) maxHx + 0.3 * dHx])
    ylim(ax5, [pos(2) maxHy + 0.3 * dHy])

     % update plot 6 (zoom)
    scatter(ax6, Hx(:), Hy(:), [], c, 'filled', 'MarkerEdgeColor', 'k', ...
        'LineWidth', 0.8);
    xlim(ax6, [pos(1) maxHx + 0.3 * dHx])
    ylim(ax6, [pos(2) maxHy + 0.3 * dHy])

    % release plots
    hold(ax1, 'off');
    hold(ax2, 'off');
    hold(ax3, 'off');
    hold(ax4, 'off');
    hold(ax5, 'off');
    hold(ax6, 'off');

    % save figure to file %%%%%%%%%%%%%%%%%%%%%%%%%%%%%%%%%%%%%%%%%%%%%%%%%%%%%%
    %%%%%%%%%%%%%%%%%%%%%%%%%%%%%%%%%%%%%%%%%%%%%%%%%%%%%%%%%%%%%%%%%%%%%%%%%%%%
    % get file path to save figure with angle index
    [~, fName, ~] = fileparts(ds.Info.filePath);

%     % save to various formats
%     yesno = input('Save? [y/n]: ', 's');
%     if strcmp(yesno, 'y')
%         fLabel = input('Enter file label: ', 's');
%         fName = fName + sprintf("_AnglePlot_%d_", idx) + fLabel;
%         savefig(fig, fullfile(fPath, fName));
%         print(fig, fullfile(fPath, fName), '-dsvg');
%         print(fig, fullfile(fPath, fName), '-depsc', '-tiff', '-loose');
%         print(fig, fullfile(fPath, fName), '-dpdf', '-loose', '-fillpage');
%     end
%     close(fig);
end
\end{lstlisting}



\end{document}

