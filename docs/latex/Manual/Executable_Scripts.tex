
% This LaTeX was auto-generated from MATLAB code.
% To make changes, update the MATLAB code and republish this document.

\documentclass{standalone}
\usepackage{graphicx}
\usepackage{listings}
\usepackage{xcolor}
\usepackage{textcomp}
\usepackage[framed, numbered]{matlab-prettifier}

\sloppy
\definecolor{lightgray}{gray}{0.5}
\setlength{\parindent}{0pt}

\begin{document}

    
    \begin{par}
Executable scripts of the project to solve various actions or project tasks. The main approach of project scripts is an automated way to collect and execute certain actions in an example to run project documentation at once or generate project configuration file which are used by other scripts or loaded by functions to control and execute task in a unified project structure.
\end{par} \vspace{1em}


{\footnotesize\textbf{compareGPRKernels}}

\begin{par}
Compares GPR kernel functions with each and another.
\end{par} \vspace{1em}


{\footnotesize\textbf{investigateKernelParameters}}

\begin{par}
Analyzes covariance kernel parameters with contour plots.
\end{par} \vspace{1em}


{\footnotesize\textbf{demoGPRModule}}

\begin{par}
Demonstrates the use of the gaussianProcessRegression module.
\end{par} \vspace{1em}


{\footnotesize\textbf{exportPublishedToPdf}}

\begin{par}
Export published HTML files to a pdf manual.
\end{par} \vspace{1em}


{\footnotesize\textbf{deleteSimulationPlots}}

\begin{par}
Delete simulation training and test dataset plots from figures and images path with training and test filename pattern.
\end{par} \vspace{1em}


{\footnotesize\textbf{deleteSimulationDatasets}}

\begin{par}
Delete generated simulation datasets from data path.
\end{par} \vspace{1em}


{\footnotesize\textbf{generateSimulationDatasets}}

\begin{par}
Generate simulation datasets from sensor array simulation configuration.
\end{par} \vspace{1em}


{\footnotesize\textbf{publishProjectFilesHTML}}

\begin{par}
Publish Matlab help browser integrated HTML documentation.
\end{par} \vspace{1em}


{\footnotesize\textbf{generateConfigMat}}

\begin{par}
Generate configuration for generic use or part use in different program layers.
\end{par} \vspace{1em}
\begin{par}
Created on September 21. 2020 by Tobias Wulf. Copyright Tobias Wulf 2020.
\end{par} \vspace{1em}
\begin{par}

\end{par} \vspace{1em}



\end{document}

