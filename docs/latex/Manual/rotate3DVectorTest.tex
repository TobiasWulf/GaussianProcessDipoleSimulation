
% This LaTeX was auto-generated from MATLAB code.
% To make changes, update the MATLAB code and republish this document.

\documentclass{standalone}
\usepackage{graphicx}
\usepackage{listings}
\usepackage{xcolor}
\usepackage{textcomp}
\usepackage[framed, numbered]{matlab-prettifier}

\sloppy
\definecolor{lightgray}{gray}{0.5}
\setlength{\parindent}{0pt}

\begin{document}

    
    
{\footnotesize\textbf{Contents}}

\begin{itemize}
\setlength{\itemsep}{-1ex}
   \item Test 1: output dimensions
   \item Test 2: rotate vectors in x-axes
   \item Test 3: rotate vectors in y-axes
   \item Test 4: rotate vectors in z-axes
\end{itemize}
\begin{par}

\end{par} \vspace{1em}
\begin{lstlisting}[style=Matlab-editor, basicstyle=\ttfamily\scriptsize]
% create column vectors with simple direction for rotations along the axes
% without tilts in other achses.
x = [-1; 0; 0];
y = [0; -1; 0];
z = [0; 0; -1];

% set angle step width in degree to rotate at choosen axes (x, y, or z)
angle = 90;
\end{lstlisting}


{\footnotesize\textbf{Test 1: output dimensions}}

\begin{lstlisting}[style=Matlab-editor, basicstyle=\ttfamily\scriptsize]
rotated = rotate3DVector(x, 0, 0, angle);
assert(isequal(size(rotated), [3, 1]))
rotated = rotate3DVector([x x x x x x], 0, 0, angle);
assert(isequal(size(rotated), [3, 6]))
\end{lstlisting}


{\footnotesize\textbf{Test 2: rotate vectors in x-axes}}

\begin{lstlisting}[style=Matlab-editor, basicstyle=\ttfamily\scriptsize]
rotated = rotate3DVector([x y z], 0, 0, 0); % 0 degree
assert(isequal(rotated, [-1 0 0; 0 -1 0; 0 0 -1]))

rotated = rotate3DVector([x y z], angle, 0, 0); % 90 degree
assert(isequal(rotated, [-1 0 0; 0 0 1; 0 -1 0]))

rotated = rotate3DVector([x y z], 2 * angle, 0, 0); % 180 degree
assert(isequal(rotated, [-1 0 0; 0 1 0; 0 0 1]))

rotated = rotate3DVector([x y z], 3 * angle, 0, 0); % 270 degree
assert(isequal(rotated, [-1 0 0; 0 0 -1; 0 1 0]))

rotated = rotate3DVector([x y z], 4 * angle, 0, 0); % 360 degree
assert(isequal(rotated, [-1 0 0; 0 -1 0; 0 0 -1]))
\end{lstlisting}


{\footnotesize\textbf{Test 3: rotate vectors in y-axes}}

\begin{lstlisting}[style=Matlab-editor, basicstyle=\ttfamily\scriptsize]
rotated = rotate3DVector([x y z], 0, 0, 0); % 0 degree
assert(isequal(rotated, [-1 0 0; 0 -1 0; 0 0 -1]))

rotated = rotate3DVector([x y z], 0, angle, 0); % 90 degree
assert(isequal(rotated, [0 0 -1; 0 -1 0; 1 0 0]))

rotated = rotate3DVector([x y z], 0, 2 * angle, 0); % 180 degree
assert(isequal(rotated, [1 0 0; 0 -1 0; 0 0 1]))

rotated = rotate3DVector([x y z], 0, 3 * angle, 0); % 270 degree
assert(isequal(rotated, [0 0 1; 0 -1 0; -1 0 0]))

rotated = rotate3DVector([x y z], 0, 4 * angle, 0); % 360 degree
assert(isequal(rotated, [-1 0 0; 0 -1 0; 0 0 -1]))
\end{lstlisting}


{\footnotesize\textbf{Test 4: rotate vectors in z-axes}}

\begin{lstlisting}[style=Matlab-editor, basicstyle=\ttfamily\scriptsize]
rotated = rotate3DVector([x y z], 0, 0, 0); % 0 degree
assert(isequal(rotated, [-1 0 0; 0 -1 0; 0 0 -1]))

rotated = rotate3DVector([x y z], 0, 0, angle); % 90 degree
assert(isequal(rotated, [0 1 0; -1 0 0; 0 0 -1]))

rotated = rotate3DVector([x y z], 0, 0, 2 * angle); % 180 degree
assert(isequal(rotated, [1 0 0; 0 1 0; 0 0 -1]))

rotated = rotate3DVector([x y z], 0, 0, 3 * angle); % 270 degree
assert(isequal(rotated, [0 -1 0; 1 0 0; 0 0 -1]))

rotated = rotate3DVector([x y z], 0, 0, 4 * angle); % 360 degree
assert(isequal(rotated, [-1 0 0; 0 -1 0; 0 0 -1]))
\end{lstlisting}



\end{document}

