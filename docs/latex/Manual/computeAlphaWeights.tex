
% This LaTeX was auto-generated from MATLAB code.
% To make changes, update the MATLAB code and republish this document.

\documentclass{standalone}
\usepackage{graphicx}
\usepackage{listings}
\usepackage{xcolor}
\usepackage{textcomp}
\usepackage[framed, numbered]{matlab-prettifier}

\sloppy
\definecolor{lightgray}{gray}{0.5}
\setlength{\parindent}{0pt}

\begin{document}

    
    \begin{par}
Computes alpha weights from feature space product HT*beta and target vector y as porduct with inverse covariance matrix with additve noise Ky\^{}-1 represented by its cholesky decomposed lower triangle matrix L. Ky\^{}-1 * (y - m(x)).
\end{par} \vspace{1em}


{\textbf{Syntax}}

\begin{lstlisting}[style=Matlab-editor, basicstyle=\ttfamily\scriptsize]
alpha = computeAlphaWeights(L, y, m)
\end{lstlisting}


{\textbf{Description}}

\begin{par}
\textbf{alpha = computeAlphaWeights(L, y, m)} prepare data and forward it to matrix computation.
\end{par} \vspace{1em}


{\textbf{Input Argurments}}

\begin{par}
\textbf{L} lower triangle matrix of cholesky decomposed K matrix.
\end{par} \vspace{1em}
\begin{par}
\textbf{y} regression target vector.
\end{par} \vspace{1em}
\begin{par}
\textbf{m} regression mean vector.
\end{par} \vspace{1em}


{\textbf{Output Argurments}}

\begin{par}
\textbf{alpha} regression weights.
\end{par} \vspace{1em}


{\textbf{Requirements}}

\begin{itemize}
\setlength{\itemsep}{-1ex}
   \item Other m-files required: None
   \item Subfunctions: computeInverseMatrixProduct
   \item MAT-files required: None
\end{itemize}


{\textbf{See Also}}

\begin{itemize}
\setlength{\itemsep}{-1ex}
   \item \begin{verbatim}decomposeChol\end{verbatim}
   \item \begin{verbatim}computeInverseMatrixProduct\end{verbatim}
   \item \begin{verbatim}initKernelParameters\end{verbatim}
\end{itemize}
\begin{par}
Created on November 06. 2019 by Klaus Juenemann. Copyright Klaus Juenemann 2019.
\end{par} \vspace{1em}
\begin{par}

\end{par} \vspace{1em}
\begin{lstlisting}[style=Matlab-editor, basicstyle=\ttfamily\scriptsize]
function alpha = computeAlphaWeights(L, y, m)
    % get residual
    residual = y - m;
    % L and residual is validated in computation below, get weights
    alpha = computeInverseMatrixProduct(L, residual);
end
\end{lstlisting}



\end{document}

