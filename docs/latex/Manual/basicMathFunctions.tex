
% This LaTeX was auto-generated from MATLAB code.
% To make changes, update the MATLAB code and republish this document.

\documentclass{standalone}
\usepackage{graphicx}
\usepackage{listings}
\usepackage{xcolor}
\usepackage{textcomp}
\usepackage[framed, numbered]{matlab-prettifier}

\sloppy
\definecolor{lightgray}{gray}{0.5}
\setlength{\parindent}{0pt}

\begin{document}

    
    
\section*{basicMathFunctions}

\begin{par}
Set of basic algebraic, analytic and trigonometric functions.
\end{par} \vspace{1em}

\subsection*{Contents}

\begin{itemize}
\setlength{\itemsep}{-1ex}
   \item sinoids2angles
   \item angles2sinoids
   \item decomposeChol
   \item frobeniusNorm
   \item computeInverseMatrixProduct
   \item computeTransposeInverseProduct
   \item addNoise2Covariance
   \item computeAlphaWeights
   \item computeStdLogLoss
   \item computeLogLikelihood
   \item estimateBeta
\end{itemize}


\subsection*{sinoids2angles}

\begin{par}
Converts sinoids to angles.
\end{par} \vspace{1em}


\subsection*{angles2sinoids}

\begin{par}
Converts angles to sinoids.
\end{par} \vspace{1em}


\subsection*{decomposeChol}

\begin{par}
Performs a Cholesky Decomposition of a matrix to its lower triangle matrix. Returns logaritmic determinate of the matrix as side product.
\end{par} \vspace{1em}


\subsection*{frobeniusNorm}

\begin{par}
Computes the Frobenius Norm of a matrix.
\end{par} \vspace{1em}


\subsection*{computeInverseMatrixProduct}

\begin{par}
Computes the product of an inverted matrix A and a vector b or a matrix B by the represented lower triangle matrix L of Cholesky decomposed A.
\end{par} \vspace{1em}


\subsection*{computeTransposeInverseProduct}

\begin{par}
Computes the both side porduct of an inverted matrix A with a vector b or matrix B (left product) and the transposed vector b or matrix B (right product).
\end{par} \vspace{1em}


\subsection*{addNoise2Covariance}

\begin{par}
Additive noise for noisy GPR observations. Add noise along covariance matrix diagonal.
\end{par} \vspace{1em}


\subsection*{computeAlphaWeights}

\begin{par}
Computes regression weights by residual of regression targets and regression mean values.
\end{par} \vspace{1em}


\subsection*{computeStdLogLoss}

\begin{par}
Computes standardized logarithmic loss of test data and predicted data.
\end{par} \vspace{1em}


\subsection*{computeLogLikelihood}

\begin{par}
Computes regression evidence as log marginal likelihood.
\end{par} \vspace{1em}


\subsection*{estimateBeta}

\begin{par}
Compute polynom coefficients for mean approximation on training data.
\end{par} \vspace{1em}
\begin{par}
Created on February 14. 2021 by Tobias Wulf. Copyright Tobias Wulf 2021.
\end{par} \vspace{1em}
\begin{par}

\end{par} \vspace{1em}



\end{document}

