
% This LaTeX was auto-generated from MATLAB code.
% To make changes, update the MATLAB code and republish this document.

\documentclass{standalone}
\usepackage{graphicx}
\usepackage{listings}
\usepackage{xcolor}
\usepackage{textcomp}
\usepackage[framed, numbered]{matlab-prettifier}

\sloppy
\definecolor{lightgray}{gray}{0.5}
\setlength{\parindent}{0pt}

\begin{document}

    
    
\section*{initTrainDS}

\begin{par}
Initiates needed data from training dataset to GPR model struct. Builds GPR target vectors depending on which sensor type was used to process the training dataset.
\end{par} \vspace{1em}

\subsection*{Contents}

\begin{itemize}
\setlength{\itemsep}{-1ex}
   \item Syntax
   \item Description
   \item Input Argurments
   \item Output Argurments
   \item Requirements
   \item See Also
\end{itemize}


\subsection*{Syntax}

\begin{lstlisting}[style=Matlab-editor]
Mdl = initTrainDS(Mdl, TrainDS)
\end{lstlisting}


\subsection*{Description}

\begin{par}
\textbf{Mdl = initTrainDS(Mdl, TrainDS)} attaches regression relevant data information to model struct and initiates the training data with references and regression targets.
\end{par} \vspace{1em}


\subsection*{Input Argurments}

\begin{par}
\textbf{Mdl} model struct.
\end{par} \vspace{1em}
\begin{par}
\textbf{TrainDS} training data struct which includes Info and Data struct.
\end{par} \vspace{1em}


\subsection*{Output Argurments}

\begin{par}
\textbf{Mdl} with attached dataset information, raw training data, refernce angles and regression targets for cosine and sine predictions.
\end{par} \vspace{1em}


\subsection*{Requirements}

\begin{itemize}
\setlength{\itemsep}{-1ex}
   \item Other m-files required: None
   \item Subfunctions: None
   \item MAT-files required: Train\_*.mat
\end{itemize}


\subsection*{See Also}

\begin{itemize}
\setlength{\itemsep}{-1ex}
   \item \begin{verbatim}initGPR\end{verbatim}
   \item \begin{verbatim}Training and Test Datasets\end{verbatim}
\end{itemize}
\begin{par}
Created on February 20. 2021 by Tobias Wulf. Copyright Tobias Wulf 2021.
\end{par} \vspace{1em}
\begin{par}

\end{par} \vspace{1em}
\begin{lstlisting}[style=Matlab-editor]
function Mdl = initTrainDS(Mdl, TrainDS)

    % set model parameters from training dataset and training data dependencies
    % N number of angles and refernces in degree
    Mdl.N = TrainDS.Info.UseOptions.nAngles;
    Mdl.Angles = TrainDS.Data.angles';

    % D sensor array square dimension of DxD sensor array
    Mdl.D = TrainDS.Info.SensorArrayOptions.dimension;

    % P number of predictors in sensor array
    Mdl.P = TrainDS.Info.SensorArrayOptions.SensorCount;

    % get sensor type from dataset
    Mdl.Sensor = TrainDS.Info.UseOptions.BaseReference;

    % choose period factor depending on sensor type
    % how many sinoid periods are abstract on a full rotation by 360
    switch Mdl.Sensor
        case 'TDK'
            Mdl.PF = 1;

        case 'KMZ60'
            Mdl.PF = 2;

        otherwise
            error('Unkown Sensor %s.', Mdl.Sensor);
    end

    % get reference angles in degree and transpose to column vector
    % get sinoid target vectors depending period factor,
    % transpose because angles2sinoids works with row vectors
    [Mdl.Ysin, Mdl.Ycos] = angles2sinoids(Mdl.Angles, ...
        false, Mdl.PF);

    % attach training data fro cosine and sine to model
    Mdl.Xcos = TrainDS.Data.Vcos;
    Mdl.Xsin = TrainDS.Data.Vsin;
end
\end{lstlisting}



\end{document}

