
% This LaTeX was auto-generated from MATLAB code.
% To make changes, update the MATLAB code and republish this document.

\documentclass{standalone}
\usepackage{graphicx}
\usepackage{listings}
\usepackage{xcolor}
\usepackage{textcomp}
\usepackage[framed, numbered]{matlab-prettifier}

\sloppy
\definecolor{lightgray}{gray}{0.5}
\setlength{\parindent}{0pt}

\begin{document}

    
    
\section*{predDS}

\begin{par}
Predicts all frames of a test dataset at once.
\end{par} \vspace{1em}

\subsection*{Contents}

\begin{itemize}
\setlength{\itemsep}{-1ex}
   \item Syntax
   \item Description
   \item Input Argurments
   \item Output Argurments
   \item Requirements
   \item See Also
\end{itemize}


\subsection*{Syntax}

\begin{lstlisting}[style=Matlab-editor]
[fang, frad, fcos, fsin, fcov, s, ciang, cirad] = predDS(Mdl, TestDS)
predicts whole dataset at once using predFrame in a loop.
\end{lstlisting}


\subsection*{Description}

\begin{par}
\textbf{[fang, frad, fcos, fsin, fcov, s, ciang, cirad] = predDS(Mdl, TestDS)}
\end{par} \vspace{1em}


\subsection*{Input Argurments}

\begin{par}
\textbf{Mdl} model struct.
\end{par} \vspace{1em}
\begin{par}
\textbf{TestDS} struct of loaded test dataset.
\end{par} \vspace{1em}


\subsection*{Output Argurments}

\begin{par}
\textbf{fang} vector of computed angle by predicted cosine and sine results.
\end{par} \vspace{1em}
\begin{par}
\textbf{frad} vector of computed radius by predicted cosine and sine results.
\end{par} \vspace{1em}
\begin{par}
\textbf{fcos} vector of predictive mean result of cosine regression.
\end{par} \vspace{1em}
\begin{par}
\textbf{fsin} vector of predictive mean result of sine regression.
\end{par} \vspace{1em}
\begin{par}
\textbf{fcov} vector of predictive variance for both predictive means.
\end{par} \vspace{1em}
\begin{par}
\textbf{s} vector of resulting standard deviation by predictive variance and noise level.
\end{par} \vspace{1em}
\begin{par}
\textbf{ciang} vector of confidence interval of computed angle.
\end{par} \vspace{1em}
\begin{par}
\textbf{cirad} vector of confidence interval of computed radius.
\end{par} \vspace{1em}


\subsection*{Requirements}

\begin{itemize}
\setlength{\itemsep}{-1ex}
   \item Other m-files required: None
   \item Subfunctions: predFrame
   \item MAT-files required: None
\end{itemize}


\subsection*{See Also}

\begin{itemize}
\setlength{\itemsep}{-1ex}
   \item \begin{verbatim}predFrame\end{verbatim}
   \item \begin{verbatim}Training and Test Datasets\end{verbatim}
\end{itemize}
\begin{par}
Created on March 03. 2021 by Tobias Wulf. Copyright Tobias Wulf 2021.
\end{par} \vspace{1em}
\begin{par}

\end{par} \vspace{1em}
\begin{lstlisting}[style=Matlab-editor]
function [fang, frad, fcos, fsin, fcov, s, ciang, cirad] = predDS(Mdl, TestDS)

    % get number of angles in dataset
    N = TestDS.Info.UseOptions.nAngles;

    % allocate memory for results
    fang = zeros(N, 1); % angle
    frad = zeros(N, 1); % radius
    fcos = zeros(N, 1); % cosine
    fsin = zeros(N, 1); % sine
    fcov = zeros(N, 1); % predictive covariance over radius
    s = zeros(N, 1); % sigma standard deviation over radius
    ciang = zeros(N, 2); % confidence 95% interval over angles lower and upper
    cirad = zeros(N, 2); % confidence 95% interval over raidus lower and upper

    % predict angle by angle from dataset
    for n = 1:N
        % get cosine and sine at n-th angle
        Xcos = TestDS.Data.Vcos(:,:,n);
        Xsin = TestDS.Data.Vsin(:,:,n);

        % predict frame
        [fang(n), frad(n), fcos(n), fsin(n), ...
         fcov(n), s(n), ciang(n,:), cirad(n,:)] = predFrame(Mdl, Xcos, Xsin);
    end
end
\end{lstlisting}



\end{document}

