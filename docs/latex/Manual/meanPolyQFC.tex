
% This LaTeX was auto-generated from MATLAB code.
% To make changes, update the MATLAB code and republish this document.

\documentclass{standalone}
\usepackage{graphicx}
\usepackage{listings}
\usepackage{xcolor}
\usepackage{textcomp}
\usepackage[framed, numbered]{matlab-prettifier}

\sloppy
\definecolor{lightgray}{gray}{0.5}
\setlength{\parindent}{0pt}

\begin{document}

    
    
\section*{meanPolyQFC}

\begin{par}
Basis or trend function to compute the H matrix as set of h(x) vectors for each predictor to apply a mean feature space as polynom approximated mean with beta coefficients. Compute H matrix to estimate beta.
\end{par} \vspace{1em}

\subsection*{Contents}

\begin{itemize}
\setlength{\itemsep}{-1ex}
   \item Syntax
   \item Description
   \item Input Argurments
   \item Output Argurments
   \item Requirements
   \item See Also
\end{itemize}


\subsection*{Syntax}

\begin{lstlisting}[style=Matlab-editor]
H = meanPolyQFC(X, degree)
\end{lstlisting}


\subsection*{Description}

\begin{par}
\textbf{H = meanPolyQFC(X, degree)} build polynom by passed data. Fires Frobenius Norm on matrix data.
\end{par} \vspace{1em}


\subsection*{Input Argurments}

\begin{par}
\textbf{X} matrix data.
\end{par} \vspace{1em}
\begin{par}
\textbf{degree} polynom degree.
\end{par} \vspace{1em}


\subsection*{Output Argurments}

\begin{par}
\textbf{H} polynom.
\end{par} \vspace{1em}


\subsection*{Requirements}

\begin{itemize}
\setlength{\itemsep}{-1ex}
   \item Other m-files required: None
   \item Subfunctions: frobeniusNorm
   \item MAT-files required: None
\end{itemize}


\subsection*{See Also}

\begin{itemize}
\setlength{\itemsep}{-1ex}
   \item \begin{verbatim}initQFC\end{verbatim}
   \item \begin{verbatim}frobeniusNorm\end{verbatim}
\end{itemize}
\begin{par}
Created on February 15. 2021 by Tobias Wulf. Copyright Tobias Wulf 2021.
\end{par} \vspace{1em}
\begin{par}

\end{par} \vspace{1em}
\begin{lstlisting}[style=Matlab-editor]
function H = meanPolyQFC(X, degree)
    % get number of observations
    [~, ~, N] = size(X);

    % returns only ones if p = 0
    H = ones(degree + 1, N);

    % compute polynom for degrees > 0
    if degree > 0
        for n = 1:N
            H(2,n) = frobeniusNorm(X(:,:,n), false);
        end
    end

    % compute none linear polynoms if degree > 1
    if degree > 1
        for p = 2:degree
            H(p+1,:) = H(2,:).^p;
        end
    end
end
\end{lstlisting}



\end{document}

