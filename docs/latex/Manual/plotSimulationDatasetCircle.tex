
% This LaTeX was auto-generated from MATLAB code.
% To make changes, update the MATLAB code and republish this document.

\documentclass{standalone}
\usepackage{graphicx}
\usepackage{listings}
\usepackage{xcolor}
\usepackage{textcomp}
\usepackage[framed, numbered]{matlab-prettifier}

\sloppy
\definecolor{lightgray}{gray}{0.5}
\setlength{\parindent}{0pt}

\begin{document}

    
    
\section*{plotSimulationDatasetCircle}

\begin{par}
Search for available trainings or test dataset and plot dataset. Follow user input dialog to choose which dataset to plot. Save created plot to file. Filename same as dataset with attached info.
\end{par} \vspace{1em}

\subsection*{Contents}

\begin{itemize}
\setlength{\itemsep}{-1ex}
   \item Syntax
   \item Description
   \item Examples
   \item Input Argurments
   \item Output Argurments
   \item Requirements
   \item See Also
\end{itemize}


\subsection*{Syntax}

\begin{lstlisting}[style=Matlab-editor]
plotSimulationDatasetCircle()
\end{lstlisting}


\subsection*{Description}

\begin{par}
\textbf{plotSimulationDatasetCircle()} plot training or test dataset which are loacated in data/test or data/training. The function lists all datasets and the user must decide during user input dialog which dataset to plot. It loads path from config.mat and scans for file automatically.
\end{par} \vspace{1em}


\subsection*{Examples}

\begin{lstlisting}[style=Matlab-editor]
plotSimulationDatasetCircle()
\end{lstlisting}


\subsection*{Input Argurments}

\begin{par}
\textbf{None}
\end{par} \vspace{1em}


\subsection*{Output Argurments}

\begin{par}
\textbf{None}
\end{par} \vspace{1em}


\subsection*{Requirements}

\begin{itemize}
\setlength{\itemsep}{-1ex}
   \item Other m-files required: None
   \item Subfunctions: None
   \item MAT-files required: config.mat
\end{itemize}


\subsection*{See Also}

\begin{itemize}
\setlength{\itemsep}{-1ex}
   \item \begin{verbatim}generateSimulationDatasets\end{verbatim}
   \item \begin{verbatim}sensorArraySimulation\end{verbatim}
   \item \begin{verbatim}generateConfigMat\end{verbatim}
\end{itemize}
\begin{par}
Created on December 02. 2020 by Tobias Wulf. Copyright Tobias Wulf 2020.
\end{par} \vspace{1em}
\begin{par}

\end{par} \vspace{1em}
\begin{lstlisting}[style=Matlab-editor]
function plotSimulationDatasetCircle()
    % scan for datasets and load needed configurations %%%%%%%%%%%%%%%%%%%%%%%%%
    %%%%%%%%%%%%%%%%%%%%%%%%%%%%%%%%%%%%%%%%%%%%%%%%%%%%%%%%%%%%%%%%%%%%%%%%%%%%
    try
        disp('Plot simulation dataset ...');
        close all;
        % load path variables
        load('config.mat', 'PathVariables');
        % scan for datasets
        TrainingDatasets = dir(fullfile(PathVariables.trainingDataPath, ...
            'Training_*.mat'));
        TestDatasets = dir(fullfile(PathVariables.testDataPath, 'Test_*.mat'));
        allDatasets = [TrainingDatasets; TestDatasets];
        % check if files available
        if isempty(allDatasets)
            error('No training or test datasets found.');
        end
    catch ME
        rethrow(ME)
    end

    % display availabe datasets to user, decide which to plot %%%%%%%%%%%%%%%%%%
    %%%%%%%%%%%%%%%%%%%%%%%%%%%%%%%%%%%%%%%%%%%%%%%%%%%%%%%%%%%%%%%%%%%%%%%%%%%%

    % number of datasets
    nDatasets = length(allDatasets);
    fprintf('Found %d datasets:\n', nDatasets)
    for i = 1:nDatasets
        fprintf('%s\t:\t(%d)\n', allDatasets(i).name, i)
    end
    % get numeric user input to indicate which dataset to plot
    iDataset = input('Type number to choose dataset to plot to: ');

    % load dataset and ask user which one and how many angles %%%%%%%%%%%%%%%%%%
    %%%%%%%%%%%%%%%%%%%%%%%%%%%%%%%%%%%%%%%%%%%%%%%%%%%%%%%%%%%%%%%%%%%%%%%%%%%%
    try
        ds = load(fullfile(allDatasets(iDataset).folder, ...
            allDatasets(iDataset).name));
    catch ME
        rethrow(ME)
    end

    % figure save path for different formats %%%%%%%%%%%%%%%%%%%%%%%%%%%%%%%%%%%
    %%%%%%%%%%%%%%%%%%%%%%%%%%%%%%%%%%%%%%%%%%%%%%%%%%%%%%%%%%%%%%%%%%%%%%%%%%%%
    fPath = PathVariables.saveImagesPath;

    % create dataset figure for a subset or all angle %%%%%%%%%%%%%%%%%%%%%%%%%%
    %%%%%%%%%%%%%%%%%%%%%%%%%%%%%%%%%%%%%%%%%%%%%%%%%%%%%%%%%%%%%%%%%%%%%%%%%%%%
    fig = figure('Name', 'Sensor Array', ...
        'NumberTitle' , 'off', ...
        'WindowStyle', 'normal', ...
        'MenuBar', 'none', ...
        'ToolBar', 'none', ...
        'Units', 'centimeters', ...
        'OuterPosition', [0 0 30 30], ...
        'PaperType', 'a4', ...
        'PaperUnits', 'centimeters', ...
        'PaperOrientation', 'landscape', ...
        'PaperPositionMode', 'auto', ...
        'DoubleBuffer', 'on', ...
        'RendererMode', 'manual', ...
        'Renderer', 'painters');

    tdl = tiledlayout(fig, 2, 2, ...
        'Padding', 'compact', ...
        'TileSpacing' , 'compact');

    title(tdl, 'Sensor Array Simulation', ...
        'FontWeight', 'normal', ...
        'FontSize', 18, ...
        'FontName', 'Times', ...
        'Interpreter', 'latex');

    subline1 = "Sensor Array (%s) of $%d\\times%d$ sensors," + ...
        " an edge length of $%.1f$ mm, a rel. pos. to magnet surface of";
    subline2 = " $(%.1f, %.1f, -(%.1f))$ in mm, a magnet tilt" + ...
        " of $%.1f^\\circ$, a sphere radius of $%.1f$ mm, a imprinted";
    subline3 = "field strength of $%.1f$ kA/m at $%.1f$ mm from" + ...
        " sphere surface in z-axis, $%d$ rotation angles with a ";
    subline4 = "step width of $%.1f^\\circ$ and a resolution of" + ...
        " $%.1f^\\circ$. Visualized are circular path of each array position  ";
    subline5 = "Based on %s characterization reference %s.";
    sub = [sprintf(subline1, ...
                   ds.Info.SensorArrayOptions.geometry, ...
                   ds.Info.SensorArrayOptions.dimension, ...
                   ds.Info.SensorArrayOptions.dimension, ...
                   ds.Info.SensorArrayOptions.edge); ...
           sprintf(subline2, ...
                   ds.Info.UseOptions.xPos, ...
                   ds.Info.UseOptions.yPos, ...
                   ds.Info.UseOptions.zPos, ...
                   ds.Info.UseOptions.tilt, ...
                   ds.Info.DipoleOptions.sphereRadius); ...
           sprintf(subline3, ...
                   ds.Info.DipoleOptions.H0mag, ...
                   ds.Info.DipoleOptions.z0, ...
                   ds.Info.UseOptions.nAngles); ...
           sprintf(subline4, ...
                   ds.Data.angleStep, ...
                   ds.Info.UseOptions.angleRes)
           sprintf(subline5, ...
                   ds.Info.CharData, ...
                   ds.Info.UseOptions.BridgeReference)];

    subtitle(tdl, sub, ...
        'FontWeight', 'normal', ...
        'FontSize', 14, ...
        'FontName', 'Times', ...
        'Interpreter', 'latex');

    % get subset of needed data to plot, only one load %%%%%%%%%%%%%%%%%%%%%%%%%
    %%%%%%%%%%%%%%%%%%%%%%%%%%%%%%%%%%%%%%%%%%%%%%%%%%%%%%%%%%%%%%%%%%%%%%%%%%%%
    N = ds.Info.SensorArrayOptions.dimension;
    M = ds.Info.UseOptions.nAngles;
    Voff = ds.Info.SensorArrayOptions.Voff;
    Vcos = ds.Data.Vcos - Voff;
    Vsin = ds.Data.Vsin - Voff;
    Hx = ds.Data.Hx;
    Hy = ds.Data.Hy;

    % calulate norm values to align circles around position only for x,y
    % directition for each sensor dot over all angles.
    Vmag = sqrt(Vcos.^2 + Vsin.^2);
    Hmag = sqrt(Hx.^2 + Hy.^2);
    %Hmag = ds.Data.Habs;

    % related to position, multiply scale factor for circle diameter
    diameterFactor = 2 * N / ds.Info.SensorArrayOptions.edge;
    MaxVmagPos = max(Vmag, [], 3) * diameterFactor;
    MaxHmagPos = max(Hmag, [], 3) * diameterFactor;

    % Overall maxima, scalar, multiply scale factor for circle diameter
    MaxVmagOA =  max(Vmag, [], 'all') * diameterFactor;
    MaxHmagOA =  max(Hmag, [], 'all') * diameterFactor;

    % norm and scale volatages and filed strengths
    VcosNorm = Vcos ./ MaxVmagPos;
    VcosScaled = Vcos / MaxVmagOA;
    VsinNorm = Vsin ./ MaxVmagPos;
    VsinScaled = Vsin / MaxVmagOA;

    HxNorm = Hx ./ MaxHmagPos;
    HxScaled = Hx / MaxHmagOA;
    HyNorm = Hy ./ MaxHmagPos;
    HyScaled = Hy / MaxHmagOA;

    % sensor array grid
    X = ds.Data.X;
    Y = ds.Data.Y;
    Z = ds.Data.Z;

    % calc limits of plot 1
    maxX = ds.Info.UseOptions.xPos + 0.7 * ds.Info.SensorArrayOptions.edge;
    maxY = ds.Info.UseOptions.yPos + 0.7 * ds.Info.SensorArrayOptions.edge;
    minX = ds.Info.UseOptions.xPos - 0.7 * ds.Info.SensorArrayOptions.edge;
    minY = ds.Info.UseOptions.yPos - 0.7 * ds.Info.SensorArrayOptions.edge;

    % plot sensor grid in x and y coordinates %%%%%%%%%%%%%%%%%%%%%%%%%%%%%%%%%%
    %%%%%%%%%%%%%%%%%%%%%%%%%%%%%%%%%%%%%%%%%%%%%%%%%%%%%%%%%%%%%%%%%%%%%%%%%%%%
    % plot each cooredinate in loop to create a special shading constant
    % reliable to orientation for all matrice
     % calculate colormap to identify scatter points
    c=zeros(N,N,3);
    for i = 1:N
        for j = 1:N
            c(i,j,:) = [(2*N+1-2*i), (2*N+1-2*j), (i+j)]/2/N;
        end
    end
    c = squeeze(reshape(c, N^2, 1, 3));
    % reshape RGB for picking single sensors
    R = reshape(c(:,1), N, N);
    G = reshape(c(:,2), N, N);
    B = reshape(c(:,3), N, N);

    % Field strength scaled to overall maxima %%%%%%%%%%%%%%%%%%%%%%%%%%%%%%%%%%
    %%%%%%%%%%%%%%%%%%%%%%%%%%%%%%%%%%%%%%%%%%%%%%%%%%%%%%%%%%%%%%%%%%%%%%%%%%%%
    nexttile;
    hold on;
    for i = 1:N
        for j = 1:N
            plot(squeeze(HxScaled(i, j, :)) + X(i,j), ...
                 squeeze(HyScaled(i, j, :)) + Y(i,j), ...
                 'Color', [R(i,j) G(i,j) B(i,j)], ...
                 'LineWidth' , 1.5)
            line([X(i,j), HxScaled(i,j,1) + X(i,j)], ...
                 [Y(i,j), HyScaled(i,j,1)  + Y(i,j)], ...
                'Color','k','LineWidth',1.5)
        end
    end

    % scatter magnet x,y position (0,0,z)
    scatter(0, 0, 32, 'r', 'filled');

    hold off;

    % axis shape and ticks
    axis square xy;
    axis tight;
    grid on;
    xlim([minX maxX]);
    ylim([minY maxY]);

    xlabel('$X$ in mm', ...
        'FontWeight', 'normal', ...
        'FontSize', 12, ...
        'FontName', 'Times', ...
        'Interpreter', 'latex');

    ylabel('$Y$ in mm', ...
        'FontWeight', 'normal', ...
        'FontSize', 12, ...
        'FontName', 'Times', ...
        'Interpreter', 'latex');

    title('$H_x$, $H_y$ Normed to Max overall Positions', ...
        'FontWeight', 'normal', ...
        'FontSize', 12, ...
        'FontName', 'Times', ...
        'Interpreter', 'latex');

    % Cosinus, sinus voltage scaled to overall maxima %%%%%%%%%%%%%%%%%%%%%%%%%%
    %%%%%%%%%%%%%%%%%%%%%%%%%%%%%%%%%%%%%%%%%%%%%%%%%%%%%%%%%%%%%%%%%%%%%%%%%%%%
    nexttile;
    hold on;
    for i = 1:N
        for j = 1:N
            plot(squeeze(VcosScaled(i, j, :)) + X(i,j), ...
                 squeeze(VsinScaled(i, j, :)) + Y(i,j), ...
                 'Color', [R(i,j) G(i,j) B(i,j)], ...
                 'LineWidth' , 1.5)
            line([X(i,j), VcosScaled(i,j,1) + X(i,j)], ...
                 [Y(i,j), VsinScaled(i,j,1)  + Y(i,j)], ...
                'Color','k','LineWidth',1.5)
        end
    end

    % scatter magnet x,y position (0,0,z)
    scatter(0, 0, 32, 'r', 'filled');

    hold off;

    % axis shape and ticks
    axis square xy;
    axis tight;
    grid on;
    xlim([minX maxX]);
    ylim([minY maxY]);

    xlabel('$X$ in mm', ...
        'FontWeight', 'normal', ...
        'FontSize', 12, ...
        'FontName', 'Times', ...
        'Interpreter', 'latex');

    ylabel('$Y$ in mm', ...
        'FontWeight', 'normal', ...
        'FontSize', 12, ...
        'FontName', 'Times', ...
        'Interpreter', 'latex');

    title('$V_{cos}$, $V_{sin}$ Normed to Max overall Positions', ...
        'FontWeight', 'normal', ...
        'FontSize', 12, ...
        'FontName', 'Times', ...
        'Interpreter', 'latex');

    % Field strength normed each maxima at position %%%%%%%%%%%%%%%%%%%%%%%%%%%%
    %%%%%%%%%%%%%%%%%%%%%%%%%%%%%%%%%%%%%%%%%%%%%%%%%%%%%%%%%%%%%%%%%%%%%%%%%%%%
    nexttile;
    hold on;
    for i = 1:N
        for j = 1:N
            plot(squeeze(HxNorm(i, j, :)) + X(i,j), ...
                 squeeze(HyNorm(i, j, :)) + Y(i,j), ...
                 'Color', [R(i,j) G(i,j) B(i,j)], ...
                 'LineWidth' , 1.5)
            line([X(i,j), HxNorm(i,j,1) + X(i,j)], ...
                 [Y(i,j), HyNorm(i,j,1)  + Y(i,j)], ...
                'Color','k','LineWidth',1.5)
        end
    end

    % scatter magnet x,y position (0,0,z)
    scatter(0, 0, 32, 'r', 'filled');

    hold off;

    % axis shape and ticks
    axis square xy;
    axis tight;
    grid on;
    xlim([minX maxX]);
    ylim([minY maxY]);

    xlabel('$X$ in mm', ...
        'FontWeight', 'normal', ...
        'FontSize', 12, ...
        'FontName', 'Times', ...
        'Interpreter', 'latex');

    ylabel('$Y$ in mm', ...
        'FontWeight', 'normal', ...
        'FontSize', 12, ...
        'FontName', 'Times', ...
        'Interpreter', 'latex');

    title('$H_x$, $H_y$ Normed to Max at each Position', ...
        'FontWeight', 'normal', ...
        'FontSize', 12, ...
        'FontName', 'Times', ...
        'Interpreter', 'latex');

    % Cosinus, sinus voltage normed to each maxima at position %%%%%%%%%%%%%%%%%
    %%%%%%%%%%%%%%%%%%%%%%%%%%%%%%%%%%%%%%%%%%%%%%%%%%%%%%%%%%%%%%%%%%%%%%%%%%%%
    nexttile;
    hold on;
    for i = 1:N
        for j = 1:N
            plot(squeeze(VcosNorm(i, j, :)) + X(i,j), ...
                 squeeze(VsinNorm(i, j, :)) + Y(i,j), ...
                 'Color', [R(i,j) G(i,j) B(i,j)], ...
                 'LineWidth' , 1.5)
            line([X(i,j), VcosNorm(i,j,1) + X(i,j)], ...
                 [Y(i,j), VsinNorm(i,j,1)  + Y(i,j)], ...
                'Color','k','LineWidth',1.5)
        end
    end

    % scatter magnet x,y position (0,0,z)
    scatter(0, 0, 32, 'r', 'filled');

    hold off;

    % axis shape and ticks
    axis square xy;
    axis tight;
    grid on;
    xlim([minX maxX]);
    ylim([minY maxY]);

    xlabel('$X$ in mm', ...
        'FontWeight', 'normal', ...
        'FontSize', 12, ...
        'FontName', 'Times', ...
        'Interpreter', 'latex');

    ylabel('$Y$ in mm', ...
        'FontWeight', 'normal', ...
        'FontSize', 12, ...
        'FontName', 'Times', ...
        'Interpreter', 'latex');

    title('$V_{cos}$, $V_{sin}$ Normed to Max at each Positions', ...
        'FontWeight', 'normal', ...
        'FontSize', 12, ...
        'FontName', 'Times', ...
        'Interpreter', 'latex');

    % save figure to file %%%%%%%%%%%%%%%%%%%%%%%%%%%%%%%%%%%%%%%%%%%%%%%%%%%%%%
    %%%%%%%%%%%%%%%%%%%%%%%%%%%%%%%%%%%%%%%%%%%%%%%%%%%%%%%%%%%%%%%%%%%%%%%%%%%%
    % get file path to save figure with angle index
    [~, fName, ~] = fileparts(ds.Info.filePath);

    % save to various formats
    yesno = input('Save? [y/n]: ', 's');
    if strcmp(yesno, 'y')
        fLabel = input('Enter file label: ', 's');
        fName = fName + "_CirclePlot_" + fLabel;
        savefig(fig, fullfile(fPath, fName));
        print(fig, fullfile(fPath, fName), '-dsvg');
        print(fig, fullfile(fPath, fName), '-depsc', '-tiff', '-loose');
        print(fig, fullfile(fPath, fName), '-dpdf', '-loose', '-fillpage');
    end
    close(fig);
end
\end{lstlisting}



\end{document}

