
% This LaTeX was auto-generated from MATLAB code.
% To make changes, update the MATLAB code and republish this document.

\documentclass{standalone}
\usepackage{graphicx}
\usepackage{listings}
\usepackage{xcolor}
\usepackage{textcomp}
\usepackage[framed, numbered]{matlab-prettifier}

\sloppy
\definecolor{lightgray}{gray}{0.5}
\setlength{\parindent}{0pt}

\begin{document}

    
    
\section*{lossDS}

\begin{par}
Predicts all angles of passed test dataset and computes logaritmic losses for radius and angles plus several squared errors.
\end{par} \vspace{1em}

\subsection*{Contents}

\begin{itemize}
\setlength{\itemsep}{-1ex}
   \item Syntax
   \item Description
   \item Examples
   \item Input Argurments
   \item Output Argurments
   \item Requirements
   \item See Also
\end{itemize}


\subsection*{Syntax}

\begin{lstlisting}[style=Matlab-editor]
[AAED, SLLA, SLLR, SEA, SER, SEC, SES] = lossDS(Mdl, TestDS)
\end{lstlisting}


\subsection*{Description}

\begin{par}
\textbf{[AAED, SLLA, SLLR, SEA, SER, SEC, SES] = lossDS(Mdl, TestDS)} computes losses and prediction erros of a whole datasets
\end{par} \vspace{1em}


\subsection*{Examples}

\begin{lstlisting}[style=Matlab-editor]
Enter example matlab code for each use case.
\end{lstlisting}


\subsection*{Input Argurments}

\begin{par}
\textbf{positionalArg} argurment description.
\end{par} \vspace{1em}
\begin{par}
\textbf{optionalArg} argurment description.
\end{par} \vspace{1em}


\subsection*{Output Argurments}

\begin{par}
\textbf{AAED} Absolute Angular Error in Degrees \textbf{SLLA} Std. Log. Loss Angular \textbf{SLLR} Std. Log Loss Radius \textbf{SEA} Squared Error Angular \textbf{SER} Squared Error Radius \textbf{SEC} Squared Error Cosine \textbf{SES} Squared Error Sine
\end{par} \vspace{1em}


\subsection*{Requirements}

\begin{itemize}
\setlength{\itemsep}{-1ex}
   \item Other m-files required: None
   \item Subfunctions: angles2sinoids, computeStdLogLoss
   \item MAT-files required: None
\end{itemize}


\subsection*{See Also}

\begin{itemize}
\setlength{\itemsep}{-1ex}
   \item \begin{verbatim}predDS\end{verbatim}
   \item \begin{verbatim}Training and Test Datasets\end{verbatim}
   \item \begin{verbatim}angles2sinoids\end{verbatim}
   \item \begin{verbatim}computeStdLogLoss\end{verbatim}
\end{itemize}
\begin{par}
Created on March 03. 2021 by Tobias Wulf. Copyright Tobias Wulf 2021.
\end{par} \vspace{1em}
\begin{par}

\end{par} \vspace{1em}
\begin{lstlisting}[style=Matlab-editor]
function [AAED, SLLA, SLLR, SEA, SER, SEC, SES] = lossDS(Mdl, TestDS)

    % get number of angles in dataset
    N = TestDS.Info.UseOptions.nAngles;

    % get simulated cosin and sine references from dataset angles in degrees
    % and transpose to column vector, get sinoids and angles in rads
    [ysin, ycos, yang] = angles2sinoids(TestDS.Data.angles', false, Mdl.PF);

    % create reference radius of unit cricle, radius must be one for all angles
    yrad = ones(N, 1);

    % predict angles in rads not in degrees
    [fang, frad, fcos, fsin, ~, s, ~, ~] = predDS(Mdl, TestDS);

    % compute log loss and squared error for angles in rad
    [SLLA, SEA] = computeStdLogLoss(yang, fang, asin(s) * sqrt(2));

    % compute abslute angular error in degrees
    AAED = sqrt(SEA) * 180/pi;

    % compute log loss and squared error for radius
    [SLLR, SER] = computeStdLogLoss(yrad, frad, sqrt(2) * s);

    % compute squared error of sinoids
    SEC = (ycos - fcos).^2;
    SES = (ysin - fsin).^2;

end
\end{lstlisting}



\end{document}

