
% This LaTeX was auto-generated from MATLAB code.
% To make changes, update the MATLAB code and republish this document.

\documentclass{standalone}
\usepackage{graphicx}
\usepackage{listings}
\usepackage{xcolor}
\usepackage{textcomp}
\usepackage[framed, numbered]{matlab-prettifier}

\sloppy
\definecolor{lightgray}{gray}{0.5}
\setlength{\parindent}{0pt}

\begin{document}

    
    \begin{par}
Approximates QFC with triangle inequation, norming is pulled out to input stage kernel is feeded with norm vectors or scalars instead of matrices.
\end{par} \vspace{1em}


{\footnotesize\textbf{Syntax}}

\begin{lstlisting}[style=Matlab-editor, basicstyle=\ttfamily\scriptsize]
K = QFCAPX(ax, bx, ay, by, theta)
\end{lstlisting}


{\footnotesize\textbf{Description}}

\begin{par}
\textbf{K = QFCAPX(ax, bx, ay, by, theta)} computes quadratic distances bewtween data points and parametrize it with height and length scales. Computes distance with quadratic euclidian norm.
\end{par} \vspace{1em}


{\footnotesize\textbf{Input Argurments}}

\begin{par}
\textbf{ax} vector of cosine simulation components.
\end{par} \vspace{1em}
\begin{par}
\textbf{bx} vector of cosine simulation components.
\end{par} \vspace{1em}
\begin{par}
\textbf{ay} vector of sine simulation components.
\end{par} \vspace{1em}
\begin{par}
\textbf{by} vector of sine simulation components.
\end{par} \vspace{1em}
\begin{par}
\textbf{theta} vector of kernel parameters.
\end{par} \vspace{1em}


{\footnotesize\textbf{Output Argurments}}

\begin{par}
\textbf{K} noise free covarianc matrix.
\end{par} \vspace{1em}


{\footnotesize\textbf{Requirements}}

\begin{itemize}
\setlength{\itemsep}{-1ex}
   \item Other m-files required: None
   \item Subfunctions: None
   \item MAT-files required: None
\end{itemize}


{\footnotesize\textbf{See Also}}

\begin{itemize}
\setlength{\itemsep}{-1ex}
   \item \begin{verbatim}initQFCAPX\end{verbatim}
   \item \begin{verbatim}meanPolyQFCAPX\end{verbatim}
\end{itemize}
\begin{par}
Created on February 15. 2021 by Tobias Wulf. Copyright Tobias Wulf 2021.
\end{par} \vspace{1em}
\begin{par}

\end{par} \vspace{1em}
\begin{lstlisting}[style=Matlab-editor, basicstyle=\ttfamily\scriptsize]
function K = QFCAPX(ax, bx, ay, by, theta)
     arguments
        % validate data as real vector of same size
        ax (:,:) double {mustBeReal}
        bx (:,:) double {mustBeReal, mustBeFitSize(ax,bx)}
        ay (:,:) double {mustBeReal, mustBeFitSize(ax,ay)}
        by (:,:) double {mustBeReal, mustBeFitSize(ax,by)}
        % validate kernel parameters as 1x2 vector
        theta (1,2) double {mustBeReal}
    end

    % get number of observations for each dataset, cosine and sine
    M = length(ax);
    N = length(bx);

    % expand covariance parameters, variance and lengthscale
    c2 = 2 * theta(2)^2; % 2*sl^2
    c1 = theta(1) * c2;   % s2f * c

    % allocate memory for K
    K = zeros(M, N);

    % loop through observation points and compute the covariance for each
    % observation against another
    for m = 1:M
        for n = 1:N
            % get distance between m-th and n-th observation
            % compute distance with quadratic frobenius normed vectors
            r2 = (ax(m) - bx(n))^2 + (ay(m) - by(n))^2;

            % engage lengthscale and variance on distance
            K(m,n) = c1 / (c2 + r2);

        end
    end
end

function mustBeFitSize(a, b)
    % Test for equal size
    if ~isequal(size(a,2), size(b,2))
        eid = 'Size:notEqual';
        msg = 'Sizes of  are not fitting.';
        throwAsCaller(MException(eid,msg))
    end
end
\end{lstlisting}



\end{document}

