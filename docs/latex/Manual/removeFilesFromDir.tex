
% This LaTeX was auto-generated from MATLAB code.
% To make changes, update the MATLAB code and republish this document.

\documentclass{standalone}
\usepackage{graphicx}
\usepackage{listings}
\usepackage{xcolor}
\usepackage{textcomp}
\usepackage[framed, numbered]{matlab-prettifier}

\sloppy
\definecolor{lightgray}{gray}{0.5}
\setlength{\parindent}{0pt}

\begin{document}

    
    \begin{par}
Remove files from passed direcotory.
\end{par} \vspace{1em}


{\textbf{Syntax}}

\begin{lstlisting}[style=Matlab-editor, basicstyle=\ttfamily\scriptsize]
removeStatus = removeFilesFromDir(directory)
removeStatus = removeFilesFromDir(directory, filePattern)
\end{lstlisting}


{\textbf{Description}}

\begin{par}
\textbf{removeStatus = removeFilesFromDir(directory)} removes all files that are located in the passed directory and returns a logical 1 if the operation was successful or 0 if not. The directory argument must be char vector of 1xN and valid path to a existing directory.
\end{par} \vspace{1em}
\begin{par}
\textbf{removeStatus = removeFilesFromDir(directory, filePattern)} removes all files in the located directory which matches the passed file pattern. The filePattern argument must be be char vector of 1xN. It is an optional argument with a default value of '*.*', valid file patterns can be filenames which part replace names by * character before the dot and exisiting file extensions e.g. myfile\_*.m or *.txt and so on.
\end{par} \vspace{1em}


{\textbf{Examples}}

\begin{lstlisting}[style=Matlab-editor, basicstyle=\ttfamily\scriptsize]
d = fullfile('rootPath', 'subfolder')
rs = removeFileFromDir(d)
\end{lstlisting}
\begin{lstlisting}[style=Matlab-editor, basicstyle=\ttfamily\scriptsize]
d = fullfile('rootPath', 'subfolder')
rs = removeFileFromDir(d, '*.mat')
\end{lstlisting}


{\textbf{Input Arguments}}

\begin{par}
\textbf{directory} char vector, path directory in which to scan for files with file pattern and to delete found files.
\end{par} \vspace{1em}
\begin{par}
\textbf{filePattern} char vector of file pattern with extension. Default is to delete all files. Possible patterns can be passed with filename parts with start operator as place holder.
\end{par} \vspace{1em}


{\textbf{Output Arguments}}

\begin{par}
\textbf{removeStatus} locgical scalar which is true if all files wich matches the file pattern are deleted successfully from passed directory path.
\end{par} \vspace{1em}


{\textbf{Requirements}}

\begin{itemize}
\setlength{\itemsep}{-1ex}
   \item Other m-files required: None
   \item Subfunctions: None
   \item MAT-files required: None
\end{itemize}


{\textbf{See Also}}

\begin{itemize}
\setlength{\itemsep}{-1ex}
   \item \begin{verbatim}fullfile\end{verbatim}
   \item \begin{verbatim}dir\end{verbatim}
   \item \begin{verbatim}delete\end{verbatim}
   \item \begin{verbatim}isfile\end{verbatim}
   \item \begin{verbatim}isempty\end{verbatim}
   \item \begin{verbatim}ismember\end{verbatim}
   \item \begin{verbatim}mustBeFolder\end{verbatim}
   \item \begin{verbatim}mustBeText\end{verbatim}
\end{itemize}
\begin{par}
Created on October 10. 2020 by Tobias Wulf. Copyright Tobias Wulf 2020.
\end{par} \vspace{1em}
\begin{par}

\end{par} \vspace{1em}
\begin{lstlisting}[style=Matlab-editor, basicstyle=\ttfamily\scriptsize]
function [removeStatus] = removeFilesFromDir(directory, filePattern)
    arguments
        % validate directory
        directory (1,:) char {mustBeFolder}
        % validate filePattern
        filePattern (1,:) char {mustBeText} = '*.*'
    end
    % parse pattern for dir
    parsePattern = fullfile(directory, filePattern);
    % parse directory, returns struct
    filesToRemove = dir(parsePattern);
    % delete files, tranpose to loop through struct
    for file = filesToRemove'
        % check before delete
        filePath = fullfile(file.folder, file.name);
        if isfile(filePath)
            delete(filePath);
        end
    end
    % check if dir returns an empty struct now
    check = dir(parsePattern);
    removeStatus = isempty(check(~ismember({check.name}, {'.', '..'})));
end
\end{lstlisting}



\end{document}

