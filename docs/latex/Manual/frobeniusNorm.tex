
% This LaTeX was auto-generated from MATLAB code.
% To make changes, update the MATLAB code and republish this document.

\documentclass{standalone}
\usepackage{graphicx}
\usepackage{listings}
\usepackage{xcolor}
\usepackage{textcomp}
\usepackage[framed, numbered]{matlab-prettifier}

\sloppy
\definecolor{lightgray}{gray}{0.5}
\setlength{\parindent}{0pt}

\begin{document}

    
    \begin{par}
Computes the Frobenius Norm of a matrix A.
\end{par} \vspace{1em}


{\footnotesize\textbf{Syntax}}

\begin{lstlisting}[style=Matlab-editor, basicstyle=\ttfamily\scriptsize]
nv = frobeniusNorm(A, approx)
\end{lstlisting}


{\footnotesize\textbf{Description}}

\begin{par}
\textbf{frobeniusNorm(A, approx)} computes Frobenius Norm of M x N matrix. If approx is true the Norm is approximated with mean2 function.
\end{par} \vspace{1em}


{\footnotesize\textbf{Examples}}

\begin{lstlisting}[style=Matlab-editor, basicstyle=\ttfamily\scriptsize]
A = magic(8);
nv = frobeniusNorm(A, approx)
\end{lstlisting}


{\footnotesize\textbf{Input Argurments}}

\begin{par}
\textbf{A} is a M x N matrix of real values.
\end{par} \vspace{1em}
\begin{par}
\textbf{apporx} is boolean flag. If true the norm is approximated. Default is false.
\end{par} \vspace{1em}


{\footnotesize\textbf{Output Argurments}}

\begin{par}
\textbf{nv} is a scalar norm value.
\end{par} \vspace{1em}


{\footnotesize\textbf{Requirements}}

\begin{itemize}
\setlength{\itemsep}{-1ex}
   \item Other m-files required: None
   \item Subfunctions: mean2, sqrt, sum
   \item MAT-files required: None
\end{itemize}


{\footnotesize\textbf{See Also}}

\begin{itemize}
\setlength{\itemsep}{-1ex}
   \item \begin{verbatim}QFCAPX\end{verbatim}
   \item \begin{verbatim}meanPolyQFC\end{verbatim}
\end{itemize}
\begin{par}
Created on January 05. 2021 by Tobias Wulf. Copyright Tobias Wulf 2021.
\end{par} \vspace{1em}
\begin{par}

\end{par} \vspace{1em}
\begin{lstlisting}[style=Matlab-editor, basicstyle=\ttfamily\scriptsize]
function nv = frobeniusNorm(A, approx)
    arguments
        % validate A as real matrix
        A (:,:) double {mustBeReal}
        % validate approx as flag with default false
        approx (1,1) logical {mustBeNumericOrLogical} = false
    end

    % norm matrix
    if approx
        % approximate frobenis with mean and multiply with radicant of RMS
        % frobenius norm is a RMS * sqrt(N x N), RMS >= mean
        nv = mean2(A) * sqrt(numel(A));
    else
        % norm with frobenius
        nv = sqrt(sum(A.^2, 'all'));
    end
end
\end{lstlisting}



\end{document}

