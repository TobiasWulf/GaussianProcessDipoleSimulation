
% This LaTeX was auto-generated from MATLAB code.
% To make changes, update the MATLAB code and republish this document.

\documentclass{standalone}
\usepackage{graphicx}
\usepackage{listings}
\usepackage{xcolor}
\usepackage{textcomp}
\usepackage[framed, numbered]{matlab-prettifier}

\sloppy
\definecolor{lightgray}{gray}{0.5}
\setlength{\parindent}{0pt}

\begin{document}

    
    

\section*{GaussianProcessDipoleSimulation}

\begin{par}
The project of sensor array simulations and Gaussian Processes for angle predictions on simulation datasets started in
\end{par} \vspace{1em}
\begin{par}
\textbf{May 06. 2019}
\end{par} \vspace{1em}
\begin{par}
with IEEE paper by Thorben Schüthe which is a base investigation of "Two-Dimensional Characterization and Simplified Simualtion Procedure for Tunnel Magnetorersistive Angle Sensors". This produces characterization datasets of different current available angular sensors on the market.
\end{par} \vspace{1em}
\begin{par}
\textbf{June 11. 2019}
\end{par} \vspace{1em}
\begin{par}
Thorben Schüthe came up with a high experimental scripting for abstracting sensor characterization fields to an array of sensor fields which was stimulated by magnetic dipole field equautions to approximate a spherical magnet.
\end{par} \vspace{1em}
\begin{par}
\textbf{November 06. 2019}
\end{par} \vspace{1em}
\begin{par}
Prof. Dr. Klaus Jünemann supports the team around Prof. Dr.-Ing. Karl-Ragmar Riemschneider and Thorben Schüthe with an apply of Gaussian Process learning to investigate on angle predictions for sensor array simualtion results. The attempt of the solution was working for tight set of parameter and was highly experimental with rare documentation and few set of functions and scripts. The math of this very solution based on the standard book for Gaussian Process by Williams and Rasmussen. The algorithm is related to the guidline for linear regression model which worked fine for a setup of standard use cases but needed further investigation for a wider set of parameters and functions to identify general and relevant parameter settings to provide an applicable angular prediction.
\end{par} \vspace{1em}
\begin{par}
\textbf{September 21. 2020}
\end{par} \vspace{1em}
\begin{par}
Tobias Wulf establish a Matlab project structure and programming guidance and flows to document the source code integrated in the Matlab project architecture. That includes templating for scripts and functions and general descriptions of project structure and guidance for testing and documenting project results or new source code including automation for publishing html in Matlab integrated fashion.
\end{par} \vspace{1em}
\begin{par}
\textbf{October 22. 2020}
\end{par} \vspace{1em}
\begin{par}
Tobias Wulf added TDK TAS2141 TMR characterization to the project. Thorben Schüthe provided a raw dataset which was manually modified by Tobias Wulf to dataset which is plotable and reconstructable in stimulus and characterization field investigations.
\end{par} \vspace{1em}
\begin{par}
\textbf{October 31. 2020}
\end{par} \vspace{1em}
\begin{par}
Tobias Wulf establish a general configuration flow to control part of software via config file which is partly loaded as needed into workspace.
\end{par} \vspace{1em}
\begin{par}
\textbf{November 29. 2020}
\end{par} \vspace{1em}
\begin{par}
Tobias Wulf finished the implementation of sensor array simulation which uses TDK TAS2141 as base of simualtion. The software includes now simulation for situmulus magnet (dipole sphere) and automated way fast generate training and test datasets by set configuration. Various plots and animation for datasets and a best practice workflow for simulation. Also included are unittest and Matlab integrated documentation in html files. A full description of generated datasets is included too.
\end{par} \vspace{1em}
\begin{par}
\textbf{December 05. 2020}
\end{par} \vspace{1em}
\begin{par}
Tobias Wulf integrated a second characterization dataset for NXP KMZ60 into the sensor array simulation software. The dataset was manually modified in the same way as the TDK TAS2141 dataset. The KMZ60 raw data was provided from Thorben Schüthe. The simulation software was adjusted to run with both datasets now. Additional plots for transfer curves are included for both and same plots for characterization view of KMZ60 as for TAS2141 too.
\end{par} \vspace{1em}
\begin{par}
\textbf{April 01. 2021}
\end{par} \vspace{1em}
\begin{par}
Tobias Wulf integrated GPR algorithms made by Klaus Jünemann as gaussianProcessRegression modul. Additionaly a second kernel was implemented based on the first one by Jünemann. The implementation was transfered from a functional and script based draft version of GPR mechanism into fully initialized model based version which loads needed functionality and parameters into a struct. So prediction and optimization algorithms are working on a structured model frame. Missing model optimization is added to fit model on training data and generalize it to test data. Interface to Sensor Array simulations are done by work on datasets.
\end{par} \vspace{1em}
\begin{par}
Created on September 21. 2020 by Tobias Wulf. Copyright Tobias Wulf 2020.
\end{par} \vspace{1em}
\begin{par}

\end{par} \vspace{1em}



\end{document}

