
% This LaTeX was auto-generated from MATLAB code.
% To make changes, update the MATLAB code and republish this document.

\documentclass{standalone}
\usepackage{graphicx}
\usepackage{listings}
\usepackage{xcolor}
\usepackage{textcomp}
\usepackage[framed, numbered]{matlab-prettifier}

\sloppy
\definecolor{lightgray}{gray}{0.5}
\setlength{\parindent}{0pt}

\begin{document}

    
    
\section*{exportPublishedToPdf}

\begin{par}
Export Matlab generated Tex documentation (publish) to combined LaTeX index file ready compile to appendix manual.
\end{par} \vspace{1em}

\subsection*{Contents}

\begin{itemize}
\setlength{\itemsep}{-1ex}
   \item Requirements
   \item See Also
   \item Start Exporting Script, Clean Up and Load Config
   \item Define Manual TOC
   \item Scan for Tex Files
   \item Export Tex
   \item Write TOC to LaTeX File
\end{itemize}


\subsection*{Requirements}

\begin{itemize}
\setlength{\itemsep}{-1ex}
   \item Other m-files None
   \item Subfunctions: removeFilesFromDir
   \item MAT-files required: data/config.mat
\end{itemize}


\subsection*{See Also}

\begin{itemize}
\setlength{\itemsep}{-1ex}
   \item \begin{verbatim}generateConfigMat\end{verbatim}
   \item \begin{verbatim}publishProjectFilesToHTML\end{verbatim}
   \item \begin{verbatim}Documentation Workflow\end{verbatim}
\end{itemize}
\begin{par}
Created on December 10. 2020 by Tobias Wulf. Copyright Tobias Wulf 2020.
\end{par} \vspace{1em}
\begin{par}

\end{par} \vspace{1em}


\subsection*{Start Exporting Script, Clean Up and Load Config}

\begin{par}
At first clean up junk from workspace and clear prompt for new output. Set project root path to create absolute file path with fullfile function. Load absolute path variables and publishing options from config.mat
\end{par} \vspace{1em}
\begin{lstlisting}[style=Matlab-editor]
disp('Workspace cleaned up ...');
clearvars;
clc;
disp('Load configuration ...');
try
    load('config.mat', 'PathVariables');
catch ME
    rethrow(ME);
end
\end{lstlisting}


\subsection*{Define Manual TOC}

\begin{par}
The maual toc must be in the same order as in helptoc.xml in the publish html folder. The toc is used to generate a latex file to include for appendices.
\end{par} \vspace{1em}
\begin{lstlisting}[style=Matlab-editor]
toc = ["section",           "GaussianProcessDipoleSimulation.tex";
       "section"            "Workflows.tex";
       "subsection",        "Project_Preparation.tex";
       "subsection",        "Project_Structure.tex";
       "subsection",        "Git_Feature_Branch_Workflow.tex";
       "subsection",        "Documentation_Workflow.tex";
       "subsection",        "Simulation_Workflow.tex";
       "section",           "Executable_Scripts.tex";
       "subsection",        "publishProjectFilesToHTML.tex";
       "subsection",        "generateConfigMat.tex";
       "subsection",        "generateSimulationDatasets.tex";
       "subsection",        "deleteSimulationDatasets.tex";
       "subsection",        "deleteSimulationPlots.tex";
       "subsection",        "exportPublishedToPdf.tex";
       "subsection",        "demoGPRModule.tex";
       "subsection",        "investigateKernelParameters.tex";
       "subsection",        "compareGPRKernels.tex";
       "section",           "Source_Code.tex";
       "subsection",        "sensorArraySimulation.tex";
       "subsubsection",     "rotate3DVector.tex";
       "subsubsection",     "generateDipoleRotationMoments.tex";
       "subsubsection",     "generateSensorArraySquareGrid.tex";
       "subsubsection",     "computeDipoleH0Norm.tex";
       "subsubsection",     "computeDipoleHField.tex";
       "subsubsection",     "simulateDipoleSquareSensorArray.tex";
       "subsection",        "gaussianProcessRegression.tex";
       "subsubsection",     "initGPR.tex";
       "subsubsection",     "initGPROptions.tex";
       "subsubsection",     "initTrainDS.tex";
       "subsubsection",     "initKernel.tex";
       "subsubsection",     "initKernelParameters.tex";
       "subsubsection",     "tuneKernel.tex";
       "subsubsection",     "computeTuneCriteria.tex";
       "subsubsection",     "predFrame.tex";
       "subsubsection",     "predDS.tex";
       "subsubsection",     "lossDS.tex";
       "subsubsection",     "optimGPR.tex";
       "subsubsection",     "computeOptimCriteria.tex";
       "subsubsection",     "kernelQFCAPX.tex";
       "paragraph",         "QFCAPX.tex";
       "paragraph",         "meanPolyQFCAPX.tex";
       "paragraph",         "initQFCAPX.tex";
       "subsubsection",     "kernelQFC.tex";
       "paragraph",         "QFC.tex";
       "paragraph",         "meanPolyQFC.tex";
       "paragraph",         "initQFC.tex";
       "subsubsection",     "basicMathFunctions.tex";
       "paragraph",         "sinoids2angles.tex";
       "paragraph",         "angles2sinoids.tex";
       "paragraph",         "decomposeChol.tex";
       "paragraph",         "frobeniusNorm.tex";
       "paragraph",         "computeInverseMatrixProduct.tex";
       "paragraph",         "computeTransposeInverseProduct.tex";
       "paragraph",         "addNoise2Covariance.tex";
       "paragraph",         "computeAlphaWeights.tex";
       "paragraph",         "computeStdLogLoss.tex";
       "paragraph",         "computeLogLikelihood.tex";
       "paragraph",         "estimateBeta.tex";
       "subsection",        "util.tex";
       "subsubsection",     "removeFilesFromDir.tex";
       "subsubsection",     "publishFilesFromDir.tex";
       "subsubsection",     "plotFunctions.tex";
       "paragraph",         "plotTDKCharDataset.tex";
       "paragraph",         "plotTDKCharField.tex";
       "paragraph",         "plotTDKTransferCurves.tex";
       "paragraph",         "plotKMZ60CharDataset.tex";
       "paragraph",         "plotKMZ60CharField.tex";
       "paragraph",         "plotKMZ60TransferCurves.tex";
       "paragraph",         "plotDipoleMagnet.tex";
       "paragraph",         "plotSimulationDataset.tex";
       "paragraph",         "plotSingleSimulationAngle.tex";
       "paragraph",         "plotSimulationSubset.tex";
       "paragraph",         "plotSimulationCosSinStats.tex"
       "paragraph",         "plotSimulationDatasetCircle.tex";
       "section",           "Datasets.tex";
       "subsection",        "TDK_TAS2141_Characterization.tex";
       "subsection",        "NXP_KMZ60_Characterization.tex";
       "subsection",        "Config_Mat.tex";
       "subsection",        "Training_and_Test_Datasets.tex";
       "section",           "Unit_Tests.tex";
       "subsection",        "runTests.tex";
       "subsection",        "removeFilesFromDirTest.tex";
       "subsection",        "rotate3DVectorTest.tex";
       "subsection",        "generateDipoleRotationMomentsTest.tex";
       "subsection",        "generateSensorArraySquareGridTest.tex";
       "subsection",        "computeDipoleH0NormTest.tex";
       "subsection",        "computeDipoleHFieldTest.tex";
       "subsection",        "tiltRotationTest.tex";];

nToc = length(toc);
fprintf("%d toc entries remarked ...\n", nToc);
\end{lstlisting}


\subsection*{Scan for Tex Files}

\begin{par}
Scan for all published Tex files in the project publish directory.
\end{par} \vspace{1em}
\begin{lstlisting}[style=Matlab-editor]
disp('Scan for published files ...');
TEX = dir(fullfile(PathVariables.publishHtmlPath, '*.tex'));
if nToc ~= length(TEX)
    warning(...
        'TOC (%d) length and found Tex (%d) files are diverging.', ...
        nToc, length(TEX));
end
\end{lstlisting}


\subsection*{Export Tex}

\begin{par}
Export found Tex files to Manual file. Each file gets its own represenstation. Filename is kept. Write files into Manual folder under LaTeX subdirectory in docs path. Get filename, move to new path. Write Manual.
\end{par} \vspace{1em}
\begin{lstlisting}[style=Matlab-editor]
disp('Export published Tex to Manual ...');
fprintf('Source: %s\n', TEX(1).folder);
fprintf('Destination: %s\n', PathVariables.exportPublishPath);
for ftex = TEX'
    disp(ftex.name);
    sourcePath = fullfile(ftex.folder, ftex.name);
    destinationPath = fullfile(...
        PathVariables.exportPublishPath, ftex.name);
    try
        [status, msg] = movefile(sourcePath, destinationPath);
        % disp(cmdout);
        if status ~= 1
            error('Export failure.');
        end
    catch ME
        disp(msg);
        rethrow(ME)
    end
end
\end{lstlisting}


\subsection*{Write TOC to LaTeX File}

\begin{par}
Wirete TOC to LaTeX file and generate for each file a subimport along toc content line with marked toc depth.
\end{par} \vspace{1em}
\begin{lstlisting}[style=Matlab-editor]
disp('Write TOC to Manual.tex ...');
fileID = fopen(fullfile(...
    PathVariables.exportPublishPath, 'Manual.tex'), 'w');
fprintf(fileID, "%% appendix software documentation\n");
fprintf(fileID, "%% @author Tobias Wulf\n");
fprintf(fileID, ...
    "%% Autogenerated LaTeX file. Generated by exportPublishedToPdf.\n");
fprintf(fileID, ...
    "%% Software manual with TOC generated in the same script.\n");
fprintf(fileID, "%% Generated on %s.\n\n", datestr(datetime('now')));
fprintf(fileID, "\\documentclass[class=article, crop=false]{standalone}\n");
fprintf(fileID, "\\usepackage[subpreambles=true]{standalone}\n");
fprintf(fileID, "\\usepackage{import}\n\n");
fprintf(fileID, "\\begin{document}\n");
fprintf(fileID, "\\clearpage\n");
fprintf(fileID, ...
    "\\textbf{Matlab software appendix auto generated on %s.}\n\n", ...
    datestr(datetime('now', 'Format', 'y-MM-d')));

for i = 1:nToc
    level = toc(i);
    fName = toc(i,2);
    [~, fstr, ~] = fileparts(fName);
    lstr = lower(strrep(fstr, '_', '-'));
    tstr = strrep(fstr, '_', ' ');

    fprintf(fileID, "\\%s{%s}", level, tstr);
    fprintf(fileID, "\\label{mcode:%s}\n", lstr);
    fprintf(fileID, "\\subimport{./}{%s}\n", fName);
    fprintf(fileID, "\\clearpage\n");
end

fprintf(fileID, "\\end{document}\n");
fclose(fileID);
\end{lstlisting}



\end{document}

