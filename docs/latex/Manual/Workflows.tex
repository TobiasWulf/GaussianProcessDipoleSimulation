
% This LaTeX was auto-generated from MATLAB code.
% To make changes, update the MATLAB code and republish this document.

\documentclass{standalone}
\usepackage{graphicx}
\usepackage{listings}
\usepackage{xcolor}
\usepackage{textcomp}
\usepackage[framed, numbered]{matlab-prettifier}

\sloppy
\definecolor{lightgray}{gray}{0.5}
\setlength{\parindent}{0pt}

\begin{document}

    
    \begin{par}
Developing software needs conventions to produce common results and good working software. There are certain points which matches good written software:
\end{par} \vspace{1em}
\begin{enumerate}
\setlength{\itemsep}{-1ex}
   \item The reuse factor of the wirtten souce code.
   \item Good source code structure or hierachy to expand.
   \item Testing with aprobat frameworks e.g. Unittest.
   \item Source code versioning.
   \item Source code readability and detailed commenting and documentation
\end{enumerate}
\begin{par}
The last point can be split into two points but Matlab provides a publish porcess with in source code comments can be used for documentation. What is probably not detailed enough and needs further documents in completition. Ongoing on that to provide support in guidance for current or upcoming project work it is recommended to declare common workflows for those points.
\end{par} \vspace{1em}
\begin{par}
\textbf{Coding conventions are used from MATLAB Style Guidlines 2.0 by Richard Johnson.}
\end{par} \vspace{1em}


{\footnotesize\textbf{See Also}}

\begin{itemize}
\setlength{\itemsep}{-1ex}
   \item \begin{verbatim}MATLAB Stye Guidlines 2.0\end{verbatim}
\end{itemize}


{\footnotesize\textbf{Project Preparation}}

\begin{par}
How to setup a Matlab project with Git support and simple backup plan.
\end{par} \vspace{1em}


{\footnotesize\textbf{Project Structure}}

\begin{par}
Directory structure, associated task and how to add new elements.
\end{par} \vspace{1em}


{\footnotesize\textbf{Git Feature Branch Workflow}}

\begin{par}
How to work in the project with Git support in feature driven way.
\end{par} \vspace{1em}


{\footnotesize\textbf{Documentation Workflow}}

\begin{par}
How to document the project work in progress and introduce new project elements to publishing process.
\end{par} \vspace{1em}


{\footnotesize\textbf{Simulation Workflow}}

\begin{par}
Best practice simulation workflow for sensor array simulations to generate training and test datasets.
\end{par} \vspace{1em}
\begin{par}
Created on September 21. 2020 by Tobias Wulf. Copyright Tobias Wulf 2020.
\end{par} \vspace{1em}
\begin{par}

\end{par} \vspace{1em}



\end{document}

