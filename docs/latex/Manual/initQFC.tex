
% This LaTeX was auto-generated from MATLAB code.
% To make changes, update the MATLAB code and republish this document.

\documentclass{standalone}
\usepackage{graphicx}
\usepackage{listings}
\usepackage{xcolor}
\usepackage{textcomp}
\usepackage[framed, numbered]{matlab-prettifier}

\sloppy
\definecolor{lightgray}{gray}{0.5}
\setlength{\parindent}{0pt}

\begin{document}

    
    \begin{par}
Attaches QFC kernel to model struct. Depending on mean options attach zero mean functions and sets all related kernel parameters and dependencies to zero. If mean is polynom fitting, attaches meanPolyQFC as basis function to build polynom matrix H and sets a none zero mean function. Bypasses dataset inputs as they are. Kernel works on matrix data.
\end{par} \vspace{1em}

{\footnotesize\textbf{Contents}}

\begin{itemize}
\setlength{\itemsep}{-1ex}
   \item Syntax
   \item Description
   \item Input Argurments
   \item Output Argurments
   \item Requirements
   \item See Also
\end{itemize}


{\footnotesize\textbf{Syntax}}

\begin{lstlisting}[style=Matlab-editor, basicstyle=\ttfamily\scriptsize]
Mdl = initQFC(Mdl)
\end{lstlisting}


{\footnotesize\textbf{Description}}

\begin{par}
\textbf{Mdl = initQFC(Mdl)} loads quadratic fraction covariance function and basis function depending on mean in \textbf{Mdl} struct. Sets input function as bypass.
\end{par} \vspace{1em}


{\footnotesize\textbf{Input Argurments}}

\begin{par}
\textbf{Mdl} struct with model parameter and training data.
\end{par} \vspace{1em}


{\footnotesize\textbf{Output Argurments}}

\begin{par}
\textbf{Mdl} struct with attached kernel functionality
\end{par} \vspace{1em}


{\footnotesize\textbf{Requirements}}

\begin{itemize}
\setlength{\itemsep}{-1ex}
   \item Other m-files required: None
   \item Subfunctions: QFC, meanPolyQFC
   \item MAT-files required: None
\end{itemize}


{\footnotesize\textbf{See Also}}

\begin{itemize}
\setlength{\itemsep}{-1ex}
   \item \begin{verbatim}initKernel\end{verbatim}
   \item \begin{verbatim}meanPolyQFC\end{verbatim}
   \item \begin{verbatim}QFC\end{verbatim}
\end{itemize}
\begin{par}
Created on February 15. 2021 by Tobias Wulf. Copyright Tobias Wulf 2021.
\end{par} \vspace{1em}
\begin{par}

\end{par} \vspace{1em}
\begin{lstlisting}[style=Matlab-editor, basicstyle=\ttfamily\scriptsize]
function Mdl = initQFC(Mdl)

    % set QFC kernel function
    Mdl.kernelFun = @QFC;

    % set input transformation function to apply adjustments to
    % covariance function, here bypass inputs as they are, no transformation of
    % training data needed
    Mdl.inputFun = @(X) X;

    % set mean function to compute cosine and sine H matrix
    switch Mdl.mean
        % zero mean m(x) = 0
        case 'zero'
            % set polyDegree to -1 for no polynom indication
            Mdl.polyDegree = -1;

            % set basis function
            Mdl.basisFun = @(X) 0;

        % mean by polynom m(x) = H' * beta
        case 'poly'
            % set basis function produces a (polyDeg+1)xN H matrix
            Mdl.basisFun = @(X) meanPolyQFC(X, Mdl.polyDegree);

        % end mean select QFC kernel
        otherwise
            error('Unknown mean function %.', Mdl.mean);
    end
end
\end{lstlisting}



\end{document}

