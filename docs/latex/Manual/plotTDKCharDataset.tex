
% This LaTeX was auto-generated from MATLAB code.
% To make changes, update the MATLAB code and republish this document.

\documentclass{standalone}
\usepackage{graphicx}
\usepackage{listings}
\usepackage{xcolor}
\usepackage{textcomp}
\usepackage[framed, numbered]{matlab-prettifier}

\sloppy
\definecolor{lightgray}{gray}{0.5}
\setlength{\parindent}{0pt}

\begin{document}

    
    
\section*{plotTDKCharDataset}

\begin{par}
Explore TDK TAS2141 characterization dataset and plot its content.
\end{par} \vspace{1em}

\subsection*{Contents}

\begin{itemize}
\setlength{\itemsep}{-1ex}
   \item Syntax
   \item Description
   \item Examples
   \item Input Arguments
   \item Output Arguments
   \item Requirements
   \item See Also
\end{itemize}


\subsection*{Syntax}

\begin{lstlisting}[style=Matlab-editor]
plotTDKCharDataset()
\end{lstlisting}


\subsection*{Description}

\begin{par}
\textbf{plotTDKCharDataset()} explores the dataset and plot its content in three docked figure windows. Loads dataset location from config.mat.
\end{par} \vspace{1em}


\subsection*{Examples}

\begin{lstlisting}[style=Matlab-editor]
plotTDKCharDataset();
\end{lstlisting}


\subsection*{Input Arguments}

\begin{par}
\textbf{None}
\end{par} \vspace{1em}


\subsection*{Output Arguments}

\begin{par}
\textbf{None}
\end{par} \vspace{1em}


\subsection*{Requirements}

\begin{itemize}
\setlength{\itemsep}{-1ex}
   \item Other m-files: none
   \item Subfunctions: none
   \item MAT-files required: data/TDK\_TAS2141\_Characterization\_2020-10-22\_18-12-16-827.mat,   data/config.mat
\end{itemize}


\subsection*{See Also}

\begin{itemize}
\setlength{\itemsep}{-1ex}
   \item \begin{verbatim}plot\end{verbatim}
   \item \begin{verbatim}imagesc\end{verbatim}
   \item \begin{verbatim}polarplot\end{verbatim}
\end{itemize}
\begin{par}
Created on October 24. 2020 by Tobias Wulf. Copyright Tobias Wulf 2020.
\end{par} \vspace{1em}
\begin{par}

\end{par} \vspace{1em}
\begin{lstlisting}[style=Matlab-editor]
function plotTDKCharDataset()
    try
        % load dataset path and dataset content into function workspace
        load('config.mat', 'PathVariables');
        load(PathVariables.tdkDatasetPath, 'Data', 'Info');
%         close all;
    catch ME
        rethrow(ME)
    end

    % figure save path for different formats
    %%%%%%%%%%%%%%%%%%%%%%%%%%%%%%%%%%%%%%%%%%%%%%%%%%%%%%%%%%%%%%%%%%%%%%%%%%%%
    %%%%%%%%%%%%%%%%%%%%%%%%%%%%%%%%%%%%%%%%%%%%%%%%%%%%%%%%%%%%%%%%%%%%%%%%%%%%
    fig1Filename = 'tdk_magnetic_stimulus';
    fig1Path = fullfile(PathVariables.saveImagesPath, fig1Filename);

    fig2Filename = 'tdk_bridge_charistic';
    fig2Path = fullfile(PathVariables.saveImagesPath, fig2Filename);

    % load needed data from dataset in to local variables for better handling
    %%%%%%%%%%%%%%%%%%%%%%%%%%%%%%%%%%%%%%%%%%%%%%%%%%%%%%%%%%%%%%%%%%%%%%%%%%%%
    %%%%%%%%%%%%%%%%%%%%%%%%%%%%%%%%%%%%%%%%%%%%%%%%%%%%%%%%%%%%%%%%%%%%%%%%%%%%
    % check if modulation fits to following reconstructioning
    if ~strcmp("triang", Info.MagneticField.Modulation)
        error("Modulation function is not triang.");
    end
    if ~(strcmp("cos", Info.MagneticField.CarrierHx) && ...
            strcmp("sin", Info.MagneticField.CarrierHy))
        error("Carrier functions are not cos or sin.");
    end

    % modulation frequency
    fm = Info.MagneticField.ModulationFrequency;
    % carrier frequency
    fc = Info.MagneticField.CarrierFrequency;
    % max and min amplitude
    Hmax = Info.MagneticField.MaxAmplitude;
    Hmin = Info.MagneticField.MinAmplitude;
    % step range or window size for output picking
    Hsteps = Info.MagneticField.Steps;
    % resoulution of H steps
    Hres = Info.MagneticField.Resolution;
    % get unit strings from
    kApm = Info.Units.MagneticFieldStrength;
    Hz = Info.Units.Frequency;
    mV = Info.Units.SensorOutputVoltage;

    % get dataset infos and format strings to place in figures
    % subtitle string for all figures
    infoStr = join([Info.SensorManufacturer, Info.Sensor, ...
        Info.SensorTechnology, ...
        Info.SensorType, "Sensor Characterization Dataset."]);
    dateStr = join(["Created on", Info.Created, "by", 'Thorben Sch\"uthe', ...
        "and updated on", Info.Edited, "by", Info.Editor + "."]);

    % load characterization data
    Vcos = Data.SensorOutput.CosinusBridge;
    Vsin = Data.SensorOutput.SinusBridge;
    gain = Info.SensorOutput.BridgeGain;

    % clear dataset all loaded
    clear Data Info;
    disp('Info:');
    disp([infoStr; dateStr]);

    % reconstruct magnetic stimulus and reduce the view for example plot by 10
    %%%%%%%%%%%%%%%%%%%%%%%%%%%%%%%%%%%%%%%%%%%%%%%%%%%%%%%%%%%%%%%%%%%%%%%%%%%%
    %%%%%%%%%%%%%%%%%%%%%%%%%%%%%%%%%%%%%%%%%%%%%%%%%%%%%%%%%%%%%%%%%%%%%%%%%%%%
    % number of periods reduced by factor 10
    reduced = 10;
    nPeriods = fc / fm / reduced;
    % number of samples for good looking 40 times nPeriods
    nSamples = nPeriods * 400;
    % half number of samples
    nHalf = round(nSamples / 2);
    % generate angle base
    phi = linspace(0, nPeriods * 2 * pi, nSamples);
    % calculate modulated amplitude, triang returns a column vector, transpose
    Hmag = Hmax * triang(nSamples)';
    % calculate Hx and Hy stimulus
    Hx = Hmag .* cos(phi);
    Hy = Hmag .* sin(phi);
    % index for rising and falling stimulus
    idxR = 1:nHalf;
    idxF = nHalf:nSamples;
    % find absolute min and max values in bridge outputs for uniform colormap
    A = cat(3, Vcos.Rise, Vcos.Fall, Vcos.All, Vcos.Diff, Vsin.Rise, ...
        Vsin.Fall, Vsin.All, Vsin.Diff);
    Vmax = max(A, [], 'all');
    Vmin = min(A, [], 'all');
    clear A;

    % figure 1 magnetic stimulus
    %%%%%%%%%%%%%%%%%%%%%%%%%%%%%%%%%%%%%%%%%%%%%%%%%%%%%%%%%%%%%%%%%%%%%%%%%%%%
    %%%%%%%%%%%%%%%%%%%%%%%%%%%%%%%%%%%%%%%%%%%%%%%%%%%%%%%%%%%%%%%%%%%%%%%%%%%%
    fig1 = figure('Name', 'Magnetic Stimulus');
    tiledlayout(fig1, 2, 2);

    % title and description
    disp("Title: Magnetic Stimulus Reconstructed H_x-/ H_y-Stimulus" + ...
         "in Reduced View");
    disp("Description: Stimulus for characterization in H_x and H_y in " + ...
         "reduced period view by factor 10");
    disp(["a) Triangle modulated cosine carrier for H_x stimulus."; ...
          "b) Triangle modulated sine carrier for H_x stimulus."; ...
          "c) Modulation trajectory for rising stimulus"; ...
          "d) Modulation trajectory for falling stimulus"]);

    % Hx stimulus
    nexttile;
    p = plot(phi, Hmag, phi, -Hmag, phi(idxR), Hx(idxR), phi(idxF), Hx(idxF));
    set(p, {'Color'}, {'k', 'k', 'b', 'r'}');
    legend([p(1) p(3) p(4)], {'mod', 'rise', 'fall'}, 'Location', 'NorthEast');
    xticks((0:0.25*pi:2*pi) * nPeriods);
    xticklabels({'$0$', '$8\pi$', '$16\pi$', '$24\pi$', '$32\pi$', ...
        '$40\pi$', '$48\pi$', '$56\pi$', '$64\pi$'});
    xlim([0 phi(end)]);
    ylim([Hmin Hmax]);
    xlabel('$\phi$ in rad, Periode $\times 10$');
    ylabel(sprintf('$H_x(\\phi)$ in %s', kApm));
    title(sprintf('a) $f_m = %1.2f$ %s, $f_c = %1.2f$ %s', fm, Hz, fc, Hz));

    % Hy stimulus
    nexttile;
    p = plot(phi, Hmag, phi, -Hmag, phi(idxR), Hy(idxR), phi(idxF), Hy(idxF));
    set(p, {'Color'}, {'k', 'k', 'b', 'r'}');
    legend([p(1) p(3) p(4)], {'mod', 'rise', 'fall'}, 'Location', 'NorthEast');
    xticks((0:0.25*pi:2*pi) * nPeriods);
    xticklabels({'$0$', '$8\pi$', '$16\pi$', '$24\pi$', '$32\pi$', ...
        '$40\pi$', '$48\pi$', '$56\pi$', '$64\pi$'});
    xlim([0 phi(end)]);
    ylim([Hmin Hmax]);
    xlabel('$\phi$ in rad, Periode $\times 10$');
    ylabel(sprintf('$H_y(\\phi)$ in %s', kApm));
    title(sprintf('b) $f_m = %1.2f$ %s, $f_c = %1.2f$ %s', fm, Hz, fc, Hz));

    % polar for rising modulation
    nexttile;
    polarplot(phi(idxR), Hmag(idxR), 'b');
    p = gca;
    p.ThetaAxisUnits = 'radians';
    title('c) $|\vec{H}(\phi)| \cdot e^{j\phi}$,  $0<\phi<320\pi$');

    % polar for rising modulation
    nexttile;
    polarplot(phi(idxF), Hmag(idxF), 'r');
    p = gca;
    p.ThetaAxisUnits = 'radians';
    title('d) $|\vec{H}(\phi)| \cdot e^{j\phi}$,  $320<\phi<640\pi$');

    % figure 2 cosinus bridge outputs
    %%%%%%%%%%%%%%%%%%%%%%%%%%%%%%%%%%%%%%%%%%%%%%%%%%%%%%%%%%%%%%%%%%%%%%%%%%%%
    %%%%%%%%%%%%%%%%%%%%%%%%%%%%%%%%%%%%%%%%%%%%%%%%%%%%%%%%%%%%%%%%%%%%%%%%%%%%
    fig2 = figure('Name', 'Cosine and Sine Bridge', 'Position', [0 0 33 30]);

    tiledlayout(fig2, 2, 2);

    % title and description
    disp("Title: Cosine and Sine Bridge. Measured Bridge Outputs" + ...
         " of Corresponding H_x-/ H_y-Amplitudes");
    disp("Description: " + sprintf("H_x, H_y in %s, %d Steps in %.4f %s", ...
        kApm, Hsteps, Hres, kApm));
    disp(["a) Cosine Bridge Rising H-Amplitudes"; ...
          "b) Cosine Bridge Falling H-Amplitudes"; ...
          "c) Sine Bridge Rising H-Amplitudes"; ...
          "d) Sine Bridge Falling H-Amplitudes"]);

    colormap('jet');

    % cosinus bridge recorded during rising stimulus
    nexttile;
    im = imagesc([Hmin Hmax], [Hmin Hmax], Vcos.Rise);
    set(gca, 'YDir', 'normal');
    set(im, 'AlphaData', ~isnan(Vcos.Rise));
    caxis([Vmin, Vmax]);
    xlim([Hmin Hmax]);
    ylim([Hmin Hmax]);
    axis square xy;
    xlabel('$H_x$ in kA/m');
    ylabel('$H_y$ in kA/m');
    title('a) $V_{cos}(H_x, H_y)$');
    yticks([-20 -10 0 10 20]);
    xticks([-20 -10 0 10 20]);

    % cosinus bridge recorded during falling stimulus
    nexttile;
    im = imagesc([Hmin Hmax], [Hmin Hmax], Vcos.Fall);
    set(gca, 'YDir', 'normal');
    set(im, 'AlphaData', ~isnan(Vcos.Fall));
    caxis([Vmin, Vmax]);
    xlim([Hmin Hmax]);
    ylim([Hmin Hmax]);
    axis square xy;
    xlabel('$H_x$ in kA/m');
    ylabel('$H_y$ in kA/m');
    title('b) $V_{cos}(H_x, H_y)$');
    yticks([-20 -10 0 10 20]);
    xticks([-20 -10 0 10 20]);

    % sinus bridge recorded during rising stimulus
    nexttile;
    im = imagesc([Hmin Hmax], [Hmin Hmax], Vsin.Rise);
    set(gca, 'YDir', 'normal');
    set(im, 'AlphaData', ~isnan(Vsin.Rise));
    caxis([Vmin, Vmax]);
    xlim([Hmin Hmax]);
    ylim([Hmin Hmax]);
    axis square xy;
    xlabel('$H_x$ in kA/m');
    ylabel('$H_y$ in kA/m');
    title('c) $V_{sin}(H_x, H_y)$');
    yticks([-20 -10 0 10 20]);
    xticks([-20 -10 0 10 20]);

    % sinus bridge recorded during falling stimulus
    nexttile;
    im = imagesc([Hmin Hmax], [Hmin Hmax], Vsin.Fall);
    set(gca, 'YDir', 'normal');
    set(im, 'AlphaData', ~isnan(Vsin.Fall));
    caxis([Vmin, Vmax]);
    xlim([Hmin Hmax]);
    ylim([Hmin Hmax]);
    axis square xy;
    xlabel('$H_x$ in kA/m');
    ylabel('$H_y$ in kA/m');
    title('d) $V_{sin}(H_x, H_y)$');
    yticks([-20 -10 0 10 20]);
    xticks([-20 -10 0 10 20]);

    % add colorbar and place it overall plots
    cb = colorbar;
    cb.Layout.Tile = 'east';
    cb.Label.String = sprintf(...
        '$V(H_x, H_y)$ in %s, Gain $ = %.1f$', mV, gain);
    cb.Label.Interpreter = 'latex';
    cb.TickLabelInterpreter = 'latex';
    cb.Label.FontSize = 24;

%     yesno = input('Save? [y/n]: ', 's');
%     if strcmp(yesno, 'y')
%         % save results of figure 1
%         savefig(fig1, fig1Path);
%         print(fig1, fig1Path, '-dsvg');
%         print(fig1, fig1Path, '-depsc', '-tiff', '-loose');
%         print(fig1, fig1Path, '-dpdf', '-loose', '-fillpage');
%
%         % save results of figure 2
%         savefig(fig2, fig2Path);
%         print(fig2, fig2Path, '-dsvg');
%         print(fig2, fig2Path, '-depsc', '-tiff', '-loose');
%         print(fig2, fig2Path, '-dpdf', '-loose', '-fillpage');
%     end
%     close(fig1)
%     close(fig2)
end
\end{lstlisting}



\end{document}

