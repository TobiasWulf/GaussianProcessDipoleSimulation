
% This LaTeX was auto-generated from MATLAB code.
% To make changes, update the MATLAB code and republish this document.

\documentclass{standalone}
\usepackage{graphicx}
\usepackage{listings}
\usepackage{xcolor}
\usepackage{textcomp}
\usepackage[framed, numbered]{matlab-prettifier}

\sloppy
\definecolor{lightgray}{gray}{0.5}
\setlength{\parindent}{0pt}

\begin{document}

    
    \begin{par}
Unit Tests are provideing a way to test core functionallity of the written software components. Matlab supports various methods to apply Unit Tests. The designed tests are using script-based testing. So far each function or functionallity needs to be tested in a own test script and further on gathered into a main test script where all standalone test scripts are combined to a test suite and executed at once.
\end{par} \vspace{1em}

{\footnotesize\textbf{Contents}}

\begin{itemize}
\setlength{\itemsep}{-1ex}
   \item runTests
   \item removeFilesFromDirTest
   \item rotate3DVectorTest
   \item generateDipoleRotationMomentsTest
   \item generateSensorArraySquareGridTest
   \item computeDipoleH0NormTest
   \item computeDipoleHFieldTest
   \item tiltRotationTest
   \item Requirements
   \item See Also
\end{itemize}


{\footnotesize\textbf{runTests}}

\begin{par}
Test suite script which executes all Unit Tests scripts at once and gathers the test results in a Matlab table.
\end{par} \vspace{1em}


{\footnotesize\textbf{removeFilesFromDirTest}}

\begin{par}
Test of function removeFilesFromDir. Creates several files and directories and deletes them during testing.
\end{par} \vspace{1em}


{\footnotesize\textbf{rotate3DVectorTest}}

\begin{par}
Test rotate3DVector function. Do some rotations and check results.
\end{par} \vspace{1em}


{\footnotesize\textbf{generateDipoleRotationMomentsTest}}

\begin{par}
Test the generation of magnetic dipole moments for a full rotation between 0° and 360°.
\end{par} \vspace{1em}


{\footnotesize\textbf{generateSensorArraySquareGridTest}}

\begin{par}
Test the meshgrid generation of the sensor array and shifting it in x and y direction.
\end{par} \vspace{1em}


{\footnotesize\textbf{computeDipoleH0NormTest}}

\begin{par}
Test magnetic field norming function. Simple test of consitent data.
\end{par} \vspace{1em}


{\footnotesize\textbf{computeDipoleHFieldTest}}

\begin{par}
Test the magnetic dipole equation to generate dipole fields in 3D meshgrid of data points. Test field characteristics like symmetry and so on.
\end{par} \vspace{1em}


{\footnotesize\textbf{tiltRotationTest}}

\begin{par}
Test tilt rotation of a dipole magnetic. Tilt magnet and coordinate cross to fetch pole values during rotation.
\end{par} \vspace{1em}


{\footnotesize\textbf{Requirements}}

\begin{itemize}
\setlength{\itemsep}{-1ex}
   \item Other m-files required: None
   \item Subfunctions: None
   \item MAT-files required: None
\end{itemize}


{\footnotesize\textbf{See Also}}

\begin{itemize}
\setlength{\itemsep}{-1ex}
   \item \begin{verbatim}Script-Based Unit Tests\end{verbatim}
   \item \begin{verbatim}Write Script-Based Unit Tests\end{verbatim}
   \item \begin{verbatim}Write Script-Based Unit Tests Using Local Functions\end{verbatim}
   \item \begin{verbatim}Analyze Test Case Result\end{verbatim}
\end{itemize}
\begin{par}
Created on December 14. 2020 by Tobias Wulf. Copyright Tobias Wulf 2020.
\end{par} \vspace{1em}
\begin{par}

\end{par} \vspace{1em}



\end{document}

