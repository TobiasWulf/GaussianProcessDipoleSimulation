
% This LaTeX was auto-generated from MATLAB code.
% To make changes, update the MATLAB code and republish this document.

\documentclass{standalone}
\usepackage{graphicx}
\usepackage{listings}
\usepackage{xcolor}
\usepackage{textcomp}
\usepackage[framed, numbered]{matlab-prettifier}

\sloppy
\definecolor{lightgray}{gray}{0.5}
\setlength{\parindent}{0pt}

\begin{document}

    
    
\section*{investigateKernelParameters}

\begin{par}
Sweep kernel parameters against inner tuning criteria which is built by the logarithmic likelihoods for cosine and sine fit on training datasets.
\end{par} \vspace{1em}

\subsection*{Contents}

\begin{itemize}
\setlength{\itemsep}{-1ex}
   \item Requirements
   \item See Also
   \item Start Script, Load Config and Read in Datasets
   \item Create GPR Model for Investigation
   \item Sweep Title with Model Parameters
   \item Execute Parameter Sweep with Constant Noise
   \item Execute Parameter Sweep with Constant Variance
   \item Execute Parameter Sweep with Constant Lengthscale
   \item Sweep Kernel Parameters vs. Likelihood Criteria with Constant Noise
   \item Sweep Kernel Parameters vs. Likelihood Criteria with Constant Variance
   \item Sweep Kernel Parameters vs. Likelihood Criteria with Constant Lengthscale
\end{itemize}


\subsection*{Requirements}

\begin{itemize}
\setlength{\itemsep}{-1ex}
   \item Other m-files required: gaussianProcessRegression module files
   \item Subfunctions: none
   \item MAT-files required: data/config.mat, corresponding Training and Test dataset
\end{itemize}


\subsection*{See Also}

\begin{itemize}
\setlength{\itemsep}{-1ex}
   \item \begin{verbatim}gaussianProcessRegression\end{verbatim}
   \item \begin{verbatim}initGPR\end{verbatim}
   \item \begin{verbatim}tuneGPR\end{verbatim}
   \item \begin{verbatim}optimGPR.html\end{verbatim}
   \item \begin{verbatim}generateConfigMat\end{verbatim}
\end{itemize}
\begin{par}
Created on March 13. 2021 by Tobias Wulf. Copyright Tobias Wulf 2021.
\end{par} \vspace{1em}
\begin{par}

\end{par} \vspace{1em}


\subsection*{Start Script, Load Config and Read in Datasets}

\begin{lstlisting}[style=Matlab-editor]
clc;
disp('Start GPR module demonstration ...');
clearvars;
%close all;

disp('Load config ...');
load config.mat PathVariables GPROptions;

disp('Search for datasets ...');
TrainFiles = dir(fullfile(PathVariables.trainingDataPath, 'Training*.mat'));
TestFiles = dir(fullfile(PathVariables.testDataPath, 'Test*.mat'));
assert(~isempty(TrainFiles), 'No training datasets found.');
assert(~isempty(TestFiles), 'No test datasets found.');

disp('Load first found datasets ...');
try
    TrainDS = load(fullfile(TrainFiles(1).folder, TrainFiles(1).name));
    TestDS = load(fullfile(TestFiles(1).folder, TestFiles(1).name));

catch ME
    rethrow(ME)
end

disp('Check dataset coordinates corresponds ...');
assert(all(TrainDS.Data.X == TestDS.Data.X, 'all'), 'Wrong X grid.');
assert(all(TrainDS.Data.Y == TestDS.Data.Y, 'all'), 'Wrong Y grid.');
assert(all(TrainDS.Data.Z == TestDS.Data.Z, 'all'), 'Wrong Z grid.');
\end{lstlisting}


\subsection*{Create GPR Model for Investigation}

\begin{lstlisting}[style=Matlab-editor]
disp('Create GPR modles ...');
Mdl1 = optimGPR(TrainDS, TestDS, GPROptions, 0);
\end{lstlisting}


\subsection*{Sweep Title with Model Parameters}

\begin{lstlisting}[style=Matlab-editor]
titleStr = "Kernel %s: $\\sigma_f = %1.2f$, $\\sigma_l = %1.2f$," + ...
    " $\\sigma_n^2 = %1.2e$, $N = %d$\n" +...
    "$%d \\times %d$ Sensor-Array, Posistion: $(%1.1f,%1.1f,-%1.1f)$ mm," + ...
    " Magnet Tilt: $%2.1f^\\circ$";
titleStr = sprintf(titleStr, ...
    Mdl1.kernel, Mdl1.theta(1), Mdl1.theta(2), Mdl1.s2n, ...
    Mdl1.N, Mdl1.D, Mdl1.D, ...
    TestDS.Info.UseOptions.xPos, ...
    TestDS.Info.UseOptions.yPos, ...
    TestDS.Info.UseOptions.zPos, ...
    TestDS.Info.UseOptions.tilt);
\end{lstlisting}


\subsection*{Execute Parameter Sweep with Constant Noise}

\begin{lstlisting}[style=Matlab-editor]
nEval = 300;
disp('Sweep kernel parameters with constant noise ...');
sweepKernelWithConstNoise(Mdl1, nEval, titleStr, PathVariables)
\end{lstlisting}


\subsection*{Execute Parameter Sweep with Constant Variance}

\begin{lstlisting}[style=Matlab-editor]
nEval = 300;
disp('Sweep kernel parameters with constant variance ...');
sweepKernelWithConstVariance(Mdl1, nEval, titleStr, PathVariables)
\end{lstlisting}


\subsection*{Execute Parameter Sweep with Constant Lengthscale}

\begin{lstlisting}[style=Matlab-editor]
nEval = 300;
disp('Sweep kernel parameters with constant lengthscale ...');
sweepKernelWithConstLengthscale(Mdl1, nEval, titleStr, PathVariables)
\end{lstlisting}


\subsection*{Sweep Kernel Parameters vs. Likelihood Criteria with Constant Noise}

\begin{lstlisting}[style=Matlab-editor]
function sweepKernelWithConstNoise(Mdl, nEval, titleStr, PathVariables)

    % create sweep parameters for sweeping theta to given modle
    s2f = linspace(Mdl.s2fBounds(1) * 0.1, Mdl.s2fBounds(2) * 10, nEval);
    sl = linspace(Mdl.slBounds(1) * 0.1, Mdl.slBounds(2) * 10, nEval);
    [sl, s2f] = meshgrid(sl, s2f);

    % allocate memory for inner tuning criteria, combined likelihoods for cosine
    % and sine fit on trainings data
    RLI = zeros(nEval, nEval);

    % run sweep in multiprocess pool to gain speed
    parfor i = 1:nEval
        for j = 1:nEval
            % compute sweep with tuning criteria of inner GPR optimization of
            % tuning GPR kernel parameters
            RLI(i,j) = computeTuneCriteria([s2f(i,j) sl(i,j)], Mdl);
        end
    end

    % plot results in countour plot
    fig = figure('Name', 'Sweep Kernel Parameters with Constant Noise', ...
        'Units', 'normalize', 'OuterPosition', [0 0 1 1]);

    % plot sweep with log axis
    contourf(sl, s2f, RLI, linspace(min(RLI, [], 'all') + 1, 1, 10), ...
        'LineWidth', 1.5);
    set(gca, 'YScale', 'log')
    set(gca, 'XScale', 'log')
    hold on;
    grid on;

    % plot bounds origin model parameters
    p1 = yline(Mdl.s2fBounds(1), 'k-.', 'LineWidth', 2.5);
    yline(Mdl.s2fBounds(2), 'k-.', 'LineWidth', 2.5);
    yline(Mdl.theta(1), 'k', 'LineWidth', 2.5);
    xline(Mdl.slBounds(1), 'k-.', 'LineWidth', 2.5);
    xline(Mdl.slBounds(2), 'k-.', 'LineWidth', 2.5);
    xline(Mdl.theta(2), 'k', 'LineWidth', 2.5);

    % plot fmincon search area
    p2 = patch( ...
        [Mdl.slBounds(1), Mdl.slBounds(2), ...
         Mdl.slBounds(2), Mdl.slBounds(1)], ...
        [Mdl.s2fBounds(1) Mdl.s2fBounds(1), ...
         Mdl.s2fBounds(2) Mdl.s2fBounds(2)],...
        [0.8 0.8 0.8], 'FaceAlpha', 0.7);

    % plot argmin fmincon result
    p3 = scatter(Mdl.theta(2), Mdl.theta(1), 60, [0.8 0.8 0.8], ...
        'filled', 'MarkerEdgeColor', 'k', 'LineWidth', 1.5);

    % labels, titles, legends
    xlabel('$\sigma_l$')
    ylabel('$\sigma_f^2$')
    title(titleStr);
    stStr = "$\sigma_f^2,\sigma_l|\sigma_n^2 = " + ...
        "\arg\min\tilde{R}_\mathcal{LI}" + ...
        "(\sigma_f^2,\sigma_l|\sigma_n^2)$ f. $\sigma_n^2 = const.$";
    subtitle(stStr);
    legend([p1, p2, p3], ...
        {"Parameter Bounds", "Search Area", ...
         sprintf("fmincon $\\tilde{R}_\\mathcal{LI}(%1.2f,%1.2f|%1.2e)=%1.2f$",...
            Mdl.theta, Mdl.s2n, -(Mdl.LMLcos + Mdl.LMLsin))}, ...
        'Location', 'South')

    cb = colorbar;
    cb.TickLabelInterpreter = 'latex';
    cb.Label.Interpreter = 'latex';
    cb.Label.FontSize = 24;
    cbStr = "$\tilde{R}_\mathcal{LI}(\sigma_f^2,\sigma_l|\sigma_n^2)$";
    cb.Label.String = cbStr;

    % save and close
%     fPath = fullfile(PathVariables.saveImagesPath, 'Sweep_Kernel_Const_Noise');
%     print(fig, fPath, '-dsvg');
%     close(fig);
end
\end{lstlisting}


\subsection*{Sweep Kernel Parameters vs. Likelihood Criteria with Constant Variance}

\begin{lstlisting}[style=Matlab-editor]
function sweepKernelWithConstVariance(Mdl, nEval, titleStr, PathVariables)

    % kepp s2n origin to plot later
    s2nOrigin = Mdl.s2n;

    % create sweep parameters for sweeping lengthscale and noise to given modle
    s2n = linspace(Mdl.s2nBounds(1) * 0.1, Mdl.s2nBounds(2) * 10, nEval);
    sl = linspace(Mdl.slBounds(1) * 0.1, Mdl.slBounds(2) * 10, nEval);
    s2f = Mdl.theta(1);

    % allocate memory for inner tuning criteria, combined likelihoods for cosine
    % and sine fit on trainings data
    RLI = zeros(nEval, nEval);

    % run sweep in multiprocess pool to gain speed
    for i = 1:nEval
        % assign struct values to compute corresponding lenght scale row wise
        % due to parfor struct issue
        Mdl.s2n = s2n(i);
        parfor j = 1:nEval
            % compute sweep with tuning criteria of inner GPR optimization of
            % tuning GPR kernel parameters, variance is set to 1
            RLI(i,j) = computeTuneCriteria([s2f sl(j)], Mdl);
        end
    end

    % generate grid on vectors to plot results
    [sl, s2n] = meshgrid(sl, s2n);

    % plot results in countour plot
    fig = figure('Name', 'Sweep Kernel Parameters with Constant Variance', ...
        'Units', 'normalize', 'OuterPosition', [0 0 1 1]);

    % plot sweep with log axis
    contourf(sl, s2n, RLI, linspace(min(RLI, [], 'all') + 1, 1, 10), ...
        'LineWidth', 1.5);
    set(gca, 'YScale', 'log')
    set(gca, 'XScale', 'log')
    hold on;
    grid on;

    % plot bounds origin model parameters
    p1 = yline(Mdl.s2nBounds(1), 'k-.', 'LineWidth', 2.5);
    yline(Mdl.s2nBounds(2), 'k-.', 'LineWidth', 2.5);
    yline(s2nOrigin, 'k', 'LineWidth', 2.5);
    xline(Mdl.slBounds(1), 'k-.', 'LineWidth', 2.5);
    xline(Mdl.slBounds(2), 'k-.', 'LineWidth', 2.5);
    xline(Mdl.theta(2), 'k', 'LineWidth', 2.5);

    % plot fmincon search area
    p2 = patch( ...
        [Mdl.slBounds(1), Mdl.slBounds(2), ...
         Mdl.slBounds(2), Mdl.slBounds(1)], ...
        [Mdl.s2nBounds(1) Mdl.s2nBounds(1), ...
         Mdl.s2nBounds(2) Mdl.s2nBounds(2)],...
        [0.8 0.8 0.8], 'FaceAlpha', 0.7);

    % plot argmin fmincon result
    p3 = scatter(Mdl.theta(2), s2nOrigin, 60, [0.8 0.8 0.8], ...
        'filled', 'MarkerEdgeColor', 'k', 'LineWidth', 1.5);

    % labels, titles, legends
    xlabel('$\sigma_l$')
    ylabel('$\sigma_n^2$')
    title(titleStr);
    stStr = "$\sigma_f^2,\sigma_l|\sigma_n^2 = " + ...
        "\arg\min\tilde{R}_\mathcal{LI}" + ...
        "(\sigma_f^2,\sigma_l|\sigma_n^2)$ f. $\sigma_f^2 = const.$";
    subtitle(stStr);
    legend([p1, p2, p3], ...
        {"Parameter Bounds", "Search Area", ...
         sprintf("fmincon $\\tilde{R}_\\mathcal{LI}(%1.2f,%1.2f|%1.2e)=%1.2f$",...
            Mdl.theta, s2nOrigin, -(Mdl.LMLcos + Mdl.LMLsin))}, ...
        'Location', 'South')

    cb = colorbar;
    cb.TickLabelInterpreter = 'latex';
    cb.Label.Interpreter = 'latex';
    cb.Label.FontSize = 24;
    cbStr = "$\tilde{R}_\mathcal{LI}(\sigma_f^2,\sigma_l|\sigma_n^2)$";
    cb.Label.String = cbStr;

    % save and close
%     fPath = fullfile(PathVariables.saveImagesPath, 'Sweep_Kernel_Const_Var');
%     print(fig, fPath, '-dsvg');
%     close(fig);
end
\end{lstlisting}


\subsection*{Sweep Kernel Parameters vs. Likelihood Criteria with Constant Lengthscale}

\begin{lstlisting}[style=Matlab-editor]
function sweepKernelWithConstLengthscale(Mdl, nEval, titleStr, PathVariables)

    % kepp s2n origin to plot later
    s2nOrigin = Mdl.s2n;

    % create sweep parameters for sweeping lengthscale and noise to given modle
    s2n = linspace(Mdl.s2nBounds(1) * 0.1, Mdl.s2nBounds(2) * 10, nEval);
    s2f = linspace(Mdl.s2fBounds(1) * 0.1, Mdl.s2fBounds(2) * 10, nEval);
    sl = Mdl.theta(2);

    % allocate memory for inner tuning criteria, combined likelihoods for cosine
    % and sine fit on trainings data
    RLI = zeros(nEval, nEval);

    % run sweep in multiprocess pool to gain speed
    for i = 1:nEval
        % assign struct values to compute corresponding lenght scale row wise
        % due to parfor struct issue
        Mdl.s2n = s2n(i);
        parfor j = 1:nEval
            % compute sweep with tuning criteria of inner GPR optimization of
            % tuning GPR kernel parameters, variance is set to 1
            RLI(i,j) = computeTuneCriteria([s2f(j) sl], Mdl);
        end
    end

    % generate grid on vectors to plot results
    [s2f, s2n] = meshgrid(s2f, s2n);

    % plot results in countour plot
    fig = figure('Name', 'Sweep Kernel Parameters with Constant Lenghtscale',...
        'Units', 'normalize', 'OuterPosition', [0 0 1 1]);

    % plot sweep with log axis
    contourf(s2f, s2n, RLI, linspace(min(RLI, [], 'all') + 1, 1, 10), ...
        'LineWidth', 1.5);
    set(gca, 'YScale', 'log')
    set(gca, 'XScale', 'log')
    hold on;
    grid on;

    % plot bounds origin model parameters
    p1 = yline(Mdl.s2nBounds(1), 'k-.', 'LineWidth', 2.5);
    yline(Mdl.s2nBounds(2), 'k-.', 'LineWidth', 2.5);
    yline(s2nOrigin, 'k', 'LineWidth', 2.5);
    xline(Mdl.s2fBounds(1), 'k-.', 'LineWidth', 2.5);
    xline(Mdl.s2fBounds(2), 'k-.', 'LineWidth', 2.5);
    xline(Mdl.theta(1), 'k', 'LineWidth', 2.5);

    % plot fmincon search area
    p2 = patch( ...
        [Mdl.s2fBounds(1), Mdl.s2fBounds(2), ...
         Mdl.s2fBounds(2), Mdl.s2fBounds(1)], ...
        [Mdl.s2nBounds(1) Mdl.s2nBounds(1), ...
         Mdl.s2nBounds(2) Mdl.s2nBounds(2)],...
        [0.8 0.8 0.8], 'FaceAlpha', 0.7);

    % plot argmin fmincon result
    p3 = scatter(Mdl.theta(1), s2nOrigin, 60, [0.8 0.8 0.8], ...
        'filled', 'MarkerEdgeColor', 'k', 'LineWidth', 1.5);

    % labels, titles, legends
    xlabel('$\sigma_f^2$')
    ylabel('$\sigma_n^2$')
    title(titleStr);
    stStr = "$\sigma_f^2,\sigma_l|\sigma_n^2 = " + ...
        "\arg\min\tilde{R}_\mathcal{LI}" + ...
        "(\sigma_f^2,\sigma_l|\sigma_n^2)$ f. $\sigma_l = const.$";
    subtitle(stStr);
    legend([p1, p2, p3], ...
        {"Parameter Bounds", "Search Area", ...
         sprintf("fmincon $\\tilde{R}_\\mathcal{LI}(%1.2f,%1.2f|%1.2e)=%1.2f$",...
            Mdl.theta, s2nOrigin, -(Mdl.LMLcos + Mdl.LMLsin))}, ...
        'Location', 'South')

    cb = colorbar;
    cb.TickLabelInterpreter = 'latex';
    cb.Label.Interpreter = 'latex';
    cb.Label.FontSize = 24;
    cbStr = "$\tilde{R}_\mathcal{LI}(\sigma_f^2,\sigma_l|\sigma_n^2)$";
    cb.Label.String = cbStr;

    % save and close
%     fPath = fullfile(PathVariables.saveImagesPath, 'Sweep_Kernel_Const_Len');
%     print(fig, fPath, '-dsvg');
%     close(fig);
end
\end{lstlisting}



\end{document}

