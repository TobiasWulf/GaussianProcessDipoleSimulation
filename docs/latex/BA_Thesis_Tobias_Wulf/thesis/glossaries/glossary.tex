% !TEX root = ./thesis.tex
% list of all glossaries in thesis
% @author Tobias Wulf

% take care that all labels are unique in the whole document

%\newglossaryentry{gl:}{
%	name={},
%	description={}
%}

% glossary entries
\newglossaryentry{gl:haw}{
	name={HAW Hamburg},
	description={Die HAW Hamburg ist die Hochschule für Angewandte Wissenschaften in Hamburg und war die ehemalige Fachhochschule am Berliner Tor}
}

\newglossaryentry{gl:ags}{
	name={Arbeitsgruppe Sensorik},
	description={Die Arbeitsgruppe Sensorik steht unter Leitung von Prof. Dr.-Ing. Karl-Ragmar Riemschneider und ist unter dem Department Informations- und Elektrotechnik Teil der Fakultät Technik un Informatik an der HAW Hambug}
}


\newglossaryentry{gl:tmr}{
	name={TMR-Effekt},
	description={Tunnel-Magnetoresistiver-Effekt}
}

\newglossaryentry{gl:gmr}{
	name={GMR-Effekt},
	description={Riesiger-Magnetoresistiver-Effekt}
}

\newglossaryentry{gl:amr}{
	name={AMR-Effekt},
	description={Anisotroper-Magnetoresistiver-Effekt}
}

\newglossaryentry{gl:sensorkopf}{
	name={Sensorkopf},
	description={Signalerzeugender Teil eines Sensor-IC, dem eine Einheit zur weiteren Signalverarbeitung 
	nachgeschaltet ist. Für die Drehwinkelerfassung besteht steht die Signalerzeugung zumeist aus zwei verdrehten Wheatstone-Brücken, deren einzelne Widerstände mittels magnetoresistiven Materialien aufgebaut sind}
}


\newglossaryentry{gl:wheatstonebruecken}{
	name={Wheatstone'sche Brückenschaltung},
	description={Messbrückenschaltung, bestehend aus zwei Spannungsteilern, die parallel zu einer gemeinsamen Quelle geschaltet sind. Es wird eine Differnzspannung über die Mittelabgriffe der Spannungsteiler gemessen. Allgemein bekanntes Messprinzip. 1833 von Samuel Hunter Christie erfunden und nach dem birtischen Physiker Sir Charles Wheatstone benannt. Abgekürzt auch Wheatstone-Brücken oder in der Sensorik auch Sensorbrücken genannt. Ein Sensorkopf für die Winkelmessung setzt sich in der Regel aus zwei solch gearteter Brückenschaltungen zusammen}
}


\newglossaryentry{gl:kennfeld}{
	name={Kennfeld},
	description={Zweidimensionales Charakterisierungsabbild einer Wheatstone'schen Sensorbrücke eines magnetoresistiven Winkelsensors. Erstellt durch die Kennfeldmethode zur Charakterisierung von magnetischen Winkelsensoren}
}

\newglossaryentry{gl:kennfeldpaar}{
	name={Kennfeldpaar},
	description={Charakterisierungsergbnis der Kennfeldmethode für die Charakterisierung magnetoresistiver Winkelsensoren. Bestehend aus zwei Kennfeldern, jeweils als Repräsentanten der Wheatstone-Brücken eines Winkelsensors}
}

\newglossaryentry{gl:kennfeldmethode}{
	name={Kennfeldmethode},
	description={Charakterisierungsverfahren zur Ausmessung magnetoresitiver Winkelsensoren, bestehend aus zwei zueinander verdrehten Wheatstone-Brücken. Die resultierenden Charakterisierungergebnisse können als Kennfelddatensätze zur Simulation von magnetischen Sensoren genutzt werden}
}

\newglossaryentry{gl:kreuzspulensystem}{
	name={Kreuzspulen-System},
	description={Spulensystem in Kreuzanordnung in dessen Mitte ein zu messender Senor-IC platziert wird. Die Spulen sind Maßanfertigungen mit ganz bestimmten Messeigenschaften. Spulenfaktoren sind speziell ausgerechnet und nachgemessen. Kernelement des Kreuzspulen-Messstandes. Eingespeiste Spulenströme erzeugen entsprechende magnetische Felder in $X$- und $Y$-Richtung. Die Spulenströme erzeugen entsprechend der Spulenfaktoren, $H_x$- und $H_y$-Feldstärken. Die Feldstärken sind direkt proportional zu den Einspeiseströmen. Die Ströme werden über niederohmige Shunt-Widerstände gemessen. Die Feldstärken können so über die Spulenfaktoren zurückgerechnet und zur Auswertung genutzt werden}
}

\newglossaryentry{gl:kreuzspulenmessstand}{
	name={Kreuzspulen-Messstand},
	description={Automatisierter Messtand zur Charakterisierung von Winkelsensoren. Der Messtand nutzt ein Kreuzspulen-System, in dessen Mitte der Winkelsensor platziert ist. Das Spulensystem erzeugt ein moduliertes, langsam rotierendes Anregungsmagnetfeld. Parallel zeichnet der Messtand, die zur Charaktisierung nötigen Spannungsausgaben des Sensors und Anregungsströme der Spulen auf. Beides erfolgt programmatisch. Die aufgezeichneten Messdaten können im Anschluss zu Kennfeldern evaluiert werden}
}

\newglossaryentry{gl:sensor-array}{
	name={Sensor-Array},
	description={Array aus geometrisch, gleichmäßig angeordneten Einzelsenoren, die messtechnische Aufgaben im Verbund bewältigen. Jeder einzelne Senor stellt dabei ein Sensor-Pixel dar. Erhobene Messdaten sind entsprechend der Sensor-Array-Struktur in Matixdatenformaten zu bewerten und zu behandeln}
}

\newglossaryentry{gl:sensor-pixel}{
	name={Sensor-Pixel},
	description={Einzelsensor im Sensor-Array. Allein betrachtet als vollwertiger, analoger Sensor zu betrachten. Mehrere Sensor-Pixel zusammenverschaltet ergeben ein Sensor-Array}
}

\newglossaryentry{gl:gebermagnet}{
	name={Gebermagnet},
	description={Encoder bzw. magnetischer Stimulus zur Anregung des Sensor-Arrays. In dieser Arbeit approximiert als Kugelmagnet über die Dipol-Feldgleichung}
}

\newglossaryentry{gl:trainingsdatensatz}{
	name={Trainingsdatensatz},
	description={Winkeldatensatz, der Referenzwinkel und dazugehörige Trainingsdaten in dreidimensionalen Messwertmatrizen beinhaltet. Prozessiert mittels Sensor-Array-Simulation}
}

\newglossaryentry{gl:testdatensatz}{
	name={Testdatensatz},
	description={Winkeldatensatz, der eine beliebige Anzahl an Simulationswinkel und dazugehörige Testdaten in dreidimensionalen Messwertmatrizen beinhaltet. Prozessiert mittels Sensor-Array-Simulation}
}

\newglossaryentry{gl:messwetmatrizen}{
	name={Messwertmatrizen},
	description={Dreidimensionale Matrizen, die Messwerte des Sensor-Arrays beinhalten. Die erste und zweite Position korreliert mit der Anordnung des Arrays. Die dritte Dimension korreliert mit Simulationswinkeln}
}

\newglossaryentry{gl:meshgrid}{
	name={Meshgrid},
	description={Als Meshgrid ist hier die Annordung der Sensor-Pixel im Sensor-Array zu betrachten. An den einzelnen Grid-Punkten werden Magnetfelder simuliert und das Sensorverhalten abstrahiert}
}

\newglossaryentry{gl:gaus-pro-reg}{
	name={Gauß-Prozesse für Regression},
	description={Gauß- oder Gauß'sche Prozesse für Regression sind statistische Prozesse, die auf Normalverteilungen basieren und kommen in mehrdimensionalen Regressionsverfahren zur Anwendung. Dabei werden weitere Normalverteilungen über Prozessparameter gelegt, sodass über das Lösen von Wahrscheinlichkeitsdichteintegralen und Autokorrelation ein Ansatz zum maschinellen Lernen etabliert werden kann}
}

\newglossaryentry{gl:kernel}{
	name={Kernel},
	description={Kernel oder auch Kernel-Funktion wird die im Gauß-Prozess zur Anwendung kommende Kovarianzfunktion genannt. Diese ist maßgebend für das Verhalten des resultierenden mathematischen Modells als Kernfunktion}
}

\newglossaryentry{gl:kernelparam}{
	name={Kernel-Parameter},
	description={Kernel-Parameter sind die Parameter zur Skalierung der Kovarianzfunktion. In der Literatur sind diese oft auch als Hyperparameter bezeichnet, da ebenfalls Normalverteilungen für diese Parameter vorliegen. Die Verteilungen werden genutzt, um die gültigen Parameter über das Lösen von Minimierungsproblemen mittels logarithmischer Modellplausibilitäten (engl. log. Likelihood) zu finden}
}

\newglossaryentry{gl:rauschniveau}{
	name={Rauschniveau},
	description={Engl. Noise Level. Ist ein konstanter Parameter, der für verauschte Beobachtungen (engl. Noisy Observations) auf die Diagonale der Kovarianzmatrix addiert wird. Dieser Parameter ist als erweiteter Kernel-Parameter zu betrachten und kann in Kombination mit Modellverlusten zur Modellgeneralisierung herangezogen werden}
}


\newglossaryentry{gl:likelihoods}{
	name={Logarithmic-Marginal-Likelihood},
	description={Übersetzt logaritmische marginale Plausibilität oder Evidenz. Ist die logarithmische Lösung von Wahrscheinlichkeitsdichteintegralen für normalverteilte Wahrscheinlichkeiten in Abhängigkeit von weiteren Variablen oder Parametern}
}

\newglossaryentry{gl:modellverlust}{
	name={Modellverlust},
	description={Modellverlust sind standardisierte logarithmische Verluste (engl. Loss), ähnlich wie für die Modellplausibilität wird hier ein Wahrscheinlichkeitsdichteintegral logarithmisch gelöst. Allerdings bezieht sich der Verlust auf ein Residual aus idealem Erwartungswert und Regressionsergebnis unter Berücksichtigung der Standardabweichung in der Regression. Modellverluste können in Bezug auf Testdaten Aussagen über die Modellgeneralisierung liefern}
}


\newglossaryentry{gl:ref-model}{
	name={Regressionsmodell},
	description={Als Regressionsmodell ist hier das durch Gauß-Prozesse initilisierte Regressionsverfahren inklusive Optimierungsalgorithmen zu betrachten}
}

\newglossaryentry{gl:kovarianzmatrix}{
	name={Kovarianzmatrix},
	description={Die Kovarianzmatrix resultiert aus Kernel-Funktion und Einspeisung von Trainingsdaten in das Regressionsmodell. Sie stellt eine Autokorrelation der Traingisdaten dar und liefert Autokorrelationskoeffizienten der Daten zueinander und zu sich selbst}
}

\newglossaryentry{gl:Samples}{
	name={Samples},
	description={Samples sind in Bezug zur Gauß-Prozess-Regression ein einzelner Trainingsdatenpunkt. In der Literatur ist das oft auch als Beobachtung (engl. Observation) genannt}
}

\newglossaryentry{gl:kernelqfc}{
	name={KernelQFC},
	description={Kernel-Submodul Quadratic-Fractional-Covariance für die Gauß-Prozess-Regression. Nutzt die Frobenius-Norm als Abstandsfunktion. Modelltrainingsdaten basieren hier auf Matrizen}
}

\newglossaryentry{gl:kernelqfcapx}{
	name={KernelQFCAPX},
	description={Kernel-Submodul Quadratic-Fractional-Covariance-Approximated für die Gauß-Prozess-Regression. Nutzt die euklidische Norm als Abstandsfunktion. Modelltrainingsdaten basieren hier auf Vektoren bzw. Skalare. Sensor-Array-Daten in Form von Matrizen werden eingangs zu Skalaren mittels Frobenius normiert}
}

\newglossaryentry{gl:trainingsphase}{
	name={Trainingsphase},
	description={In der Trainingsphase wird das Regressionsmodell anhand von Trainingsdaten initialiert und getrimmt. Die Trimmung ist die innere Modelloptimierung und eingebettet in einer äußeren Optimierung. In der äußeren Optimierung wird das Modell unter Einbezug von weiteren Testdaten generalisiert, sodass von Trainingsdaten abweichende Datensätze verarbeitet werden können}
}

\newglossaryentry{gl:innere-opt}{
	name={Innere Modelloptimierung},
	description={Die innere Modelloptimierung sucht passende Kernel-Parameter unter Einbezug von Trainingsdaten. Es wird ein Minimierungsproblem über die Modellplausibilität in Abhängigkeit der Kernel-Parameter gelöst}
}

\newglossaryentry{gl:aus-opt}{
	name={Äußere Optimierung},
	description={Die äußere Optimierung sucht nach passenden Modellen unter Einbezug von Testdaten. Es wird ein Minimierungpoblem über die Modellverluste in Abhängigkeit der Testdaten und des Rauschniveaus gelöst}
}

\newglossaryentry{gl:arbeitsphase}{
	name={Arbeitsphase},
	description={In der Arbeitsphase liegt ein vollständig initialisiertes Regressionsmodell vor. Dieses kann optimiert sein. In der Arbeitsphase werden basierend auf Messwertmatrizen Vorhersagen über Winkel und Radius getroffen. Beides unterstützt und bewertet durch entsprechende Konfidenzintervalle, die Aussagen zur Messgenauigkeit treffen}
}

\newglossaryentry{gl:zero-gpr}{
	name={Mittelwertfreie Regression},
	description={Mittelwertfreie Regression oder Regressionsmodelle sind für Regressionen mittles Gauß'scher Prozesse Modelle, in denen keine Mittelung der Trainings- sowie Testdaten aktiv ist. Die Mittelung ist zu null gesetzt, daher oft auch als Zero-Mean-GPR in der Literatur zu finden. Für diese Regressionsvariante werden Vorhersagen alleinig aus der Autokorelation mittels Kovarianzfunktion und Trainingsdaten getroffen}
}

\newglossaryentry{gl:poly-gpr}{
	name={Mittelwertbehaftete Regression},
	description={Mittelwertbehaftete Regression oder Regressionsmodelle für Regressionen mittels Gauß'scher Prozesse sind Modelle, in denen eine Mittelung der Trainings- sowie Testdaten aktiv ist. Diese beinflusst die Modellparametrierung und ist als regressionergänzende Stützwertbildung zu verstehen. Die Mittelung ist hier dynamisch bzw. generisch über Polynomapproximation bereitgestellt. Es können beliebige Variation an mittelwertbildenen Funktion genutzt werden}
}

\newglossaryentry{gl:refpos}{
	name={Referenzposition},
	description={Als Referenzposition ist die Startposition des Sensor-Arrays in den einzelnen Experimenten bezeichnet. Diese ist vor der Durchführung der Experiemente festgelegt und variiert zwischen den Experimenten je nach Zweck. Sie dient als Vergleichspunkt zu resultierenden Simulationsergebnissen.}
}

\newglossaryentry{gl:bounds}{
	name={Bounds},
	description={Bounds oder Limits sind englische Begriffe die Grenzen von Parametern bezeichnen}
}

\newglossaryentry{gl:encoder}{
	name={Encoder},
	description={Als Encoder ist in dieser Arbeit und in der Drehwinkelapplikation der Gebermagnet zu betrachten. Es handelt sich somit um einen mechanischen Encoder. Die Winkelinformationen werden über die Feldstärken des Magneten gewonnen}
}

\newglossaryentry{gl:Frobenius}{
	name={Frobenius},
	description={Ferdinand Georg Frobenius (26. Oktober 1849 - 3. August 1917), Namensgeber unteranderem für die Frobenius-Norm sowie der Frobenius-Matrix. Er beschäftigte sich im späten 19. Jahrhundert mit Theorien zur Gruppen und ihrer Darstellungstheorie. Zusammen mit Leopold Kronecker, Lazarus Immanuel Fuchs und Hermann Amandus Schwarz gehörte er zum engeren Kreis berühmter Berliner Mathematiker seiner Zeit}
}

\newglossaryentry{gl:kovarianz}{
	name={Kovarianz},
	description={Die Kovarianz ist ein Zusammenhangmaß zweier Zufallsvariablen, das einen monotonen Zusammenhang herstellt bei gemeinsamer Wahrscheinlichkeitsverteilung der Variablen. Sie ähnelt der Korrelation, basiert hingegen der Korrelation aber auf nicht standardisierten Daten}
}

\newglossaryentry{gl:stochastik}{
	name={Stochastik},
	description={Die Stochastik ist ein Obergriff für Mathematik und Statistik, die sich mit Wahrscheinlichkeitstheorien befasst. Es existieren eine Vielzahl von Disziplinen in der Stochastik. Diese Arbeit bewegt sich innerhalb der Bayes'schen Statistik, die sich mit bedingten Wahrscheinlichkeiten auseinandersetzt. Sie ist auch bekannt als Bayes'sche Inferenz und verknüpft sich mit der induktiven Statistik (Inferenz Statistik) über Stichproben- und Intervallschätzungen}
}

\newglossaryentry{gl:sensorbruecke}{
	name={Sensorbrücke},
	description={Als Sensorbrücke ist eine Wheatstone'sche Brückenschaltung oder Messbrücke bezeichnet. Diese besteht zumeist aus vier Widerständen}
}
