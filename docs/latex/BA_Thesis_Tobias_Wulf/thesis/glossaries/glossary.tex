% !TEX root = ./thesis.tex
% list of all glossaries in thesis
% @author Tobias Wulf

% take care that all labels are unique in the whole document

%\newglossaryentry{gl:}{
%	name={},
%	description={}
%}

% glossary entries
\newglossaryentry{gl:haw}{
	name={HAW Hamburg},
	description={Die HAW Hamburg ist die Hochschule für Angewandte Wissenschaften in Hamburg und war die ehemalige Fachhochschule am Berliner Tor}
}

\newglossaryentry{gl:ags}{
	name={Arbeitsgruppe Sensorik},
	description={Die Arbeitsgruppe Sensorik steht unter Leitung von Prof. Dr.-Ing. Karl-Ragmar Riemschneider und ist unter dem Department Informations- und Elektrotechnik Teil der Fakultät Technik un Informatik an der HAW Hambug}
}


\newglossaryentry{gl:tmr}{
	name={TMR-Effekt},
	description={Tunnel-Magnetoresistiver-Effekt}
}

\newglossaryentry{gl:gmr}{
	name={GMR-Effekt},
	description={Riesiger-Magnetoresistiver-Effekt}
}

\newglossaryentry{gl:amr}{
	name={AMR-Effekt},
	description={Anisotroper-Magnetoresistiver-Effekt}
}

\newglossaryentry{gl:sensorkopf}{
	name={Sensorkopf},
	description={Signal erzeugender Teil eines Sensor-ICs, dem eine Einheit zur weiteren Signalverarbeitung 
	nachgeschaltet 
	ist}
}