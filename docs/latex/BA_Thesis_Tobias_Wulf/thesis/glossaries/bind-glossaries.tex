% !TEX root = ./thesis.tex
% bind glossaries and define different styles, load into document preamble
% @author Tobias Wulf
%

% add unit key word to for symbol glossary, do not expand to use macros by siunitx package
\glsaddkey{unit}{\glsentrytext{\glslabel}}{\glsentryunit}{\GLsentryunit}{\glsunit}{\Glsunit}{\GLSunit}
\glssetnoexpandfield{unit}

% create all glossary entries (remember: run makeglossaries manually)
\makeglossaries           
\loadglsentries[main]{./glossaries/glossary}  % load acronym, symbol and glossarie entries
\loadglsentries[\acronymtype]{./glossaries/acronyms}
\loadglsentries[symbols]{./glossaries/symbols}


% create own symbol glossary style
\newglossarystyle{mysymbolstyle}{%
	\setglossarystyle{long3col}%
	\renewenvironment{theglossary}{\begin{longtable}{lp{.65\linewidth}>{\centering\arraybackslash}p{.11\linewidth}}}{\end{longtable}}%
	\renewcommand*{\glossaryheader}{\bfseries Symbol & \bfseries Beschreibung & \bfseries Einheit \\ \hline\endhead}%
	\renewcommand*{\glossentry}[2]{\glstarget{##1}{\glossentryname{##1}} & \glossentrydesc{##1} & \glsunit{##1}  \tabularnewline}%
}


% create own acronym style
\newglossarystyle{myacronymstyle}{%
	\setglossarystyle{long3col}%
	\renewenvironment{theglossary}{\begin{longtable}{lp{.75\linewidth}>{\centering\arraybackslash}p{.01\linewidth}}}{\end{longtable}}%
	\renewcommand*{\glossaryheader}{\bfseries Abkürzung & \bfseries Beschreibung & \\ \hline\endhead}%
	\renewcommand*{\glossentry}[2]{\bfseries\glstarget{##1}{\glossentryname{##1}} & \glossentrydesc{##1} & \tabularnewline}%
}
