% !TEX root = ./thesis.tex
% list of all symbols in thesis
% @author Tobias Wulf

% take care that all labels are unique in the whole document

%\newglossaryentry{sy:}{
%	name={\ensuremath{}},
%	description={},
%	unit={\ensuremath{\si{}}},
%	%symbol={\ensuremath{}},
%	type=symbols,
%	sort=
%}

\newglossaryentry{sy:a}{
	name={\ensuremath{a,a_{Array}}},
	description={Kantenlänge des Sensor-Arrays},
	unit={\ensuremath{\si{\milli\metre}}},
	%symbol={\ensuremath{}},
	type=symbols,
	sort=Geo1
}

\newglossaryentry{sy:dpixel}{
	name={\ensuremath{d_{Pixel}}},
	description={Abstand der einzelnen Sensor-Pixel untereinander im Sensor-Array},
	unit={\ensuremath{\si{\milli\metre}}},
	%symbol={\ensuremath{}},
	type=symbols,
	sort=Geo2
}

\newglossaryentry{sy:achsen}{
	name={\ensuremath{X,Y,Z}},
	description={Kartesische Koordinatenachsen der Applikation},
	unit={\ensuremath{\si{\milli\metre}}},
	%symbol={\ensuremath{}},
	type=symbols,
	sort=Geo3
}

\newglossaryentry{sy:rotangz}{
	name={\ensuremath{\alpha,\alpha_z}},
	description={Rotationswinkel, Winkelstellung des Gebermagneten in der $Z$-Achse},
	unit={\ensuremath{\si{\degree}}},
	%symbol={\ensuremath{}},
	type=symbols,
	sort=Geo4
}

\newglossaryentry{sy:rotangxy}{
	name={\ensuremath{\alpha_x,\alpha_y}},
	description={Verkippung, Winkelstellung des Gebermagneten in der $X$-/ $Y$-Achse},
	unit={\ensuremath{\si{\degree}}},
	%symbol={\ensuremath{}},
	type=symbols,
	sort=Geo5
}

\newglossaryentry{sy:possar}{
	name={\ensuremath{\vec{p}}},
	description={Grundpositionsvektor des Sensor-Arrays mit Bezug zur geometrischen Mitte des Arrays},
	unit={\ensuremath{\si{\milli\metre}}},
	%symbol={\ensuremath{}},
	type=symbols,
	sort=Geo6
}

\newglossaryentry{sy:posxyz}{
	name={\ensuremath{p_x,p_y,p_z}},
	description={Positionskoordinaten der Sensor-Array-Position im Koordinatensystem},
	unit={\ensuremath{\si{\milli\metre}}},
	%symbol={\ensuremath{}},
	type=symbols,
	sort=Geo7
}

\newglossaryentry{sy:xyz}{
	name={\ensuremath{x,y,z}},
	description={Koordinaten im kartesischen Koordinatensystem},
	unit={\ensuremath{\si{\milli\metre}}},
	%symbol={\ensuremath{}},
	type=symbols,
	sort=Geo8
}

\newglossaryentry{sy:meshxyz}{
	name={\ensuremath{x_{i,j},y_{i,j},z_{i,j}}},
	description={Meshgrid bzw. Sensor-Pixel-Koordinate im kartesischen Koordinatensystem},
	unit={\ensuremath{\si{\milli\metre}}},
	%symbol={\ensuremath{}},
	type=symbols,
	sort=Geo9
}

\newglossaryentry{sy:a2}{
	name={\ensuremath{A_{Array}}},
	description={Fläche des Sensor-Array resultieren aus der Kantenlänge für ein quadratisches Sensor-Arrays},
	unit={\ensuremath{\si{}}},
	%symbol={\ensuremath{}},
	type=symbols,
	sort=Geo10
}

\newglossaryentry{sy:dz}{
	name={\ensuremath{d_z}},
	description={$Z$-Abstand des Sensor-Arrays zur Magnetoberfläche},
	unit={\ensuremath{\si{\milli\metre}}},
	%symbol={\ensuremath{}},
	type=symbols,
	sort=Geo11
}

\newglossaryentry{sy:rmag}{
	name={\ensuremath{r_{mag}}},
	description={Radius des Gebermagneten als Kugelmagnet},
	unit={\ensuremath{\si{\milli\metre}}},
	%symbol={\ensuremath{}},
	type=symbols,
	sort=Geo12
}

\newglossaryentry{sy:feldstaerke}{
	name={\ensuremath{\vec{H}}},
	description={Feldstärke des Gebermagnetfeldes},
	unit={\ensuremath{\si{\kilo\ampere\per\metre}}},
	%symbol={\ensuremath{}},
	type=symbols,
	sort=Feld1
}

\newglossaryentry{sy:feldkomp}{
	name={\ensuremath{H_x,H_y,H_z}},
	description={Feldstärkenkomponenten des Gebermagnetfeldes},
	unit={\ensuremath{\si{\kilo\ampere\per\metre}}},
	%symbol={\ensuremath{}},
	type=symbols,
	sort=Feld2
}

\newglossaryentry{sy:feldbetrag}{
	name={\ensuremath{|\vec{H}|,|H|}},
	description={Betragsfeldstärke des Gebermagnetfeldes},
	unit={\ensuremath{\si{\kilo\ampere\per\metre}}},
	%symbol={\ensuremath{}},
	type=symbols,
	sort=Feld3
}

\newglossaryentry{sy:rvec}{
	name={\ensuremath{\vec{r}}},
	description={Positionsvektor einzelner Meshgrid- bzw. Pixel-Positionen in der Dipol-Feldberechnung},
	unit={\ensuremath{\si{\milli\metre}}},
	%symbol={\ensuremath{}},
	type=symbols,
	sort=Feld4
}

\newglossaryentry{sy:runit}{
	name={\ensuremath{\hat{r}}},
	description={Einheitsvektor von $\vec{r}$},
	unit={\ensuremath{\si{}}},
	%symbol={\ensuremath{}},
	type=symbols,
	sort=Feld5
}

\newglossaryentry{sy:rabs}{
	name={\ensuremath{|\vec{r}|}},
	description={Zeigerlänge oder Betrag des Positionsvektors $\vec{r}$},
	unit={\ensuremath{\si{\milli\metre}}},
	%symbol={\ensuremath{}},
	type=symbols,
	sort=Feld6
}

\newglossaryentry{sy:magmom}{
	name={\ensuremath{\vec{m}}},
	description={Magnetisches Moment des magnetischen Dipols},
	unit={\ensuremath{\si{\kilo\ampere\square\metre}}},
	%symbol={\ensuremath{}},
	type=symbols,
	sort=Feld7
}

\newglossaryentry{sy:magmomxyz}{
	name={\ensuremath{m_x,m_y,m_z}},
	description={Komponenten des magnetischen Dipol-Momentes},
	unit={\ensuremath{\si{\kilo\ampere\square\metre}}},
	%symbol={\ensuremath{}},
	type=symbols,
	sort=Feld8
}

\newglossaryentry{sy:rotmat}{
	name={\ensuremath{R_x,R_y,R_z}},
	description={Rotationsmatrizen zur Drehung von Vektoren im kartesischen Koordinatensystem},
	unit={\ensuremath{\si{}}},
	%symbol={\ensuremath{}},
	type=symbols,
	sort=Feld9
}

\newglossaryentry{sy:ruhelage}{
	name={\ensuremath{\vec{r}_0,\vec{m}_0,\vec{H}_0}},
	description={Ruhelage des Gebermagneten bei definierten Magnetradius, $Z$-Abstand zum Sensor-Array, Ausrichtung des Magneten und resultierender Feldstärke},
	unit={\ensuremath{\si{}}},
	%symbol={\ensuremath{}},
	type=symbols,
	sort=Feld10
}

\newglossaryentry{sy:hmag}{
	name={\ensuremath{H_{mag}}},
	description={Aufzuprägende Betragsfeldstärke für eine normierte Dipol-Feldberechnung},
	unit={\ensuremath{\si{\kilo\ampere\per\metre}}},
	%symbol={\ensuremath{}},
	type=symbols,
	sort=Feld11
}

\newglossaryentry{sy:hnorm}{
	name={\ensuremath{\vec{H}_{Norm}}},
	description={Normiertes Dipol-Magnetfeld mit aufgeprägter Betragsfeldstärke für die Ruhelage},
	unit={\ensuremath{\si{\kilo\ampere\per\metre}}},
	%symbol={\ensuremath{}},
	type=symbols,
	sort=Feld12
}

\newglossaryentry{sy:vcos}{
	name={\ensuremath{V_{cos}}},
	description={Ausgangsspannung der Sensor-Cosinus-Brücke},
	unit={\ensuremath{\si{\volt}}},
	%symbol={\ensuremath{}},
	type=symbols,
	sort=Sensor1
}

\newglossaryentry{sy:vsin}{
	name={\ensuremath{V_{sin}}},
	description={Ausgangsspannung der Sensor-Sinus-Brücke},
	unit={\ensuremath{\si{\volt}}},
	%symbol={\ensuremath{}},
	type=symbols,
	sort=Sensor2
}

\newglossaryentry{sy:vout}{
	name={\ensuremath{V_{out}}},
	description={Sensorausgangsspannung allgemein},
	unit={\ensuremath{\si{\volt}}},
	%symbol={\ensuremath{}},
	type=symbols,
	sort=Sensor3
}

\newglossaryentry{sy:relresist}{
	name={\ensuremath{\Delta R/R}},
	description={Relative Widerstandsänderung},
	unit={\ensuremath{\si{\percent}}},
	%symbol={\ensuremath{}},
	type=symbols,
	sort=Sensor4
}

\newglossaryentry{sy:vcc}{
	name={\ensuremath{V_{CC}}},
	description={Versorgungsspannung des Sensor-Arrays},
	unit={\ensuremath{\si{\volt}}},
	%symbol={\ensuremath{}},
	type=symbols,
	sort=Sensor5
}

\newglossaryentry{sy:voff}{
	name={\ensuremath{V_{Offset}, V_{off}}},
	description={Offset-Spannung des Sensor-Arrays},
	unit={\ensuremath{\si{\volt}}},
	%symbol={\ensuremath{}},
	type=symbols,
	sort=Sensor6
}

\newglossaryentry{sy:gain}{
	name={\ensuremath{Gain}},
	description={Verstärkungsfaktor in der Signalaufbereitung im Sensor-Array},
	unit={\ensuremath{\si{}}},
	%symbol={\ensuremath{}},
	type=symbols,
	sort=Sensor6
}

\newglossaryentry{sy:vnorm}{
	name={\ensuremath{V_{Norm},V_{norm}}},
	description={Normfaktor in der Rückgewinnung von Ausgagangsspannungen aus Kennfeldern},
	unit={\ensuremath{\si{\milli\volt}}},
	%symbol={\ensuremath{}},
	type=symbols,
	sort=Sensor7
}

\newglossaryentry{sy:winkelmessung}{
	name={\ensuremath{\mathbf{A},\mathbf{B}}},
	description={Winkelmessung in Bezug auf die vektorielle Kreisdarstellung der Applikation},
	unit={\ensuremath{\si{}}},
	%symbol={\ensuremath{}},
	type=symbols,
	sort=Kreis1
}

\newglossaryentry{sy:kreisxy}{
	name={\ensuremath{a_x,a_y,b_x,b_y}},
	description={Vektorielle $X$-/ $Y$-Komponenten in der Kreisdarstellung der Applikation für die Winkelmessung},
	unit={\ensuremath{\si{}}},
	%symbol={\ensuremath{}},
	type=symbols,
	sort=Kreis2
}


\newglossaryentry{sy:radius}{
	name={\ensuremath{r}},
	description={Kreisbahnradius in der Kreisdarstellung der Winkelmessung},
	unit={\ensuremath{\si{}}},
	%symbol={\ensuremath{}},
	type=symbols,
	sort=Kreis3
}

\newglossaryentry{sy:pi}{
	name={\ensuremath{\pi}},
	description={Kreiszahl $\approx 3,1416$},
	unit={\ensuremath{\si{}}},
	%symbol={\ensuremath{}},
	type=symbols,
	sort=Kreis4
}

\newglossaryentry{sy:v2norm}{
	name={\ensuremath{\|\cdot\|_2}},
	description={Vektor-2-Norm für eindimensionale Vektoren},
	unit={\ensuremath{\si{}}},
	%symbol={\ensuremath{}},
	type=symbols,
	sort=Norm1
}

\newglossaryentry{sy:v2norm2}{
	name={\ensuremath{\|\cdot\|_2^2}},
	description={Quadrat der Vektor-2-Norm},
	unit={\ensuremath{\si{}}},
	%symbol={\ensuremath{}},
	type=symbols,
	sort=Norm2
}

\newglossaryentry{sy:euklid}{
	name={\ensuremath{d_E\langle\rangle}},
	description={Euklidische Abstandsfunktion basierend auf der Vektor-2-Norm},
	unit={\ensuremath{\si{}}},
	%symbol={\ensuremath{}},
	type=symbols,
	sort=Norm3
}

\newglossaryentry{sy:euklid2}{
	name={\ensuremath{d_E^2\langle\rangle}},
	description={Quadrat der euklidischen Abstandsfunktion},
	unit={\ensuremath{\si{}}},
	%symbol={\ensuremath{}},
	type=symbols,
	sort=Norm4
}

\newglossaryentry{sy:frobnorm}{
	name={\ensuremath{\|\cdot\|_F}},
	description={Frobenius-Norm als Matrix-Norm},
	unit={\ensuremath{\si{}}},
	%symbol={\ensuremath{}},
	type=symbols,
	sort=Norm5
}

\newglossaryentry{sy:frobnorm2}{
	name={\ensuremath{\|\cdot\|_F^2}},
	description={Quadrat der Frobenius-Norm},
	unit={\ensuremath{\si{}}},
	%symbol={\ensuremath{}},
	type=symbols,
	sort=Norm6
}

\newglossaryentry{sy:dfrob}{
	name={\ensuremath{d_F\langle\rangle}},
	description={Abstandsfunktion basierend auf der Frobenius-Norm},
	unit={\ensuremath{\si{}}},
	%symbol={\ensuremath{}},
	type=symbols,
	sort=Norm7
}

\newglossaryentry{sy:dfrob2}{
	name={\ensuremath{d_F^2\langle\rangle}},
	description={Quadrat der Abstandsfunktion basierend auf der Frobenius-Norm},
	unit={\ensuremath{\si{}}},
	%symbol={\ensuremath{}},
	type=symbols,
	sort=Norm8
}

\newglossaryentry{sy:maps}{
	name={\ensuremath{\mapsto}},
	description={Bezieht sich auf, Referenz zu},
	unit={\ensuremath{\si{}}},
	%symbol={\ensuremath{}},
	type=symbols,
	sort=Ops1
}

\newglossaryentry{sy:log10}{
	name={\ensuremath{\log_{10}}},
	description={Logarithmus zur Basis $10$},
	unit={\ensuremath{\si{}}},
	%symbol={\ensuremath{}},
	type=symbols,
	sort=Ops2
}

\newglossaryentry{sy:log}{
	name={\ensuremath{\log}},
	description={Logarithmus Naturalis},
	unit={\ensuremath{\si{}}},
	%symbol={\ensuremath{}},
	type=symbols,
	sort=Ops3
}

\newglossaryentry{sy:atan2}{
	name={\ensuremath{\textrm{arctan2}}},
	description={Arkustangens-Funktion mit erweiterten Winkelintervall},
	unit={\ensuremath{\si{}}},
	%symbol={\ensuremath{}},
	type=symbols,
	sort=Ops4
}

\newglossaryentry{sy:cosdaten}{
	name={\ensuremath{\mathbf{A_x},X_{cos,i},X_{cos*}}},
	description={Einzelne Messwertmatrix für die Cosinus-Funktion des Sensor-Arrays mit $*$ als Bezug auf Testdaten},
	unit={\ensuremath{\si{}}},
	%symbol={\ensuremath{}},
	type=symbols,
	sort=daten1
}

\newglossaryentry{sy:sindaten}{
	name={\ensuremath{\mathbf{A_y},X_{sin,i},X_{sin*}}},
	description={Einzelne Messwertmatrix für die Sinus-Funktion des Sensor-Arrays mit $*$ als Bezug auf Testdaten},
	unit={\ensuremath{\si{}}},
	%symbol={\ensuremath{}},
	type=symbols,
	sort=daten2
}

\newglossaryentry{sy:traindata}{
	name={\ensuremath{X}},
	description={Vollständiger Trainingsdatensatz bestehend aus mehren Matrizen aus Cosinus- und Sinus-Messwertmatrizen},
	unit={\ensuremath{\si{}}},
	%symbol={\ensuremath{}},
	type=symbols,
	sort=daten3
}

% hier weiter machen gpr anhang daten formate

\newglossaryentry{sy:fmod}{
	name={\ensuremath{f_m}},
	description={Modulationsfrequenz in der Erstellung von Kennfeldern},
	unit={\ensuremath{\si{\hertz}}},
	%symbol={\ensuremath{}},
	type=symbols,
	sort=Kennfeld1
}

\newglossaryentry{sy:fcar}{
	name={\ensuremath{f_c}},
	description={Trägerfrequenz in der Erstellung von Kennfeldern},
	unit={\ensuremath{\si{\hertz}}},
	%symbol={\ensuremath{}},
	type=symbols,
	sort=Kennfeld2
}

\newglossaryentry{sy:phase}{
	name={\ensuremath{\phi}},
	description={Phasenwinkel der Stimulanz für die Kennfelderzeugung},
	unit={\ensuremath{\si{\radian}}},
	%symbol={\ensuremath{}},
	type=symbols,
	sort=Kennfeld3
}