% !TEX root = ./thesis.tex
% list of all symbols in thesis
% @author Tobias Wulf

% take care that all labels are unique in the whole document

%\newglossaryentry{sy:}{
%	name={\ensuremath{}},
%	description={},
%	unit={\ensuremath{\si{}}},
%	%symbol={\ensuremath{}},
%	type=symbols,
%	sort=
%}

\newglossaryentry{sy:a}{
	name={\ensuremath{a,a_{Array}}},
	description={Kantenlänge des Sensor-Arrays.},
	unit={\ensuremath{\si{\milli\metre}}},
	%symbol={\ensuremath{}},
	type=symbols,
	sort=Geo1
}

\newglossaryentry{sy:dpixel}{
	name={\ensuremath{d_{Pixel}}},
	description={Abstand der einzelnen Sensor-Pixel untereinander im Sensor-Array.},
	unit={\ensuremath{\si{\milli\metre}}},
	%symbol={\ensuremath{}},
	type=symbols,
	sort=Geo2
}

\newglossaryentry{sy:achsen}{
	name={\ensuremath{X,Y,Z}},
	description={Kartesische Koordinatenachsen der Applikation.},
	unit={\ensuremath{\si{\milli\metre}}},
	%symbol={\ensuremath{}},
	type=symbols,
	sort=Geo3
}

\newglossaryentry{sy:rotangz}{
	name={\ensuremath{\alpha,\alpha_z}},
	description={Rotationswinkel, Winkelstellung des Gebermagneten in der $Z$-Achse.},
	unit={\ensuremath{\si{\degree}}},
	%symbol={\ensuremath{}},
	type=symbols,
	sort=Geo4
}

\newglossaryentry{sy:rotangxy}{
	name={\ensuremath{\alpha_x,\alpha_y}},
	description={Verkippung, Winkelstellung des Gebermagneten in der $X$-/ $Y$-Achse.},
	unit={\ensuremath{\si{\degree}}},
	%symbol={\ensuremath{}},
	type=symbols,
	sort=Geo5
}

\newglossaryentry{sy:possar}{
	name={\ensuremath{\vec{p}}},
	description={Grundpositionsvektor des Sensor-Arrays mit Bezug zur geometrischen Mitte des Arrays.},
	unit={\ensuremath{\si{\milli\metre}}},
	%symbol={\ensuremath{}},
	type=symbols,
	sort=Geo6
}

\newglossaryentry{sy:posxyz}{
	name={\ensuremath{p_x,p_y,p_z}},
	description={Positionskoordinaten der Sensor-Array-Position im Koordinatensystem.},
	unit={\ensuremath{\si{\milli\metre}}},
	%symbol={\ensuremath{}},
	type=symbols,
	sort=Geo7
}

\newglossaryentry{sy:xyz}{
	name={\ensuremath{x,y,z}},
	description={Koordinaten im kartesischen Koordinatensystem.},
	unit={\ensuremath{\si{\milli\metre}}},
	%symbol={\ensuremath{}},
	type=symbols,
	sort=Geo8
}

\newglossaryentry{sy:meshxyz}{
	name={\ensuremath{x_{i,j},y_{i,j},z_{i,j}}},
	description={Meshgrid bzw. Sensor-Pixel-Koordinate im kartesischen Koordinatensystem.},
	unit={\ensuremath{\si{\milli\metre}}},
	%symbol={\ensuremath{}},
	type=symbols,
	sort=Geo9
}

\newglossaryentry{sy:a2}{
	name={\ensuremath{A_{Array}}},
	description={Fläche des Sensor-Array resultieren aus der Kantenlänge für ein quadratisches Sensor-Arrays.},
	unit={\ensuremath{\si{}}},
	%symbol={\ensuremath{}},
	type=symbols,
	sort=Geo10
}

\newglossaryentry{sy:dz}{
	name={\ensuremath{d_z}},
	description={$Z$-Abstand des Sensor-Arrays zur Magnetoberfläche.},
	unit={\ensuremath{\si{\milli\metre}}},
	%symbol={\ensuremath{}},
	type=symbols,
	sort=Geo11
}

\newglossaryentry{sy:rmag}{
	name={\ensuremath{r_{mag}}},
	description={Radius des Gebermagneten als Kugelmagnet.},
	unit={\ensuremath{\si{\milli\metre}}},
	%symbol={\ensuremath{}},
	type=symbols,
	sort=Geo12
}

\newglossaryentry{sy:feldstaerke}{
	name={\ensuremath{\vec{H}}},
	description={Feldstärke des Gebermagnetfeldes.},
	unit={\ensuremath{\si{\kilo\ampere\per\metre}}},
	%symbol={\ensuremath{}},
	type=symbols,
	sort=Feld1
}

\newglossaryentry{sy:feldkomp}{
	name={\ensuremath{H_x,H_y,H_z}},
	description={Feldstärkenkomponenten des Gebermagnetfeldes.},
	unit={\ensuremath{\si{\kilo\ampere\per\metre}}},
	%symbol={\ensuremath{}},
	type=symbols,
	sort=Feld2
}

\newglossaryentry{sy:feldbetrag}{
	name={\ensuremath{|\vec{H}|,|H|}},
	description={Betragsfeldstärke des Gebermagnetfeldes.},
	unit={\ensuremath{\si{\kilo\ampere\per\metre}}},
	%symbol={\ensuremath{}},
	type=symbols,
	sort=Feld3
}

\newglossaryentry{sy:rvec}{
	name={\ensuremath{\vec{r}}},
	description={Positionsvektor einzelner Meshgrid- bzw. Pixel-Positionen in der Dipol-Feldberechnung.},
	unit={\ensuremath{\si{\milli\metre}}},
	%symbol={\ensuremath{}},
	type=symbols,
	sort=Feld4
}

\newglossaryentry{sy:runit}{
	name={\ensuremath{\hat{r}}},
	description={Einheitsvektor von $\vec{r}$.},
	unit={\ensuremath{\si{}}},
	%symbol={\ensuremath{}},
	type=symbols,
	sort=Feld5
}

\newglossaryentry{sy:rabs}{
	name={\ensuremath{|\vec{r}|}},
	description={Zeigerlänge oder Betrag des Positionsvektors $\vec{r}$.},
	unit={\ensuremath{\si{\milli\metre}}},
	%symbol={\ensuremath{}},
	type=symbols,
	sort=Feld6
}

\newglossaryentry{sy:magmom}{
	name={\ensuremath{\vec{m}}},
	description={Magnetisches Moment des magnetischen Dipols.},
	unit={\ensuremath{\si{\kilo\ampere\square\metre}}},
	%symbol={\ensuremath{}},
	type=symbols,
	sort=Feld7
}

\newglossaryentry{sy:magmomxyz}{
	name={\ensuremath{m_x,m_y,m_z}},
	description={Komponenten des magnetischen Dipol-Momentes.},
	unit={\ensuremath{\si{\kilo\ampere\square\metre}}},
	%symbol={\ensuremath{}},
	type=symbols,
	sort=Feld8
}

\newglossaryentry{sy:rotmat}{
	name={\ensuremath{R_x,R_y,R_z}},
	description={Rotationsmatrizen zur Drehung von Vektoren im kartesischen Koordinatensystem.},
	unit={\ensuremath{\si{}}},
	%symbol={\ensuremath{}},
	type=symbols,
	sort=Feld9
}

\newglossaryentry{sy:ruhelage}{
	name={\ensuremath{\vec{r}_0,\vec{m}_0,\vec{H}_0}},
	description={Ruhelage des Gebermagneten bei definierten Magnetradius, $Z$-Abstand zum Sensor-Array, Ausrichtung des Magneten und resultierender Feldstärke.},
	unit={\ensuremath{\si{}}},
	%symbol={\ensuremath{}},
	type=symbols,
	sort=Feld10
}

\newglossaryentry{sy:hmag}{
	name={\ensuremath{H_{mag}}},
	description={Aufzuprägende Betragsfeldstärke für eine normierte Dipol-Feldberechnung.},
	unit={\ensuremath{\si{\kilo\ampere\per\metre}}},
	%symbol={\ensuremath{}},
	type=symbols,
	sort=Feld11
}

\newglossaryentry{sy:hnorm}{
	name={\ensuremath{\vec{H}_{Norm}}},
	description={Normiertes Dipol-Magnetfeld mit aufgeprägter Betragsfeldstärke für die Ruhelage.},
	unit={\ensuremath{\si{\kilo\ampere\per\metre}}},
	%symbol={\ensuremath{}},
	type=symbols,
	sort=Feld12
}

\newglossaryentry{sy:vcos}{
	name={\ensuremath{V_{cos}}},
	description={Ausgangsspannung der Sensor-Cosinus-Brücke.},
	unit={\ensuremath{\si{\volt}}},
	%symbol={\ensuremath{}},
	type=symbols,
	sort=Sensor1
}

\newglossaryentry{sy:vsin}{
	name={\ensuremath{V_{sin}}},
	description={Ausgangsspannung der Sensor-Sinus-Brücke.},
	unit={\ensuremath{\si{\volt}}},
	%symbol={\ensuremath{}},
	type=symbols,
	sort=Sensor2
}

\newglossaryentry{sy:vout}{
	name={\ensuremath{V_{out}}},
	description={Sensorausgangsspannung allgemein.},
	unit={\ensuremath{\si{\volt}}},
	%symbol={\ensuremath{}},
	type=symbols,
	sort=Sensor3
}

\newglossaryentry{sy:relresist}{
	name={\ensuremath{\Delta R/R}},
	description={Relative Widerstandsänderung.},
	unit={\ensuremath{\si{\percent}}},
	%symbol={\ensuremath{}},
	type=symbols,
	sort=Sensor4
}

\newglossaryentry{sy:vcc}{
	name={\ensuremath{V_{CC}}},
	description={Versorgungsspannung des Sensor-Arrays.},
	unit={\ensuremath{\si{\volt}}},
	%symbol={\ensuremath{}},
	type=symbols,
	sort=Sensor5
}

\newglossaryentry{sy:voff}{
	name={\ensuremath{V_{Offset}, V_{off}}},
	description={Offset-Spannung des Sensor-Arrays.},
	unit={\ensuremath{\si{\volt}}},
	%symbol={\ensuremath{}},
	type=symbols,
	sort=Sensor6
}

\newglossaryentry{sy:gain}{
	name={\ensuremath{Gain}},
	description={Verstärkungsfaktor in der Signalaufbereitung im Sensor-Array.},
	unit={\ensuremath{\si{}}},
	%symbol={\ensuremath{}},
	type=symbols,
	sort=Sensor6
}

\newglossaryentry{sy:vnorm}{
	name={\ensuremath{V_{Norm},V_{norm}}},
	description={Normfaktor in der Rückgewinnung von Ausgagangsspannungen aus Kennfeldern.},
	unit={\ensuremath{\si{\milli\volt}}},
	%symbol={\ensuremath{}},
	type=symbols,
	sort=Sensor7
}

\newglossaryentry{sy:winkelmessung}{
	name={\ensuremath{\mathbf{A},\mathbf{B}}},
	description={Winkelmessung in Bezug auf die vektorielle Kreisdarstellung der Applikation.},
	unit={\ensuremath{\si{}}},
	%symbol={\ensuremath{}},
	type=symbols,
	sort=Kreis1
}

\newglossaryentry{sy:kreisxy}{
	name={\ensuremath{a_x,a_y,b_x,b_y}},
	description={Vektorielle $X$-/ $Y$-Komponenten in der Kreisdarstellung der Applikation für die Winkelmessung.},
	unit={\ensuremath{\si{}}},
	%symbol={\ensuremath{}},
	type=symbols,
	sort=Kreis2
}


\newglossaryentry{sy:radius}{
	name={\ensuremath{r}},
	description={Kreisbahnradius in der Kreisdarstellung der Winkelmessung.},
	unit={\ensuremath{\si{}}},
	%symbol={\ensuremath{}},
	type=symbols,
	sort=Kreis3
}

\newglossaryentry{sy:pi}{
	name={\ensuremath{\pi}},
	description={Kreiszahl $\approx 3,1416$.},
	unit={\ensuremath{\si{}}},
	%symbol={\ensuremath{}},
	type=symbols,
	sort=Kreis4
}

\newglossaryentry{sy:v2norm}{
	name={\ensuremath{\|\cdot\|_2}},
	description={Vektor-2-Norm für eindimensionale Vektoren.},
	unit={\ensuremath{\si{}}},
	%symbol={\ensuremath{}},
	type=symbols,
	sort=Norm1
}

\newglossaryentry{sy:v2norm2}{
	name={\ensuremath{\|\cdot\|_2^2}},
	description={Quadrat der Vektor-2-Norm.},
	unit={\ensuremath{\si{}}},
	%symbol={\ensuremath{}},
	type=symbols,
	sort=Norm2
}

\newglossaryentry{sy:euklid}{
	name={\ensuremath{d_E\langle\rangle}},
	description={Euklidische Abstandsfunktion basierend auf der Vektor-2-Norm.},
	unit={\ensuremath{\si{}}},
	%symbol={\ensuremath{}},
	type=symbols,
	sort=Norm3
}

\newglossaryentry{sy:euklid2}{
	name={\ensuremath{d_E^2\langle\rangle}},
	description={Quadrat der euklidischen Abstandsfunktion.},
	unit={\ensuremath{\si{}}},
	%symbol={\ensuremath{}},
	type=symbols,
	sort=Norm4
}

\newglossaryentry{sy:frobnorm}{
	name={\ensuremath{\|\cdot\|_F}},
	description={Frobenius-Norm als Matrix-Norm.},
	unit={\ensuremath{\si{}}},
	%symbol={\ensuremath{}},
	type=symbols,
	sort=Norm5
}

\newglossaryentry{sy:frobnorm2}{
	name={\ensuremath{\|\cdot\|_F^2}},
	description={Quadrat der Frobenius-Norm.},
	unit={\ensuremath{\si{}}},
	%symbol={\ensuremath{}},
	type=symbols,
	sort=Norm6
}

\newglossaryentry{sy:dfrob}{
	name={\ensuremath{d_F\langle\rangle}},
	description={Abstandsfunktion basierend auf der Frobenius-Norm.},
	unit={\ensuremath{\si{}}},
	%symbol={\ensuremath{}},
	type=symbols,
	sort=Norm7
}

\newglossaryentry{sy:dfrob2}{
	name={\ensuremath{d_F^2\langle\rangle}},
	description={Quadrat der Abstandsfunktion basierend auf der Frobenius-Norm.},
	unit={\ensuremath{\si{}}},
	%symbol={\ensuremath{}},
	type=symbols,
	sort=Norm8
}

\newglossaryentry{sy:maps}{
	name={\ensuremath{\mapsto}},
	description={Bezieht sich auf, Referenz zu.},
	unit={\ensuremath{\si{}}},
	%symbol={\ensuremath{}},
	type=symbols,
	sort=Ops1
}

\newglossaryentry{sy:log10}{
	name={\ensuremath{\log_{10}}},
	description={Logarithmus zur Basis $10$.},
	unit={\ensuremath{\si{}}},
	%symbol={\ensuremath{}},
	type=symbols,
	sort=Ops2
}

\newglossaryentry{sy:log}{
	name={\ensuremath{\log}},
	description={Logarithmus Naturalis.},
	unit={\ensuremath{\si{}}},
	%symbol={\ensuremath{}},
	type=symbols,
	sort=Ops3
}

\newglossaryentry{sy:atan2}{
	name={\ensuremath{\textrm{arctan2}}},
	description={Arkustangens-Funktion mit erweiterten Winkelintervall.},
	unit={\ensuremath{\si{}}},
	%symbol={\ensuremath{}},
	type=symbols,
	sort=Ops4
}

\newglossaryentry{sy:lgs}{
	name={\ensuremath{\backslash}},
	description={Kurzschreibformoperator für die Lösung linearer Gleichungssystemen.},
	unit={\ensuremath{\si{}}},
	%symbol={\ensuremath{}},
	type=symbols,
	sort=Ops5
}

\newglossaryentry{sy:invmat}{
	name={\ensuremath{A^{-1}}},
	description={Inverse Matrix einer Matrix $A$.},
	unit={\ensuremath{\si{}}},
	%symbol={\ensuremath{}},
	type=symbols,
	sort=Ops6
}

\newglossaryentry{sy:transmat}{
	name={\ensuremath{A^T}},
	description={Transponierte Matrix einer Matrix $A$.},
	unit={\ensuremath{\si{}}},
	%symbol={\ensuremath{}},
	type=symbols,
	sort=Ops7
}

\newglossaryentry{sy:argmin}{
	name={\ensuremath{\arg\min}},
	description={Argumentenabhängige Suche nach Minum Extremwerten einer beliebigen Funktion oder Parameter.},
	unit={\ensuremath{\si{}}},
	%symbol={\ensuremath{}},
	type=symbols,
	sort=Ops8
}

\newglossaryentry{sy:cosdaten}{
	name={\ensuremath{\mathbf{A_x},X_{cos,i},X_{cos*}}},
	description={Einzelne Messwertmatrix für die Cosinus-Funktion des Sensor-Arrays mit $*$ als Bezug auf Testdaten.},
	unit={\ensuremath{\si{}}},
	%symbol={\ensuremath{}},
	type=symbols,
	sort=daten1
}

\newglossaryentry{sy:sindaten}{
	name={\ensuremath{\mathbf{A_y},X_{sin,i},X_{sin*}}},
	description={Einzelne Messwertmatrix für die Sinus-Funktion des Sensor-Arrays mit $*$ als Bezug auf Testdaten.},
	unit={\ensuremath{\si{}}},
	%symbol={\ensuremath{}},
	type=symbols,
	sort=daten2
}

\newglossaryentry{sy:traindata}{
	name={\ensuremath{X}},
	description={Vollständiger Trainingsdatensatz des Regressionsmodells.},
	unit={\ensuremath{\si{}}},
	%symbol={\ensuremath{}},
	type=symbols,
	sort=daten3
}

\newglossaryentry{sy:traindata2}{
	name={\ensuremath{X_i}},
	description={Teiltraingsdatensatz des Regressionmodells.},
	unit={\ensuremath{\si{}}},
	%symbol={\ensuremath{}},
	type=symbols,
	sort=daten4
}

\newglossaryentry{sy:refang}{
	name={\ensuremath{\alpha_i, \alpha_{Ref}}},
	description={Referenzwinkel der Teiltrainingsdatensätze.},
	unit={\ensuremath{\si{\degree}}},
	%symbol={\ensuremath{}},
	type=symbols,
	sort=daten411
}

\newglossaryentry{sy:nref}{
	name={\ensuremath{N_{ref}}},
	description={Referenzwinkelanzahl},
	unit={\ensuremath{\si{}}},
	%symbol={\ensuremath{}},
	type=symbols,
	sort=daten412
}

\newglossaryentry{sy:testdata}{
	name={\ensuremath{X_*}},
	description={Testdatensatz. Der Datensatz ist nicht Teil der Trainingsdaten.},
	unit={\ensuremath{\si{}}},
	%symbol={\ensuremath{}},
	type=symbols,
	sort=daten5
}

\newglossaryentry{sy:simang}{
	name={\ensuremath{\alpha_*}},
	description={Simulationswinkel des Testdatensatzes.},
	unit={\ensuremath{\si{\degree}}},
	%symbol={\ensuremath{}},
	type=symbols,
	sort=daten511
}

\newglossaryentry{sy:targets}{
	name={\ensuremath{y_{cos}, y_{sin}}},
	description={Regressionsziele basierend auf Referenzwinkeln.},
	unit={\ensuremath{\si{}}},
	%symbol={\ensuremath{}},
	type=symbols,
	sort=daten6
}

\newglossaryentry{sy:kfun}{
	name={\ensuremath{k()}},
	description={Kovarianzfunktion auch als Kernel-Funktion bezeichnet.},
	unit={\ensuremath{\si{}}},
	%symbol={\ensuremath{}},
	type=symbols,
	sort=daten7
}

\newglossaryentry{sy:kfunp1}{
	name={\ensuremath{a,b}},
	description={Kernel-Parameter der Fractional-Covariance-Funktion.},
	unit={\ensuremath{\si{}}},
	%symbol={\ensuremath{}},
	type=symbols,
	sort=daten8
}

\newglossaryentry{sy:kfunp2}{
	name={\ensuremath{\sigma_f^2}},
	description={Höhenskalierung der Kovarianzfunktion.},
	unit={\ensuremath{\si{}}},
	%symbol={\ensuremath{}},
	type=symbols,
	sort=daten9
}

\newglossaryentry{sy:kfunp3}{
	name={\ensuremath{\sigma_l}},
	description={Längenskalierung der Kovarianzfunktion.},
	unit={\ensuremath{\si{}}},
	%symbol={\ensuremath{}},
	type=symbols,
	sort=daten911
}

\newglossaryentry{sy:kfunp4}{
	name={\ensuremath{\theta}},
	description={Abstrahierter Kernel-Parametervektor für die Skalierungsparameter der Kovarianzfunktion.},
	unit={\ensuremath{\si{}}},
	%symbol={\ensuremath{}},
	type=symbols,
	sort=daten912
}

\newglossaryentry{sy:kmat}{
	name={\ensuremath{K}},
	description={Kovarianzmatrix für Noise-Free-Observation.},
	unit={\ensuremath{\si{}}},
	%symbol={\ensuremath{}},
	type=symbols,
	sort=daten913
}

\newglossaryentry{sy:emat}{
	name={\ensuremath{I}},
	description={Einheitsmatrix.},
	unit={\ensuremath{\si{}}},
	%symbol={\ensuremath{}},
	type=symbols,
	sort=daten914
}

\newglossaryentry{sy:kfunp5}{
	name={\ensuremath{\sigma_n^2}},
	description={Rauschniveau (engl. Noise Level).},
	unit={\ensuremath{\si{}}},
	%symbol={\ensuremath{}},
	type=symbols,
	sort=daten915
}

\newglossaryentry{sy:kmat2}{
	name={\ensuremath{K_y}},
	description={Kovarianzmatrix mit aufaddiertem Rauschniveau für Noisy-Observation.},
	unit={\ensuremath{\si{}}},
	%symbol={\ensuremath{}},
	type=symbols,
	sort=daten916
}

\newglossaryentry{sy:lmat}{
	name={\ensuremath{L}},
	description={Untere Dreiecksmatrix als Resultat der Cholesky-Zerlegung.},
	unit={\ensuremath{\si{}}},
	%symbol={\ensuremath{}},
	type=symbols,
	sort=daten917
}

\newglossaryentry{sy:logkmat}{
	name={\ensuremath{\log|K_y|}},
	description={Logarithmische Determinate der Kovarainzmatrix.},
	unit={\ensuremath{\si{}}},
	%symbol={\ensuremath{}},
	type=symbols,
	sort=daten918
}

\newglossaryentry{sy:hfun}{
	name={\ensuremath{h_{cos}(),h_{sin}()}},
	description={Basisfunktion zur Bildung von Polynomen basierend auf Trainings-/ Testdatensätzen.},
	unit={\ensuremath{\si{}}},
	%symbol={\ensuremath{}},
	type=symbols,
	sort=daten919
}

\newglossaryentry{sy:hmat}{
	name={\ensuremath{H_{cos},H_{sin}}},
	description={Aus Basisfunktionen resultierende Polynommatrizen zur Mittelwertbildung basierend auf Trainingsdaten.},
	unit={\ensuremath{\si{}}},
	%symbol={\ensuremath{}},
	type=symbols,
	sort=daten920
}

\newglossaryentry{sy:beta}{
	name={\ensuremath{\beta_{cos},\beta_{sin}}},
	description={Polynomkoeffizienten zur Mittelwertbildung über Basisfunktionen oder Polynommatrizen.},
	unit={\ensuremath{\si{}}},
	%symbol={\ensuremath{}},
	type=symbols,
	sort=daten921
}

\newglossaryentry{sy:meangpr}{
	name={\ensuremath{m_{cos}(),m_{sin}()}},
	description={Mittelwertfunktionen resultierend aus Basisfunktionen und Polynomkoeffizienten.},
	unit={\ensuremath{\si{}}},
	%symbol={\ensuremath{}},
	type=symbols,
	sort=daten922
}

\newglossaryentry{sy:weights}{
	name={\ensuremath{\alpha_{cos},\alpha_{sin}}},
	description={Regressionsgewichte des Regressionsmodells.},
	unit={\ensuremath{\si{}}},
	%symbol={\ensuremath{}},
	type=symbols,
	sort=daten923
}

\newglossaryentry{sy:lcos}{
	name={\ensuremath{\log p(y_{cos}|X_{cos})}},
	description={Modellplausibilität des Regressionsmodells für die Cosinus-Funktion.},
	unit={\ensuremath{\si{}}},
	%symbol={\ensuremath{}},
	type=symbols,
	sort=daten924
}

\newglossaryentry{sy:lsin}{
	name={\ensuremath{\log p(y_{sin}|X_{sin})}},
	description={Modellplausibilität des Regressionsmodells für die Sinus-Funktion.},
	unit={\ensuremath{\si{}}},
	%symbol={\ensuremath{}},
	type=symbols,
	sort=daten925
}

\newglossaryentry{sy:lbeide}{
	name={\ensuremath{\tilde{R}_{\mathcal{L}}}},
	description={Resultierendes Minimumkriterium aus den Modellplausibilitäten.},
	unit={\ensuremath{\si{}}},
	%symbol={\ensuremath{}},
	type=symbols,
	sort=daten926
}

\newglossaryentry{sy:kvec}{
	name={\ensuremath{\mathbf{k}_*}},
	description={Kovarianzvektor resultierend aus Trainingsdaten und einem Testdatensatz.},
	unit={\ensuremath{\si{}}},
	%symbol={\ensuremath{}},
	type=symbols,
	sort=daten927
}

\newglossaryentry{sy:fmean}{
	name={\ensuremath{\bar{f}_{cos*},\bar{f}_{sin*}}},
	description={Regressionsergebnisse für die Cosinus- und Sinus-Funktion.},
	unit={\ensuremath{\si{}}},
	%symbol={\ensuremath{}},
	type=symbols,
	sort=daten928
}

\newglossaryentry{sy:rmean}{
	name={\ensuremath{\bar{r}_*}},
	description={Abgeleiteter Radius aus den Regressionsergebnissen.},
	unit={\ensuremath{\si{}}},
	%symbol={\ensuremath{}},
	type=symbols,
	sort=daten929
}

\newglossaryentry{sy:angmean}{
	name={\ensuremath{\bar{\alpha}_*}},
	description={Abgeleiteter Winkel aus den Regressionsergebnissen.},
	unit={\ensuremath{\si{}}},
	%symbol={\ensuremath{}},
	type=symbols,
	sort=daten930
}

\newglossaryentry{sy:vpred}{
	name={\ensuremath{\mathbb{V}\left[\bar{f}_*\right]}},
	description={Varianz der Regressionsvorhersage.},
	unit={\ensuremath{\si{}}},
	%symbol={\ensuremath{}},
	type=symbols,
	sort=daten931
}

\newglossaryentry{sy:std}{
	name={\ensuremath{s_*}},
	description={Standardabweichung in der Regression.},
	unit={\ensuremath{\si{}}},
	%symbol={\ensuremath{}},
	type=symbols,
	sort=daten932
}

\newglossaryentry{sy:zcdf}{
	name={\ensuremath{z_{CDF}}},
	description={Faktor aus kumulativer Verteilungsfunktion für normalverteilte Zufallsvariablen.},
	unit={\ensuremath{\si{}}},
	%symbol={\ensuremath{}},
	type=symbols,
	sort=daten933
}

\newglossaryentry{sy:cia95}{
	name={\ensuremath{CIA_{95\%},CIR_{95\%}}},
	description={$95\%$-Konfidenzintervalle ($CI$) für regressierte Winkel ($A$) und Radien ($R$).},
	unit={\ensuremath{\si{}}},
	%symbol={\ensuremath{}},
	type=symbols,
	sort=daten934
}

\newglossaryentry{sy:loss}{
	name={\ensuremath{SLLA,SLLR}},
	description={Standardisierte logarithmische Modellverluste für regressierte Winkel ($A$) und Radien ($R$).},
	unit={\ensuremath{\si{}}},
	%symbol={\ensuremath{}},
	type=symbols,
	sort=daten935
}

\newglossaryentry{sy:mloss}{
	name={\ensuremath{MSLLA,MSLLR}},
	description={Mittelung der Modellverluste für regressierte Winkel ($A$) und Radien ($R$).},
	unit={\ensuremath{\si{}}},
	%symbol={\ensuremath{}},
	type=symbols,
	sort=daten936
}

\newglossaryentry{sy:fmod}{
	name={\ensuremath{f_m}},
	description={Modulationsfrequenz in der Erstellung von Kennfeldern.},
	unit={\ensuremath{\si{\hertz}}},
	%symbol={\ensuremath{}},
	type=symbols,
	sort=Kennfeld1
}

\newglossaryentry{sy:fcar}{
	name={\ensuremath{f_c}},
	description={Trägerfrequenz in der Erstellung von Kennfeldern.},
	unit={\ensuremath{\si{\hertz}}},
	%symbol={\ensuremath{}},
	type=symbols,
	sort=Kennfeld2
}

\newglossaryentry{sy:phase}{
	name={\ensuremath{\phi}},
	description={Phasenwinkel der Stimulanz für die Kennfelderzeugung.},
	unit={\ensuremath{\si{\radian}}},
	%symbol={\ensuremath{}},
	type=symbols,
	sort=Kennfeld3
}