% !TEX root = ../thesis.tex
%
% configurations
%

% text field
%-> replace supervisor names with correct ones
\firstSupervisor{Prof. Dr. Karl-Ragmar Riemschneider}
\secondSupervisor{Prof. Dr. Klaus Jünemann}

% text field
%-> replace title with your thesis title
\thesisTitle{Winkelmessung durch magnetische Sensor-Arrays und Toleranzkompensation mittels Gauß-Prozess}
\thesisTitleEN{Angular Measurement by Magnetic Sensor Arrays and Tolerance Compensation by Gaussian Process}

% text field
%-> replace the key words with your own key words
\keywordsDE{Sensor-Array Simulation, Dipol, Magnetfeld, Kugelmagnetapproximation, TMR, TDK TAS2141, AMR, NXP KMZ60, Toleranzkompensation, Gauß'sche Prozesse, Kovarianz, Kovarianzmatrix, Regression, Winkelvorhersage, Logarithmische Modellplausibilität, Standardisierter Logarithmischer Modellverlust, Minimierungsproblem, Optimierung, ASIC-Modell}

\keywordsEN{Sensor Array Simulation, Dipole, Magnetic Field, Sperical Magnet Approximation, TMR, TDK TAS2141, AMR, NXP KMZ60, Tolerance Compensation, Gaussian Processes, Covariance Covariance Matrix, Regression, Angular Prediction, Logarithmic Marginal Likelihood, Standardized Logarithmic Loss, Minimization, Optimization, ASIC Model}

% text field
%-> replace the text with a description of the thesis
\abstractDE{Die vorliegende Bacheloarbeit umfasst den Aufbau und die Verbesserung von Simulationssoftware und Modellen für die physikalische Simulation eines tunnel-magnetoresistiven Sensor-Arrays sowie die mathematische Simulation eines dazugehörigen auswertenden  ASIC-Modells. Das mathematische ASIC-Modell für die Auswertung der Sensor-Array-Simulationsdaten beruht auf einem Regressionsverfahren für Gauß'sche Prozesse. In Vorarbeiten sind Grundlagen zur physikalischen und mathematischen Simulation der Modelle geschaffen worden, die im Rahmen dieser Arbeit zu einer modular aufgebauten Software zusammengeführt werden. Des Weiteren wird das mathematische ASIC-Modell in Bezug auf Modellressourcen und maschinelle Lernfähigkeit weiterführend optimiert und verbessert sowie durch Algorithmen zur Modelloptimierung ergänzt. Abschließend werden in dieser Arbeit Experimente zur Funktionalität, Modelloptimierung und Fähigkeit zur Toleranzkompensation des Gesamtsystems durchgeführt.}

\abstractEN{This bachelor thesis covers the construction and improvement of simulation software and models for the physical simulation of a tunnel magnetoresistive Sensor Array as well as the mathematical simulation of an associated evaluating ASIC model. The mathematical ASIC model for the evaluation of the Sensor Srray simulation data is based on a regression method for Gaussian Processes. In preliminary work, basic principles for the physical and mathematical simulation of the models have been created, which will be combined into a modular software within the scope of this work. Furthermore, the mathematical ASIC model will be further optimized and improved with respect to model resources and machine learning capability, and supplemented by algorithms for model optimization. Finally, experiments on the functionality, model optimization and tolerance compensation capability of the overall system are carried out in this thesis.}

% text field
%-> replace jon with your name
\thesisAuthor{Tobias Wulf}

% text field
%-> enter the submission date
\submissionDate{08. Juni 2021}

% switch - uncomment only one
%-> uncomment NDA or public
%\NDA{yes}
\NDA{no}

% switch - uncomment only one
%-> uncomment cover or cover Corporate Design 2017
%\Cover{CD2017}
%\Cover{CD2017NoLogo}
\Cover{Std2018}

% switch - uncomment only one
%-> uncomment to show list of figures or not
\ListOfFigures{yes}
%\ListOfFigures{no}

% switch - uncomment only one
%-> uncomment to show list of tables or not
\ListOfTables{yes}
%\ListOfTables{no}

% switch - uncomment only one
%-> uncomment to show list of accronyms or not
\ListOfAccronyms{yes}
%\ListOfAccronyms{no}

% switch - uncomment only one
%-> uncomment to show list of symbols or not
\ListOfSymbols{yes}
%\ListOfSymbols{no}

% switch - uncomment only one
%-> uncomment to show list of glossary entries or not
\Glossary{yes}
%\Glossary{no}

% switch - uncomment only one
%-> uncomment the study course you are in
%\studycourse{ITS}
%\studycourse{TI}
%\studycourse{AI}
%\studycourse{WI}
\studycourse{EI}
%\studycourse{REE}
%\studycourse{BMT}
%\studycourse{MAI}
%\studycourse{MIK}
%\studycourse{MA}
