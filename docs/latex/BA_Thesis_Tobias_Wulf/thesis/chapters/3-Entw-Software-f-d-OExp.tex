% !TEX root = ../thesis.tex
% developping software for optimization experiments
% @author Tobias Wulf
%

\chapter{Entwicklung von Software für die Optimierungs-Experimente 0.0.1 13.01.2021}

\section{Aufgabe der Software und grundsätzliche Funktion}
\begin{itemize}
	\item Identifizierung der Grundfunktionen
	\item Datengenerierung
	\item Datenanalyse
	\item Sonderfunktion
	\item Darstellungs- und Plot-Funktionen
\end{itemize}

Die Software-Entwicklung erfolgt unter dem Gesichtspunkt zur Durchführung von Versuchsreihen zu 
Parameterfindung und teilweise auf Zwischenergebnissen basieren.
Gut strukturierte Archivierung von Ergebnisse.
Graphische Unterstützung von Auswertung.
 
\section{Aufbau und Vorgehen}
	\begin{itemize}
		\item Skriptbasierte Entwurfsarbeit
		\item Überführen in modularen Aufbau von Kernfunktion
		\item Parametrierte Steuerung der Software über Zentrale Konfigurierung
		\item Ausführbare Skripte (Einbindung von Modulen und nutzen der Konfigurierung)
		\item Speicherung von Ergebnissen in Datensätzen
		\item Versionierung der Arbeitsschritte
	\end{itemize}


\section{Sensor-Array-Simulation}
	\begin{itemize}
		\item Zuordnung Datengenerierung
		\item Nutzung von vorarbeiten
		\item Darstellung des Modul Funktionsablaufdiagramm
		\item Darstellung des Algorithmus für die Simulation mehrere Positionen
		\item Nutzung des Moduls für eingestellte Konfigurierung
	\end{itemize}

\section{Gauß-Prozess-Regression}
	\begin{itemize}
		\item Zuordnung Datenanalyse
		\item Nutzung von Vorarbeiten
		\item Einordnung der Vorarbeiten in Bezug auf Regressionsverfahren (Jünemann)
		\item Skriptbasierte Voruntersuchungen zu Findung des mathematischen Modells bzw. Kovarianzfunktion (Matlab-Standard-Modelle)
		\item Bezugherstellung Einheitskreis und Orthogonalität des Ausgangssystems
		\item Beschreibung des kombinierten Systems aus der Vorarbeit (Jünemann)
		\item Optimierung des einfachen kombinierten Systems ohne Mittelwertschätzung
		\item Optimierung des einfachen kombinierten Systems mit Mittelwertschätzung
		\item Optimierung des kombinierten System mit individueller Mittelwertschätzung
		\item Einbringen des Atan2-Feature-Funktion über die Mittelwertschätzung und vereinfachte Optimierung
		\item Darstellung der einzelnen Optimierungsverfahren und Aufzeigen der Unterschiede im vorgehen
		\item Bemessung des Aufwands und Genauigkeiten
		\item Beziffern und ermitteln von Hyperparameter für die vier Regressionsmöglichkeiten des kombinierten Systems
 		\item Nutzung des Moduls für eingestellte Konfigurierung
	\end{itemize}
