% !TEX root = ../thesis.tex
% introduction
% @author Tobias Wulf
%

\chapter{Motivation 0.0.1 17.02.2021}

Neuentwicklungen in der Halbleitertechnik, auf Basis des \gls{gl:tmr}s, ermöglichen den Aufbau komplexerer Sensorstrukturen \cite{Schuethe2019}. Die \gls{gl:ags} an der \gls{gl:haw} erforscht moderne Ansätze der Signalverarbeitung für neugewonnene Sensorstrukturen, verwirklicht als magnetische Sensor-Arrays. Durch den Aufbau von Sensoren als Arrays, bieten sich Möglichkeiten zur Nutzung von Algorithmen und Regressionsverfahren an, die eine Kompensation und Detektion von mechanische Toleranzen zulassen \cite{Schuethe2020}.
\newline
Das Verarbeiten einer Vielzahl an Messwerten, bedingt durch Sensor-Array-Strukturen, ist hierbei eine der Herausforderungen die es zu bewältigen gilt. Mit Hilfe moderner Algorithmen, die Ansätze des maschinellen Lernens beinhalten, ergeben sich weitere Problemstellungen in Bezug auf Modellabbildung- und Optimierung.
Das übergeordnete Ziel bei der Lösung und Bewältigungen der einzelnen Etappen ist die Verbesserung der Messgenauigkeit, indem individuelle Abweichungen des Sensors einem geeigneten Modell antrainiert und Modellparameter optimiert werden. Moderne Regressionsverfahren liefern dabei statistische Ansätze um geeignete Qualitätskriterien zu bilden und somit trainierte Modelle und ihre Messwertgenauigkeit zu bewerten.

\input{chapters/1-1-Zielstellung}