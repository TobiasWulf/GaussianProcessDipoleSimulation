% !TEX root = ../thesis.tex
% basics
% @author Tobias Wulf
%

\chapter{Grundlagen 0.0.1 13.01.2021}
	\begin{itemize}
		\item Einleitung Aufgabenfeld
	\end{itemize}

\section{Magnetische Sensorentypen und mechatronische Anwendung}
	\begin{itemize}
		\item Anwendungsfall Drehzahlmessung
		\item Anwendungsfall Winkelmessung
		\item Ausblick TMR und Strommessung
		\item Aufbau Sensorbrücke TMR (Umriss)
	\end{itemize}

\section{Kennfeldmethode zur Modellierung der Sensoren}
	\begin{itemize}
		\item Messprinzip für das Erstellen der Brücken Kennfelder
		\item Festlegung von Arbeitsbereich (Plateau TMR), Sättigung (KMZ60)
		\item Dimensionierung des Stimulus, Dipole Anregung
	\end{itemize}

\section{Prinzip des Sensor Arrays}
	\begin{itemize}
		\item geometrischer Aufbau
		\item Brückenausgangsspannungen
		\item Resultierende Array-Datenformate und Darstellung der Sinoiden
	\end{itemize}

\section{Simulation über Dipol-Feldgleichung}
	\begin{itemize}
		\item Erzeugen des Meshgrids
		\item Normieren des Magnetfeldes
		\item Erzeugen von Rotationsmomenten (inkl. Verkippung)
		\item Referenzierung zu Kennfeldern und Gewinnung der Brückenspannungen (interp2 nearest neighbor)
	\end{itemize}

\section{Gauß-Prozess und Regressionsverfahren}
	\begin{itemize}
		\item Erläuterung des Regressionsverfahren im allg.
		\item Bedeutung der Kovarianzfunktion
		\item Möglichkeiten zur Mittelwertschätzung und -Korrektur
		\item Einbringen von Feature-Funktionen über die Mittelwertschätzung
	\end{itemize}