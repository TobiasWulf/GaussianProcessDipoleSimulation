% !TEX root = ../thesis.tex
% basics
% @author Tobias Wulf
%

\chapter{Grundlagen 0.0.2 19.02.2021}\label{ch:grundlagen}
	\begin{itemize}
		\item Einleitung Aufgabenfeld
		\item Einheitskreis
		\item Bezug zur Drehwinkelerfassung und Sensorapplikation
	\end{itemize}

\section{Magnetische Sensorentypen und mechatronische Anwendung}\label{sec:magnetische-sensorentypen}
	\begin{itemize}
		\item Die Technologie mit der ein Sensorkopf realisiert ist, klassifiziert in der Regel die Sensorbezeichnung. Anhänge in der Bezeichnung wie AMR oder TMR, geben somit Auskunft darüber welche Technologie für die Realisierung des Sensorkopfes die Grundlage bildet.
		\item Anwendungsfall Winkelmessung
		\item Aufbau Sensorbrücke TMR (Umriss aus Datenblatt)
		\item Ausblick TMR Drehzahlmessung und Strommessung
	\end{itemize}

\section{Kennfeldmethode zur Charakterisierung von Sensoren}\label{sec:kennfeldmethode-zur-charakterisierung}
	\begin{itemize}
		\item Überleitung von Sensorbrückenschaltung
		\item Messprinzip für das Erstellen der Sensorbrücken-Kennfelder
		\item Festlegung von Arbeitsbereich (Plateau TMR), Sättigung (KMZ60)
		\item Dimensionierung des Stimulus, Dipole Anregung
	\end{itemize}

\section{Prinzip des Sensor-Arrays}\label{sec:prinzip-des-sensor-arrays}
	\begin{itemize}
		\item geometrischer Aufbau
		\item Brückenausgangsspannungen
		\item Resultierende Array-Datenformate und Darstellung der Sinoiden
	\end{itemize}

\section{Sensor-Array-Simulation über Dipol-Feldgleichung}\label{sec:sensor-array-simulation-dipol-feldgleichung}
	\begin{itemize}
		\item Erzeugen des Meshgrids
		\item Normieren des Magnetfeldes
		\item Erzeugen von Rotationsmomenten (inkl. Verkippung)
		\item Referenzierung zu Kennfeldern und Gewinnung der Brückenspannungen (interp2 nearest neighbor)
	\end{itemize}

\section{Gauß-Prozesse für Regressionsverfahren}\label{sec:gauss-prozesse-regressionsverfahren}
	\begin{itemize}
		\item Erläuterung des Regressionsverfahren im allg.
		\item Bedeutung und Kriterien der Kovarianzfunktion, Spiegel der Applikation
		\item Herleitung der Quadratischen Frobenius Kovarianzfunktion mit Bezug zum Einheitskreis
		\item Möglichkeiten zur Mittelwertschätzung und -Korrektur
		\item Optimierungskriterien in der Trainingsphase
		\item Qualitätskriterien in der Arbeitsphase
	\end{itemize}