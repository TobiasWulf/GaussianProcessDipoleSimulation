% !TEX root = ../thesis.tex
% sensor array simulation with dipole field equation
% @author Tobias Wulf
%

\section{Sensor-Array-Simulation über Dipol-Feldgleichung}\label{sec:sensor-array-simulation-dipol-feldgleichung}


Die Sensor-Array-Simulation nutzt einen Kugelmagneten als Stimulanz \cite{Schuethe2019}. Es ist die Anwendungsbeschreibung aus \autoref{fig:sensor-array-prinzip} gewählt. Der Vorteil darin liegt, dass ein Kugelmagnetfernfeld mittels der Feldgleichung für einen magnetischen Dipol approximiert werden kann \cite{Pape2017}. Das innere Magnetfeld des Kugelmagneten ist dabei zu vernachlässigen. Der Radius des Kugelmagneten $r_{mag}$ ist als Offset für den räumlichen Abstand zum Magneten zu verwenden. Weitere physikalische Effekte wie magnetische Remanenz werden vernachlässigt.



\begin{figure}[ph]
	\centering
	\includegraphics[width=0.7\linewidth]{chapters/images/2-Grundlagen/Dipol-Feldgleichung}
	\caption[Simulation der Dipol-Feldgleichung]{Simulation der Dipol-Feldgleichung. Als veranschaulichendes Beispiel ist ein Magnetfeld über die Dipol-Feldgleichung simuliert. Der Dipol bildet den Koordinatenursprung bei $\vec{r} = (0,0,0)^T$. In Relation zum Dipol ist ein Meshgrid, der einzelnen Sensor-Pixel-Positionen $\vec{r}$, unterhalb des Dipols gelegt. Abhängig vom magnetischen Moment $\vec{m}$ des Dipols wird an jeder Meshgrid-Position $\vec{r}$ die Dipol-Feldgleichung gelöst und punktuell die magnetische Feldstärke $\vec{H}$ berechnet. Das magnetische Moment $\vec{m}$ bestimmt die Nord-Süd-Ausrichtung und Winkelstellungen des Dipols.}
	\label{fig:dipol-feldgleichung}
\end{figure}


\clearpage


Es wird ein Meshgrid für die einzelnen Sensor-Pixel nach \autoref{eq:arraymeshgrid} in ein dreidimensionale Koordinatensystem gelegt. Wie in \autoref{fig:dipol-feldgleichung} zu sehen, liegt der Dipol im Koordinatenursprung bei $\vec{r} = (0,0,0)^T$. Die einzelnen Pixel-Position $\vec{r}$ und das magnetische Dipol-Moment $\vec{m}$, sind durch \autoref{eq:posmom} beschrieben. Das Dipol-Moment $\vec{m}$ bestimmt die räumliche Ausrichtung des Magnetfeldes.


\begin{equation}\label{eq:posmom}
\vec{r} = \hat{r} \cdot |\vec{r}| = \begin{pmatrix} x \\ y \\ z \end{pmatrix} \qquad \vec{m} = \begin{pmatrix} m_x \\ m_y \\ m_z \end{pmatrix}
\end{equation}


Die Feldstärken $\vec{H}$, an den jeweiligen Pixel-Positionen $\vec{r}$, sind für das aktuelle Moment $\vec{m}$ nach \autoref{eq:dipfg} zu berechnen. \autoref{eq:dipfg} ist durch einsetzen von \autoref{eq:posmom} zu \autoref{eq:dipfgopt} vereinfacht. Somit ist die Feldstärke $\vec{H}$, mit dazugehörigen Moment $\vec{m}$, nur von der Richtung des Einheitsvektors $\hat{r}$ abhängig und durch 
$\frac{1}{4\pi|\vec{r}|^3}$ skaliert. Die entsprechende Betragsfeldstärke $|\vec{H}|$ setzt sich aus den resultierenden Feldstärkenkomponenten in \autoref{eq:hbetrag} zusammen.


\begin{align}
	\vec{H}(\vec{r},\vec{m}) &= \frac{1}{4\pi} \cdot \Bigg(\frac{3\vec{r} \cdot \big(\vec{m}^T \cdot \vec{r}\big)}{|\vec{r}|^5} - \frac{\vec{m}}{|\vec{r}|^3}\Bigg) \label{eq:dipfg} \\
							 \nonumber \\
							 &= \frac{1}{4\pi|\vec{r}|^3} \cdot \Big(3\hat{r} \cdot \big(\vec{m}^T \cdot \hat{r}\big) - \vec{m}\Big) = \begin{pmatrix} H_x \\ H_y \\ H_z \end{pmatrix} \label{eq:dipfgopt} \\
						     \nonumber \\
				   |\vec{H}| &= \sqrt{ H_x^2 + H_y^2 + H_z^2 } \label{eq:hbetrag}
\end{align}


\clearpage


Um eine Rotation und etwaige Verkippungen des Dipol-Magnetfeldes im Raum zu erwirken, müssen nach \autoref{eq:allgrot} entsprechende axiale Rotationen in $X,Y$ und $Z$, durch aufschalten von Drehmatrizen hergestellt werden. Durch drehen bzw. verkippen des Dipol-Momentes $\vec{m}$ nach \autoref{eq:allgrot}, ergibt sich das neue Dipol-Moment $\vec{m}'$ und somit weiter resultierende Feldstärken $\vec{H}'$ bei gleichbleibenden Pixel-Positionen $\vec{r}$ \cite{Schuethe2019}.


\begin{equation}\label{eq:allgrot}
\resizebox{.9\linewidth}{!}{$%
	\underbrace{
		\begin{pmatrix}
			 m_x' \\
			 m_y' \\
			 m_z' 
		\end{pmatrix}
	}_{\vec{m}'}  =
	\underbrace{
		\begin{pmatrix}
			\cos\alpha_z & -\sin\alpha_z & 0 \\
			\sin\alpha_z &  \cos\alpha_z & 0 \\
			0			 & 0			 & 1
		\end{pmatrix}
	}_{R_z(\alpha_z)}
	\underbrace{
		\begin{pmatrix}
			\cos\alpha_y  & 0 & \sin\alpha_y \\
			0 		      & 1 & 0 \\
			-\sin\alpha_y & 0 & \cos\alpha_y
		\end{pmatrix}
	}_{R_y(\alpha_y)}
	\underbrace{
		\begin{pmatrix}
			1 & 0 			 & 0 \\
			0 & \cos\alpha_x & -\sin\alpha_x \\
			0 & \sin\alpha_x &  \cos\alpha_x 
		\end{pmatrix}
	}_{R_x(\alpha_x)}
	\underbrace{
		\begin{pmatrix}
			m_x \\
			m_y \\
			m_z
		\end{pmatrix}
	}_{\vec{m}}$%
}
\end{equation}


Für den Standardanwendungsfall aus \autoref{sec:prinzip-des-sensor-arrays}, also Rotation in Magnet-$Z$-Achse ohne Verkippung, sind die Drehmatrizen $R_x$ und $R_y$ auszuschalten. Das kann durch zu Null setzen der Verkippungswinkel $\alpha_x$ und $\alpha_y$ erreicht werden. Die Drehmatrizen sind dadurch zur Einheitsmatrix $I$ gleichgeschaltet. Es ergeben sich somit $i-te$ Dipol-Rotationsmomente $\vec{m}_i$, für $i-te$ Winkelstellungen des Magneten $\alpha_i$ mit $\alpha_i\in\{\SI{0}{\degree},\ldots,\SI{360}{\degree}\}$, nach \autoref{eq:zrotov}. Dabei ist $\vec{m}_0 = -(m_0,0,0)^T$ das Startmoment und legt die Nord-Süd-Ausrichtung des Gebermagneten zu Beginn in seine $X$-Achse. Für Rotationen mit konstanten Verkippungen sind die Verkippungswinkel $\alpha_x \ne 0$ bzw. $\alpha_y \ne 0$ zu setzen und allgemein nach \autoref{eq:allgrot} zu berechnen. 


\begin{equation}\label{eq:zrotov}
\vec{m}_i(\alpha_i)  = R_z(\alpha_i) \cdot I \cdot I \cdot \vec{m_0}  \qquad \textrm{f. } \vec{m}_0 = -\begin{pmatrix} m_0\\0\\0 \end{pmatrix}
\end{equation}


Als Anfangswert für das Startmoment empfiehlt sich $m_0 > \SI{1000}{\ampere\square\metre}$ zu wählen, dass unterdrückt numerische Fehler beim Berechnen der Feldstärke $\vec{H}$. In einem weiteren Normierungsschritt zum Aufprägen einer Betragsfeldstärke $H_{mag}$, bei definierten Abstand $r_{mag} + d_z$ zur Magnetenoberfläche, löscht sich der hohe Anfangswert für $m_0$ rechnerisch aus. Sodass über das Dipol-Moment nur die Ausrichtung des Gebermagneten gesteuert ist und kein nominaler Einfluss auf errechnete Feldstärken $\vec{H}$ besteht.


\clearpage


Damit in der Sensor-Array-Simulation magnetische Anregungen erzeugt werden können, die den empfohlenen Kennfeldarbeitsbereich aus \autoref{ch:tdk-datensatz} treffen, ist es notwendig das approximierte Kugelmagnetfeld in einem weiteren zu manipulieren. Die Manipulation des Magnetfeldes erfolgt, durch das Aufprägen einer Betragsfeldstärke $H_{mag}$, für die Ruhelage des Magneten mit dazugehöriger Feldstärke $\vec{H}_0$. Dabei ist ein definierter Abstand zur Kugelmagnetoberfläche festzulegen, bei dem sich die aufzuprägende Betragsfeldstärke $H_{mag}$ einstellt. 


\begin{align}\label{eq:ruhepos}
	\vec{r}_0(\alpha_1,\alpha_y,\alpha_x) &= R_z(\alpha_1) \cdot R_y(\alpha_y) \cdot R_x(\alpha_x) \cdot \big(0,0,-(r_{mag} + d_z)\big)^T \\
	\label{eq:ruhemom}
	\vec{m}_0(\alpha_1,\alpha_y,\alpha_x) &= R_z(\alpha_1) \cdot R_y(\alpha_y) \cdot R_x(\alpha_x) \cdot \big(-m_0,0,0\big)^T
\end{align}


Die Ruhelage bezieht sich auf den Startwinkel $\alpha_1$ der Simulation und ist gemäß gewünschter Verkippungen axial getreu einzustellen. Die Normierungsposition ist mit \autoref{eq:ruhepos} vorgegeben und definiert den Abstand entlang der Magnet-$Z$-Achse und zur Magnetoberfläche. Der Kugelmagnet ist mit entsprechenden Ruhemoment, der Normierungsposition folgend, nach \autoref{eq:ruhemom} auszurichten. Anschließend ist die Betragsfeldstärke $|\vec{H}_0(\vec{r}_0,\vec{m}_0)|$ auszurechnen. Über den Quotient, aus gewünschter Prägung $H_{mag}$ und Betrag $|\vec{H}_0(\vec{r}_0,\vec{m}_0)|$ in Ruhelage, mündet die Berechnung für ein normiertes Kugelmagnetfeld $H_{Norm}(\vec{r},\vec{m}_i)$, für beliebige Positionen $\vec{r}$ im Koordinatenraum und $i-te$ Rotationsmomente $\vec{m}_i$ in \autoref{eq:dipnorm}.


\begin{equation}\label{eq:dipnorm}
	\vec{H}_{Norm}(\vec{r},\vec{m}_i) = \vec{H}(\vec{r},\vec{m}_i) \cdot \frac{H_{mag}}{|\vec{H}_0(\vec{r}_0,\vec{m}_0)|}
\end{equation}


Die Anwendungskonfigurierung für eine optimale Simulation und treffen der Arbeitsbereiche ist \autoref{ch:sensor-array-sim-imp} zu entnehmen. Simulierte Feldstärken sind gemäß der Meshgrid-Anordnung in Matrizen zu speichern, sodass sich in \autoref{sec:prinzip-des-sensor-arrays} beschriebene, Array-Datenformate ergeben. Diese können im Simulationsverlauf fortführend, direkt im Array-Format auf die Kennfelder, zur Entnahme von korrespondierenden Spannungsausgaben angewandt und gespeichert werden. In der Sensor-Array-Simulation ist die Verkippung in $X$-Achse deaktiviert mit $\alpha_x = 0$. Im weiteren Kontext bezieht sich der Begriff Verkippung (engl. ``tilt''), ausschließlich auf Verkippungen in der $Y$-Achse des Gebermagneten.


\clearpage

