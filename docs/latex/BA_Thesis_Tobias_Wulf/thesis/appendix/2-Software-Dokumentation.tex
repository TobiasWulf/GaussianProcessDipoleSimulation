% !TEX root = ../thesis.tex
% appendix software documentation (autogenerated)
% @author Tobias Wulf
%
\chapter{Software-Dokumentation 0.0.6 16.04.2021}\label{ch:sw-doku}

Die Software-Dokumentation ist automatisiert mit MATLAB-Skripten erstellt worden. Es ist dafür ein zweistufiger Prozess implementiert, der im ersten Schritt eine in MATLAB integrierte HTML-Dokumentation erstellt. Im Anschluss ist diese in Tex-Dateien. Als letzter Schritt sind diese zu einem LaTeX-Manualzusammengefasst im Anhang eingebunden. Mit diesem Verfahren ist es möglich, eine Dokumentation direkt aus geschriebenen M-Dateien zu generieren. Allerdings ist es dafür nötig, eine spezielle Formatierung und einen gewissen Programmierstil einzuhalten \cite{Johnson2014}. Die Dokumentation enthält neben dem erstellten Quellcode eine Reihe von Arbeitsanweisungen, wie mit der Software umzugehen ist. Zusätzlich sind Beschreibungen für die Erstellung und Pflege des Software-Projektes mit beigefügt. Die geschriebene Software ist mithilfe des Software-Versionierungsprogramms Git erstellt worden, was eine genaue Nachvollziehbarkeit in Bezug auf die einzelnen Arbeitsschritte ermöglicht. Zur Versionierung ist der Git-Feature-Branch-Workflow \cite{Bitbucket2020}
angewandt worden. Aus stilistischen Gründen ist die gesamte Software-Dokumentation in Englisch verfasst. Die Software-Dokumentation ist automatisiert durch eigens dafür geschriebene Skripte erstellt worden. Diese sind in der Dokumentation enthalten.


\import{../../Manual/}{Manual.tex}
