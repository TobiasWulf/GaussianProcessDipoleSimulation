% !TEX root = ../thesis.tex
% parametrize experiments in chapter 4
% @author Tobias Wulf
%

\chapter{Parametrierung der Erprobungs- und Optimierungsexperimente}\label{ch:param-exp}






\section{Anpassung des Rauschniveaus}\label{sec:paramexp3}


Abweichende Parameter:

\begin{itemize}
	\item TrainingsOptions: nAngles: variieren und werden schrittweise erhöht
	\item GRPOptions: kernel : 'QFCAPX'
	\item GPROptions: mean: variiert zwischen 'zero' und 'poly'
\end{itemize}


\section{Anpassung der Parametergrenzen}\label{sec:paramexp4}


Abweichende Parameter:

\begin{itemize}
	\item TrainingsOptions: nAngles: 17
	\item GRPOptions: kernel : 'QFCAPX'
	\item $\sigma_f^2$-Bounds: $(1,10)$
	\item $\sigma_l$-Bounds: $(10,30)$
	\item $\sigma_n^2$-Bounds: $(10^{-6},10^{-4})$
	\item GPROptions: mean: 'zero'
	\item OptimRuns 10
\end{itemize}


\section{Verhalten bei einfachen Fehllagen}\label{sec:paramexp5}


Abweichende Parameter:

Referenzposition (0,0,7.5)

\begin{itemize}
	\item TrainingsOptions: nAngles: 17
	\item TrainingsOptions/ TestOptions: xPos/ yPos: $-3:0,25:3$
	\item TrainingsOptions/ TestOptions: zPos: $4,5:0,25:10,5$
	\item TrainingsOptions/ TestOptions: tilt: $0:0,5:12$
	\item GRPOptions: kernel : 'QFCAPX'
	\item $\sigma_f^2$-Bounds: $(0.1,100)$
	\item $\sigma_l$-Bounds: $(1,100)$
	\item $\sigma_n^2$-Bounds: $(10^{-7},10^{-3})$
	\item GPROptions: mean: 'zero'
	\item OptimRuns 30
\end{itemize}


\section{Verhalten bei kombinierten Fehllagen}\label{sec:paramexp6}


Abweichende Parameter:

Referenzposition (0,0,4.5) , 11deg 

\begin{itemize}
	\item TrainingsOptions: nAngles: 17
	\item TrainingsOptions/ TestOptions: xPos/ yPos: $(0;0)/(0,5;1)/(2,5;2)$
	\item TrainingsOptions/ TestOptions: zPos: $4,5$
	\item TrainingsOptions/ TestOptions: tilt: $11$
	\item GRPOptions: kernel : 'QFCAPX'
	\item $\sigma_f^2$-Bounds: $(0.4,20)$
	\item $\sigma_l$-Bounds: $(4,50)$
	\item $\sigma_n^2$-Bounds: $(3\cdot10^{-7},10^{-5})$
	\item GPROptions: mean: 'zero'
	\item OptimRuns 10
\end{itemize}

