% !TEX root = ../thesis.tex
% parametrize experiments in chapter 4
% @author Tobias Wulf
%

\chapter{Parametrierung der Erprobungs- und Optimierungsexperimente 0.0.1 26.04.2021}\label{ch:param-exp}


\section{Vergleich der Kovarianzfunktionen}\label{sec:vgl-kfun-param}



\begin{table}[!htbp]
	\centering
	\resizebox{\textwidth}{!}{
		\begin{tabular}{l l c c l}
			\toprule
			\textbf{Parametergruppe}                 & \textbf{Parameter}  & \textbf{Wert}                        & \textbf{Einheit}                & \textbf{Kurzbeschreibung}                                           \\ \midrule
			\multirow{6}{*}{SensorArrayOptions}      & geometry            & 'square'                             & -                               & Array-Geometrie-Indikator                                           \\
			                                         & dimension           & $8$                                  & -                               & Sensor-Array-Pixel $N_{Pixel} \times N_{Pixel}$                     \\
			                                         & edge                & $2$                                  & $\SI{}{\milli\metre}$           & Sensor-Array-Kantenlänge                                            \\
			                                         & $V_{cc}$            & $5$                                  & $\SI{}{\volt}$                  & Sensor-Array-Betriebsspannung                                       \\
			                                         & $V_{off}$           & $2,5$                                & $\SI{}{\volt}$                  & Sensor-Brücken-Offset-Spannung                                      \\
			                                         & $V_{norm}$          & $1 \cdot 10^3$                       & $\SI{}{\milli\volt}$            & Kennfeldnormierung                                                  \\ \hline
			\multirow{4}{*}{DipoleOptions}           & sphereRadius        & $2$                                  & $\SI{}{\milli\metre}$           & Kugelmagnetradius                                                   \\
			                                         & $H_{0mag}$          & $200$                                & $\SI{}{\kilo\ampere\per\metre}$ & Betragsfeldstärke Magnetfeldnormierung                              \\
			                                         & $z_0$               & $1$                                  & $\SI{}{\milli\metre}$           & $Z$-Abstand Magnetfeldnormierung                                    \\
			                                         & $m_{0mag}$          & $1 \cdot 10^6$                       & $\SI{}{\ampere\square\metre}$   & Magnitude d. mag. Moments                                           \\ \hline
			\multirow{10}{*}{Training-/ TestOptions} & useCase             & 'Training'/ 'Test'                   & 'char'                          & Datensatzindikator f. Anwendungszweck                               \\
			                                         & xPos                & $\left[0,\right]$                    & $\SI{}{mm}$                     & Sensor-Array $X$-Positionsvektor                                    \\
			                                         & yPos                & $\left[0,\right]$                    & $\SI{}{mm}$                     & Sensor-Array $Y$-Positionsvektor                                    \\
			                                         & zPos                & $\left[7,\right]$                    & $\SI{}{mm}$                     & Sensor-Array $Z$-Positionsvektor                                    \\
			                                         & tilt                & $0$                                  & $\SI{}{\degree}$                & Magnetverkippung in $Y$-Achse                                       \\
			                                         & angleRes            & $0,5$                                & $\SI{}{\degree}$                & Winkelauflösung f. Magnetrotation                                   \\
			                                         & phaseIndex          & 0                                    & -                               & Phasenverschiebung-Index f. Startwinkel                             \\
			                                         & nAngles             & $56$/ $720$                          & -                               & Anzahl gleich verteilter Simulationswinkel                          \\
			                                         & BaseReference       & 'TDK'                                & char                            & Kennfelddatensatzindikator                                          \\
			                                         & BridgeReference     & 'Rise'                               & char                            & Kennfeldindikator                                                   \\ \hline
			\multirow{10}{*}{GPROptions}             & kernel              & 'QFC'                                & char                            & Kernel-Funktion-Indikator \eqref{eq:kfun}, 'QFC' $\leftarrow d_F^2$ \\
			                                         & $\theta$            & $(1,1)$                              & -                               & Kernel-Parametervektor $\theta$ \eqref{eq:kparam}                   \\
			                                         & $\sigma_f^2$-Bounds & $(0.1, 100)$                         & -                               & Parameter-Bounds $\theta_1$ f. \autoref{alg:fminconopt}             \\
			                                         & $\sigma_l$-Bounds   & $(0.1, 100)$                         & -                               & Parameter-Bounds $\theta_2$ f. \autoref{alg:fminconopt}             \\
			                                         & $\sigma_n^2$        & $1 \cdot 10^{-6}$                    & -                               & Rauschniveau, Rauschaufschaltung \eqref{eq:addnoise}                \\
			                                         & $\sigma_n^2$-Bounds & $(1 \cdot 10^{-8}, 1 \cdot 10^{-4})$ & -                               & Parameter-Bounds $\sigma_n^2$ f. \autoref{alg:bayesopt}             \\
			                                         & OptimRuns           & $30$                                 & -                               & Durchlaufanzahl f. \autoref{alg:bayesopt}                           \\
			                                         & SLL                 & 'SLLA'                               & char                            & Verlust-Indikator f. Winkel (A)/ R (Radius) \autoref{alg:bayesopt}  \\
			                                         & mean                & 'zero'                               & char                            & Indikator Mittelwertpolynom Ein ('poly')/ Aus ('zero')              \\
			                                         & polyDegree          & $1$                                  & -                               & Grad des Mittelwertpolynoms wenn mean = 'poly'                      \\ \bottomrule
		\end{tabular}}
	\caption[Simulationsparameter für das Vergleichsexperiment der Kovarianzfunktionen]{Simulationsparameter für das Vergleichsexperiment der Kovarianzfunktionen. Während der Durchführung des Experiments werden die Simulationsparameter $\theta$ und kernel angepasst um Funktionsverhalten zu vergleichen.}
	\label{tab:sim-params-exp-kfun}
\end{table}
